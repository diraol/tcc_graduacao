\chapter{Introdução}\label{chp:Introdução}
\lettrine{A}{mobilidade urbana} é hoje um dos mais latentes problemas das grandes cidades de todo o mundo. Em 2010, considerando 439 áreas urbanas nos Estados Unidos, os congestionamentos fizeram com que os motoristas gastassem 4,8 bilhões de horas e comprassem 7.2 bilhões de litros de combustível além do necessário, a um custo de \$101 bilhões de dólares \cite{Eisele2011}. Já em 2011, esses valores subiram para 5,5 bilhões de horas, 11 bilhões de litros de combustível e \$121 bilhões de dólares \cite{Schrank2012}. 
Assim, percebe-se que não apenas são altos os custos sociais, ambientais e econômicos, como também estes são crescentes.

A realidade brasileira não é muito diferente. De 2004 a 2007 o tempo de congestionamento de 4 das maiores cidades brasileiras (São Paulo, Rio de Janeiro, Belo Horizonte e Porto Alegre) cresceu a uma taxa anual média de 16\% \cite{resende2009}. Se considerarmos ainda as políticas públicas de incentivo ao mercado automobilístico, mais especificamente a redução do \gls{ipi} para automóveis, implementadas em maio de 2012 \cite{brasil2011}, e ainda vigentes~\cite{brasil2012}, pode-se esperar que o crescimento dos congestionamentos no período tenha sido até maior.

Para além dos congestionamentos, existem ainda outras externalidades comuns na área do transporte, como a poluição e os acidentes de trânsito \cite{vasconcellos1998}. A poluição, por exemplo, implica graves problemas de saúde pública, como o aumento da incidência de problemas cardiopulmonares, como bronquite e asma \cite{kunzli2000}, e o aumento dos índices de mortalidade \cite{finkelstein2004}.
Se essas externalidades possuem suas medidas de desempenho já consagradas e já se estuda sobre as deseconomias que as causam, a externalidade relativa à insatisfação do usuário do transporte público é terreno ainda pouco explorado na área de desenvolvimento de software, mesmo sobrando evidências de que há questionamentos quanto à qualidade dos serviços prestados \cite{UrbanaPE2010,Rodrigues,Rodrigues2006,Cellos2012}. Segundo \citeonline{cardoso2013}, o reajuste das tarifas de ônibus foi o estopim das manifestações de junho de 2013, porém as insatisfações não se limitavam a isso, mas permeavam também a qualidade do serviço prestado como um todo. Bastante polêmicas e controversas, mesmo assim ainda podendo ser entendidas sob esse prisma, estão diversas queimas de ônibus como forma de protesto contra a má-qualidade do sistema. \citeonline{fora1999} apontam como uma das recomendações para se combater as deseconomias do transporte público haver financiamentos cujos produtos visem uma melhor caracterização das ``condições atuais de transporte e trânsito para subsidiar projetos de melhoria (dos mesmos)''.

Outro ponto da atualidade a ser observado, é a cada vez mais concreta ubiquidade da internet e a massificação dos dispositivos eletrônicos que a acessam. Junto a estas, tem-se fortificado a cultura de criação e disseminação de informações por aqueles que antes eram apenas meros receptores, conforme coloca \citeonline[p.27]{guzzi2010web}.
\begin{citacao}
não é difícil entender porque se multiplica incessantemente o número de pessoas que compartilham as formas de comunicação móveis e instantâneas como um hábito diário, em qualquer lugar e a qualquer hora. Ou seja, as pessoas não só conversam mais, e, portanto, têm maior aproximação entre si, como participam mais e, de certa forma, ampliam o espaço público. Vemos nitidamente como a noção de inteligência coletiva, por exemplo, que até pouco tempo era motivo de estudos de poucos, se irradia e ganha visibilidade como prática comunitária corriqueira on-line e off-line.
\end{citacao}

Essa ``massificação instantânea'' permite que sejam levantadas as opiniões de uma quantidade de usuários impraticável para pesquisas de opinião tradicionais. Há de se destacar que é comum o questionamento à ``opinião da maioria'', sob a alegação de ``ignorância'' e que a consulta a pessoas com maior escolaridade e/ou formação traria melhores resultados. Porém, conforme coloca \apudonline[p.61]{surowiecki2006sabedoria}{primo2008fases} ``sob as circunstâncias corretas, grupos são impressionantemente inteligentes, e frequentemente são mais inteligentes que a pessoa mais inteligente em seu interior'', e \apudonline[p.61]{levy1999inteligencia}{primo2008fases} ainda destaca, ao falar sobre inteligência coletiva: ``É uma inteligência distribuída por toda a parte, incessantemente valorizada, coordenada em tempo real, que resulta em uma mobilização efetiva das competências''.

Sob a ótica desta inteligência coletiva, e com a utilização de recursos tecnológicos, vê-se um crescente incentivo a ``aplicativos sociais'', como o ``Desafio \textit{Desarrollando América Latina} (DAL)''\footnote{\textit{Desarrollando América Latina} - \url{Desarrollando América Latina} - Acesso em 01/12/2014}, o concurso ``Rio Ideias''\footnote{Rio Ideias - \url{http://ideias.rioapps.com.br/} - Acesso em 01/12/2014}, o ``I Desafio de Impacto Social Google Brasil''\footnote{Desafio de Impacto Social Google Brasil - \url{https://desafiosocial.withgoogle.com/brazil2014} - Acesso em 01/12/2014} e o ``Hackathon de Gênero e Cidadania''\footnote{Hackahton de Gênero e Cidadania - \url{http://edemocracia.camara.gov.br/web/hackathon-de-genero-e-cidadania} - Acesso em 01/12/2014} da Câmara dos Deputados. Os aplicativos sociais possuem como objetivo a transformação da sociedade, por meio da adesão massiva de usuários, seja para arrecadar fundos para uma causa, seja para buscar a melhor solução para um problema, ou mesmo para denunciar irregularidades.

É neste contexto de grandes mudanças que este projeto encontra sua motivação e respaldo. Hoje, graças à emergência de recursos tecnológicos de comunicações que utilizam conceitos como redes, geolocalização, comunicação multidirecional e outros, pode-se definir processos que unam com facilidade os extremos das cadeias de transporte e, assim, permitir um melhor planejamento e correções de rumo mesmo durante a execução do planejamento do sistema de transporte, melhorando a sua eficiência tanto para o poder público, quanto para usuários e prestadores de serviço.

\section{Justificativa do projeto}\label{sec:justificativa}
Atualmente existem diversos aplicativos que trabalham, majoritariamente, disponibilizando informações fornecidas pela \sptrans, por meio de sua \gls{api} para desenvolvedores\footnote{API da \sptrans para desenvolvedores: \url{http://www.sptrans.com.br/desenvolvedores/APIOlhoVivo/Documentacao.aspx} Acesso em 01/04/2014}, aos usuários. 
Isso se traduz em serviços de escolha de linha de ônibus, determinação de rotas e verificação de localização de ônibus com respectivo tempo estimado de chegada, e em alguns casos tempo estimado para se chegar ao destino escolhido.

Entretanto, não foram encontrados aplicativos que permitissem a coleta de avaliação dos usuários sobre elementos-chaves do sistema que não são automaticamente mensuráveis. Assim, o \trilhasp~se propõe a suprir essa demanda captando a opinião do usuário no momento do uso do transporte, permitindo a geração de informações de \textit{feedback} úteis tanto para usuários quanto para a \sptrans, enquanto fiscalizadora do sistema, e para as concessionárias do sistema de transporte, permitindo um melhor controle da qualidade da operação.

\section{Objetivos}\label{sec:objetivos}
\subsection{Geral}\label{subsec:objGerais}
O objetivo geral do projeto é desenvolver uma aplicação que corrobore para a melhoria da mobilidade nas cidades pelo incremento da qualidade e da eficiência do sistema público de transporte urbano, mais especificamente o sistema de ônibus. Será desenvolvido um produto considerando as condições de contorno da Região Metropolitana de São Paulo, que é o maior conglomerado urbano do Brasil com aproximadamente 20 milhões de habitantes.

\clearpage
\subsection{Específicos}\label{subsec:objEspec}
São objetivos específicos do projeto:
\begin{itemize}
\item gerar dados de avaliações específicas e qualitativas do serviço que permitam: \begin{itemize}
\item melhorar a qualidade do serviço prestado;
\item impactar na remuneração das empresas concessionárias do serviço de ônibus.
\end{itemize}
\item gerar informações que possam auxiliar os usuários a tomar melhores decisões sobre a utilização do sistema de ônibus;
\item coletar dados que permitam uma análise mais acurada de demanda com localização e horário;
\item estimular a educação e a conscientização dos usuários acerca do funcionamento do sistema público de transporte por ônibus.
\end{itemize}
	
\section{Escopo e critério de sucesso}\label{sec:Escopo}
	Ao final deste projeto o software contará com três módulos, a saber:
	\begin{itemize}%[\itshape a\upshape)]
		\item \textbf{módulo ``avaliação''}: para avaliar o sistema de transporte;
		\item \textbf{módulo ``gps social''}: para levantar informações de demanda do sistema de transporte e as fornecer aos usuários, em tempo real\footnote{A expressão ``tempo real'' aqui é utilizada em termos de noção de tempo de um usuário que se desloca a pé, e não no conceito formal de ``tempo real'' da área de conhecimento da computação.}, por meio de um mapa; e
		\item \textbf{módulo ``game''}: que terá as funções de atrair e incentivar a adesão dos usuários e também sirva a um propósito pedagógico de conscientizar os usuários sobre a gestão de um sistema de transporte;
	\end{itemize}

Para seu desenvolvimento será desenvolvido um servidor de aplicação e um aplicativo móvel, e estes se comunicarão por meio de uma \gls{api} \gls{rest}. Além disso, o aplicativo móvel será desenvolvido focado na plataforma \gls{android}.

Espera-se que tanto o servidor de aplicação quanto aplicativo móvel estejam implementados e integrados e que os três módulos estejam funcionais. Para o módulo de game não são esperadas animações, simulações elaboradas ou mesmo interação entre diversos usuários no que tange à mecânica do jogo.

\section{Não-escopo}\label{sec:NãoEscopo}
Foi uma opção do projeto, neste primeiro momento, não inserir as funcionalidades já encontradas em praticamente todos os outros aplicativos e descritas no primeiro parágrafo deste item. Essas funcionalidades podem vir a ser adicionadas futuramente tornando o aplicativo mais completo, assim como são possíveis diversas evoluções, que serão exploradas na seção \ref{sec:futuro}.

\section{Restrições}\label{sec:restrições}
Todo desenvolvimento será realizado utilizando licenças e tecnologias livres e possíveis integrações com softwares terceiros devem levar este fato em consideração, ou seja, será preciso que haja compatibilidade de licenças.