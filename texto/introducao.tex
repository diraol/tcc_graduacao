\chapter{Introdução}
\label{chp:Introdução}

\lettrine{A}{mobilidade urbana} é hoje um dos mais latentes problemas das grandes cidades de todo o mundo. Em 2010, considerando 439 áreas urbanas nos Estados Unidos, os congestionamentos fizeram com que os motoristas gastassem 4,8 bilhões de horas e comprassem 7.2 bilhões de litros de combustível além do necessário, a um custo de \$101 bilhões de dólares~\cite{Eisele2011}. Já em 2011, esses valores subiram para 5,5 bilhões de horas, 11 bilhões de litros de combustível e \$121 bilhões de dólares~\cite{Schrank2012}. 
Assim, percebe-se que não apenas são altos os custos sociais, ambientais e econômicos, como também estes são crescentes.

A realidade brasileira não é muito diferente. De 2004 a 2007 o tempo de congestionamento de 4 das maiores cidades brasileiras (São Paulo, Rio de Janeiro, Belo Horizonte e Porto Alegre) cresceu a uma taxa anual média de 16\%~\cite{resende2009}. Se considerarmos ainda as políticas públicas de incentivo ao mercado automobilístico, mais especificamente a redução do \gls{ipi} para automóveis implementadas em maio de 2012~\cite{brasil2011} e ainda vigentes~\cite{brasil2012}, pode-se esperar que o crescimento do índice de congestionamentos no período tenha sido até maior.

Para além dos congestionamentos, existem ainda outras externalidades comuns na área do transporte, como a poluição e os acidentes de trânsito~\cite{vasconcellos1998}. A poluição, por exemplo, implica graves problemas de saúde pública, como o aumento da incidência de problemas cardiopulmonares, como bronquite e asma~\cite{kunzli2000}, e o aumento dos índices de mortalidade~\cite{finkelstein2004}.
Se essas externalidades possuem suas medidas de desempenhos já consagradas e já se estuda sobre as deseconomias que as causam, a externalidade relativa à insatisfação do usuário do transporte público é terreno ainda pouco explorado na área de desenvolvimento de software.
Sobram evidências de que há questionamentos quanto à qualidade do serviço prestado~\cite{UrbanaPE2010,Rodrigues,Rodrigues2006,Cellos2012}. Segundo~\citeonline{cardoso2013}, o reajuste das tarifas de ônibus foi o estopim das manifestações de junho de 2013, porém as insatisfações não se limitavam a elas, mas permeavam também a qualidade do serviço prestado como um todo. Bastante polêmicas e controversas, mesmo assim ainda podendo ser entendidas sob esse prisma, estão diversas queimas de ônibus como forma de protesto contra a má-qualidade do sistema. \citeonline{fora1999} apontam como uma das recomendações para se combater as deseconomias do transporte público haver financiamentos de estudos e projetos para melhor caracterizar “condições atuais de transporte e trânsito para subsidiar projetos de melhoria”.
Aqui este projeto encontra sua motivação e respaldo. Hoje, graças à emergência de recursos tecnológicos de comunicações que utilizam conceitos como redes, geolocalização, comunicação multidirecional e outros, pode-se definir processos que unam com facilidade os extremos das cadeias de transporte e, assim, permitir um melhor planejamento e correções de rumo mesmo durante a execução do planejamento do sistema de transporte, melhorando a sua eficiência tanto para o poder público, quanto para usuários e prestadores de serviço.

\section{Justificativa do projeto}\label{sec:justificativa}
	Atualmente existem diversos aplicativos que trabalham, majoritariamente, disponibilizando informações fornecidas pela SPTrans, por meio de sua \gls{api} para desenvolvedores\footnote{\url{http://www.sptrans.com.br/desenvolvedores/APIOlhoVivo/Documentacao.aspx}}, aos usuários. 
Isso se traduz em serviços de escolha de linha de ônibus, determinação de rotas e verificação de localização de ônibus e respectivo tempo estimado de chegada, e em alguns casos tempo estimado para se chegar ao destino escolhido.

Entretanto, não foram encontrados aplicativo que permitissem a coleta de avaliação dos usuários sobre elementos-chaves do sistema que não são automaticamente mensuráveis. Assim, o \trilhasp~se propõe a suprir essa demanda captando a opinião do usuário no momento do uso do transporte, permitindo a geração de informações de \textit{feedback} úteis tanto para usuários quanto para a SPTrans, equanto fiscalizadora do sistema, e para as concessionárias do sistema de transporte, permitindo um melhor controle da qualidade da operação.

Foi uma opção do projeto, neste primeiro momento, não inserir as funcionalidades já encontradas em praticamente todos os outros aplicativos e descritas no primeiro parágrafo deste item. Essas funcionalidades podem vir a ser adicionadas futuramente tornando o aplicativo mais completo, assim como são possíveis diversas evoluções, que serão exploradas na seção \ref{sec:futuro}.

\section{Objetivos}\label{sec:objetivos}
\subsection{Geral}\label{subsec:objGerais}
O objetivo geral do projeto é desenvolver um software que corrobore para a melhoria da mobilidade nas cidades pelo incremento da qualidade e da eficiência do sistema público de transporte urbano, mais especificamente rodoviário (ônibus). Será desenvolvido um produto considerando as condições de contorno da Região Metropolitana de São Paulo, que é o maior conglomerado urbano do Brasil com aproximadamente 20 milhões de habitantes. 

\subsection{Específicos}\label{subsec:objEspec}
São objetivos específicos do projeto: gerar dados e informações de valor para os atores envolvidos no sistema, a saber, operadores dos sistemas de transporte público, usuários e SPTrans; possibilitar a troca de informações triangular entre os três atores do sistema; corroborar para a eficiência do sistema pois com mais informação haverá mais controle de operação; promover transparência já que as informações geradas serão em sua maioria dados públicos sujeitos à Lei 12.527/2011; e estimular a conscientização para o usuário acerca do funcionamento do sistema público de transporte por ônibus.
	
\section{Escopo e critério de sucesso}\label{sec:Escopo}
	Ao final deste projeto o software contará com três módulos, a saber:
	\begin{itemize}%[\itshape a\upshape)]
		\item módulo de avaliação do sistema de transporte;
		\item módulo de “gamificação” que terá as funções de oferecer um jogo que atraia e incentive a participação dos usuários e também sirva para um propósito pedagógico de conscientizar os usuários sobre a gestão de um sistema de transporte; e
		\item módulo de ``gps social'', que levantará informações de demanda do sistema de transporte e as fornecerá aos usuários em tempo real por meio de um mapa;
	\end{itemize}

\section{Não-escopo}\label{sec:NãoEscopo}
	Não está no escopo de implementação deste projeto um ou mais módulos que permitam a integração do sistema com prestadores de serviço não ligados ao sistema de transporte. Um exemplo seria a integração com comércios próximos aos pontos de ônibus que poderiam oferecer descontos e ``promoções relâmpago'' para usuários de acordo com suas ``pontuações'' no game do aplicativo, enviando notificações das promoções aos celulares dos usuários próximos ao comércio.

%\section{Stakeholders}\label{sec:stakeholders}
%	Na matriz abaixo encontram-se os \emph{stakeholders} do projeto e seus respectivos poder e influência no projeto:
%
%	\bigskip
%	\begin{table}[H]
%	\centering
%	\caption{Stakeholders, relacionamento, interesse e poder}
%    \begin{tabular}{lccc}
%      \toprule
%      \headerCell{Parte interessada} &
%      \begin{minipage}{0.2\textwidth}
%        \espacoVert
%        \headerCell{Relacionamento com o Projeto}
%        \espacoVert
%      \end{minipage} &
%      \headerCell{Interesse (A/B)} &
%      \headerCell{Poder (A/B)}\\
%        
%      \midrule
%         
%			\textbf{Diego Rabatone Oliveira} &
%			Respons\'avel &
%			A &
%			A\\
%			
%			\textbf{Reginaldo Arakaki} &
%			Orientador &
%			A &
%			A\\
%
%			\textbf{Haydée Svab} &
%			Co-orientadora &
%			A &
%			A\\
%
%			\textbf{Prefeitura de São Paulo} &
%			Interessada &
%			B &
%			B\\
%			
%			\textbf{SPTrans} &
%			Interessada &
%			B &
%			B\\
%
%			\bottomrule
%	\end{tabular}
%	\end{table}

%\bigskip

\section{Restrições}\label{sec:restrições}
Todo desenvolvimento será realizado utilizando licenças livres e possiveis integrações com softwares terceiros devem levar este fato em consideração, ou seja, será preciso que haja compatibilidade de licenças.