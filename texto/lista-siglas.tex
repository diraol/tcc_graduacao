%%\printacronyms
%%Print the glossary
%\section*{Lista de Acrônimos}
\newacronym{its}{ITS}{Intelligent Transport Systems}
\newacronym{mtv}{MTV}{Model View Template}
\newacronym{ipi}{IPI}{Imposto sobre produtos industrializados}
\newacronym{rest}{REST}{Representational State Transfer}
\newacronym{mvt_acr}{MVT}{Model View Template}
\newacronym{sgbd}{SGBD}{Sistema de Gerenciamento de Base de Dados}
\newacronym{tdd}{TDD}{\textit{Test Driven Development} (Desenvolvimento guiado por testes)}
%\section*{Glossário}
\newglossaryentry{mvt}
    {
        name=\gls{mvt_acr},
        description={Model View Template, \textit{Design Pattern} utilizada no framework \gls{django}.}
    }
\newglossaryentry{django}
    {
        name=Django,
        description={Django é um framework para desenvolvimento rápido para web, escrito em Python, que utiliza o padrão \gls{mvt_acr}.}
    }
\hyphenation{Django}
\newglossaryentry{mobfirst}
    {
        name={Mobile First},
        description={Conceito que propõe uma nova forma de se abordar o desenvolvimento de aplicações, em especial web, no qual desenvolve-se a interface primeiramente para dispositivos móveis e, posteriormente, adapta-a para outros dispositivos (tablets, notebooks, desktops, etc).}
    }



\renewcommand*{\glsseeformat}[3][\seename]{\textit{#1}
\glsseelist{#2}}
% ---


% ----
% Início do documento
% ----