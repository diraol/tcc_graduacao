%\newacronym%
%[description={}]%
%{}{}{}%

\hyphenation{Django}%
\newglossaryentry{django}{%
    name=Django,%
    description={\'{e} um framework para desenvolvimento r\'{a}pido para web, escrito em Python, que utiliza o padr\~{a}o MVT}%
}%
\newglossaryentry{nginx}{%
    name=NGINX,%
    description={\'{E} um dos servidore HTTP e de proxy reverso que mais tem ganho espa\c{c}o no mercado, em grande parte por seu elevado desempenho. Ele serve ainda como servidor de cache e permite trabalhar com \textit{load balance} entre diversos servidores}%
}%
\newglossaryentry{uwsgi}{%
    name=uWSGI,%
    description={\'{e} um servidor de aplica\c{c}\~{a}o para diversas linguagens de programa\c{c}\~{a}o, sendo Python uma delas, sendo que ele apresenta um dos melhores desempenhos}%
}%
\newglossaryentry{json}{%
    name=JSON,%
    description={acrônimo para ``\textit{JavaScript Object Notation}'', é um formato leve para intercâmbio de dados computacionais que foi proposto por Douglas Crockford e é descrito na \href{https://tools.ietf.org/html/rfc4627}{RFC 4627}}%
}%
\newglossaryentry{ionic}{%
    name=ionic,%
    description={\'{e} um Kit de Desenvolvimento de Software \textit{open source} de \textit{front-end} para desenvolvimento de aplicativos móveis híbridos, com HTML5. Mais informações em: \url{http://ionicframework.com/}}%
}%
\newglossaryentry{cordova}{%
    name=Cordova,%
    description={\'{e} uma plataforma para construção de aplicativos móveis nativos utilizando HTML, CSS e JavaScript. Em sua essência, ela encapsula uma aplicação ``web'' para cumprir tal tarefa. É uma plataforma desenvolvida pela Fundação APACHE. Mais informações em: \url{https://cordova.apache.org/}}%
}%
\newglossaryentry{angular}{%
    name=AngularJS,%
    description={\'{e} um \textit{framework} desenvolvido totalmente em JavaScript e desenvolvido pelo Google para desenvolvimento web}%
}%
\newglossaryentry{android}{%
    name=Android,%
    description={sistema operacional livre para dispositivos móveis desenvolvido sob o núcleo Linux}%
}%
\newglossaryentry{dp}{%
    name={\textit{Design Pattern}},%
    description={ou \textbf{Padrão de Projeto}. Termo da Engenharia de Software que representa uma solução geral reutilizável para um problema que ocorre com frequência dentro de um determinado contexto no projeto de software},%
    plural={\textit{Design Patterns}},
    sort={Design Pattern}
}%

\newacronym{rest}{REST}{\textit{Representational State Transfer}}

\newacronym{sgbd}{SGBD}{Sistema de Gerenciamento de Base de Dados}

\newacronym{its}{ITS}{Intelligent Transport Systems}

\newacronym{w3c}{W3C}{World Wide Web Consortium}

\newacronym{http}{HTTP}{Hyper Text Transfer Protocol}

\newacronym{https}{HTTPS}{Hyper Text Transfer Protocol Secure}

\newacronym{html}{HTML}{HyperText Markup Language}

\newdualentry{qrcode}{QRCode}{\textit{Quick Response Code}}{é um código de barras bidimensional, que ao ser lido pode ser traduzido em diversos tipos de conteúdos, como um texto, um número de telefone, uma URI, uma localização georreferenciada, etc. Existem diversos padrões que foram definidos ao longo do tempo, como, por exemplo: ISO/IEC 18004:2000\footnote{Norma ISO/IEC 18004:2000 disponível em: \url{http://www.iso.org/iso/iso_catalogue/catalogue_ics/catalogue_detail_ics.htm?csnumber=30789} - acesso em 30/11/2014} e ISO/IEC 18004:2006\footnote{Norma ISO/IEC 18004:2006 disponível em: \url{http://www.iso.org/iso/iso_catalogue/catalogue_tc/catalogue_detail.htm?csnumber=43655} - Acesso em 30/11/2014}}

\newdualentry{mvt}{MVT}{\textit{Model View Template}}{\textit{Design Pattern} utilizada no framework Django similar à tradicional MVC (Model View Controller)}

\newdualentry{ipi}{IPI}{Imposto sobre Produtos Industrializados}{imposto que incide sobre os produtos industrializados nacionais e estrangeiros no momento do desembaraço aduaneiro de produto de procedência estrangeira, ou a saída do produto do estabelecimento industrial ou equiparado a industrial}

\newdualentry{api}{API}{\textit{Application Programming Interface}}{ou Interface de Programação de Aplicações ou Interface de Programação de Aplicativos, é um conjunto de rotinas e padrões estabelecidos por um software para a utilização das suas funcionalidades por aplicativos que não pretendem envolver-se em detalhes da implementação do software, mas apenas usar seus serviços}

\newdualentry{mvc}{MVC}{\textit{Model View Controller}}{\textit{Design Pattern} muito conhecida e introduzida por \textit{Erich Gamma} em 1995}

\newdualentry{drf}{DRF}{\textit{Django Rest Framework}}{Módulo do Django para criar e expor uma API REST}

\newdualentry{psa}{PSA}{\textit{Python Social Auth}}{Módulo do Django para permitir autenticação com redes sociais}

\newdualentry{cors}{CORS}{\textit{Cross-origin resource sharing}}{ou \textbf{compartilhamento de recursos entre-origens} é uma especificação de uma tecnologia de navegadores que define meios para um servidor permitir que seus recursos sejam acessados por uma página web de um domínio diferente do domínio do próprio servidor.}

\newdualentry{vps}{VPS}{\textit{Virtual Private Server}}{ou Servidor Virtual Privado, é um servidor virtualizado no qual apenas o contratante tem acesso e tem plenos poderes de instalar, desinstalar e configurá-lo como quiser.}

\glsresetall