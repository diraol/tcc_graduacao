\section{Avaliações}\label{sec:avalia}
\subsection{Avaliações Específicas}\label{subsec:avaliaspec}
Aos usuários serão facultadas as seguintes avaliações:
\begin{description}
    \item[Lotação] \hfill \\
    Como o usuário avalia a lotação do ônibus, de muito lotado a vazio
    \item[Conforto Térmico] \hfill \\
    Como o usuário avalia o conforto térmico do ônibus, de muito quente a muito frio
    \item[Higiene] \hfill \\
    Como o usuário avalia a limpeza do ônibus, de muito sujo a muito limpo
    \item[Atendimento do Motorista] \hfill \\
    Como o usuário avalia o atendimento do motorista, de péssimo a ótimo
    \item[Atendimento do Cobrador] \hfill \\
    Como o usuário avalia o atendimento do cobrador, de péssimo a ótimo    
\end{description}

\subsection{Modelo de avaliação}\label{subsec:modavalia}
As avaliações serão realizadas por meio de um \textit{slidebar} que pode assumir qualquer posição entre o extremo superior e o extremo inferior. Assim consegue-se obter uma variável quantitativa ao invés de uma qualitativa, o que permite realizar uma série de análises como Análise de Conglomerados e Análises de Correlação.

Falta definir ainda se ao usuário será mostrada uma escala numérica ou uma escala textual, ou seja, se será ``de 0 a 10'' ou se será ``de péssimo a ótimo''. Para definir este ponto ainda é preciso estudar melhor como é a percepção e a interação dos usuários a cada tipo de escala.