\section{Avaliações}\label{sec:avalia}
\subsection{Avaliações Específicas}\label{subsec:avaliaspec}
Aos usuários serão facultadas as seguintes avaliações:
\begin{description}
    \item[Lotação] \hfill \\
    Como o usuário avalia a lotação do ônibus, de muito lotado a vazio
    \item[Conforto Térmico] \hfill \\
    Como o usuário avalia o conforto térmico do ônibus, de muito quente a muito frio
    \item[Higiene] \hfill \\
    Como o usuário avalia a limpeza do ônibus, de muito sujo a muito limpo
    \item[Atendimento do Motorista] \hfill \\
    Como o usuário avalia o atendimento do motorista, de péssimo a ótimo
    \item[Atendimento do Cobrador] \hfill \\
    Como o usuário avalia o atendimento do cobrador, de péssimo a ótimo    
\end{description}

\subsection{Modelo de avaliação}\label{subsec:modavalia}
As avaliações serão realizadas por meio de um \textit{slidebar} que pode assumir qualquer posição entre os extremos. No Slider estarão indicados, por meio de textos, os limites ``esquerdo'' e ``direito'', pior e melhor avaliações respectivamente, e também o valor central, que tende a assumir um valor mais ``neutro'' ou ``indiferente''. Por trás das \textit{labels} que serão apresentadas estarão valores, que serão registrados com a nota do usuário. Por convenção do projeto optou-se inicialmente por assumir notas de -5 (menos cinco) a +5 (mais cinco), de forma que médias positivas indicam avaliações positivas, médias negativas indicam avaliações negativas e a ``avaliação neutra'' será obtida com a média 0. Além disso, dessa forma teremos uma variáveis quantitativas ao invés de uma qualitativas, o que permitirá a realização de uma série de análises como Análise de Conglomerados e Análises de Correlação, que não seriam possíveis com as variáveis qualitativas. Também permite a elaboração de médias, medianas, desvio padrão, dentre outros, enquanto no caso das variáveis quantitativas somente poderíamos trabalhar com as frequências das respostas dadas.