\section{Requisitos Funcionais}\label{sec: RF}
Nesta se\c{c}\~ao s\~ao apresentados os Requisitos Funcionais do \trilhasp~que devem ser satisfeitos. Estes requisitos devem ainda ser indicados no pr\'oprio código fonte do projeto, assim como nos testes que verifiquem o requisito.

\StartReqFunc
\vfill
\begin{Requisito}
    \ReqNome{Cadastrar usu\noexpand\'ario}%Nome
    \ReqTipo{funcional}%Tipo
    %\ReqLabel{cadUsu}%Label
    \ReqDescr{O usu\'ario realiza seu cadastro na plataforma.}%Desc.
    \ReqPrioridade{alta}%Prioridade
    \ReqStatus{aprovado}%Status
    \ReqEstabilidade{alta}%Estabilidade
    \ReqOrigem{interna}%Origem
    \ReqRationale{Permite realizar um ``tracking'' do hist\'orico dos usu\'arios para oferecer informa\c{c}\~oes mais precisas, al\'em de permitir que o usu\'ario utilize o ``game'', que demanda um ac\'umulo de participa\c{c}\~ao na plataforma.}%Rationale
    \ReqAssoc{}%Assoc
\end{Requisito}

\vfill
\begin{Requisito}
    \ReqNome{Realizar login}%Nome
    \ReqTipo{funcional}%Tipo
    %\ReqLabel{login}%Label
    \ReqDescr{O usu\'ario deve fazer login para ter acesso ao sistema, impedindo que pessoas n\~ao autorizadas tenham acesso a certas fun\c{c}\~oes do sistema.}%Desc.
    \ReqPrioridade{alta}%Prioridade
    \ReqStatus{aprovado}%Status
    \ReqEstabilidade{alta}%Estabilidade
    \ReqOrigem{usuario}%Origem
    \ReqRationale{M\'etodo que previne que pessoas n\~ao autorizadas tenham acesso \'as funcionalidades do sistema, em especial as que alteram o banco de dados.}%Rationale
    \ReqAssoc{Realizar login social}%Assoc
\end{Requisito}

\vfill
\begin{Requisito}
    \ReqNome{Realizar login social}%Nome
    \ReqTipo{funcional}%Tipo
    %\ReqLabel{loginsocial}%Label
    \ReqDescr{Usu\'ario utiliza seu login de redes sociais para se conectar \`a plataforma}%Desc.
    \ReqPrioridade{alta}%Prioridade
    \ReqStatus{aprovado}%Status
    \ReqEstabilidade{alta}%Estabilidade
    \ReqOrigem{usuario}%Origem
    \ReqRationale{Para aumentar a facilidade de acesso \`a plataforma \'e fundamental permitir que os usu\'arios acessem o sistema utilizando sistemas de autentica\c{c}\~ao cruzada com redes sociais, principalmente \textit{Facebook}, \textit{Twitter} e \textit{Google}.}%Rationale
    \ReqAssoc{Realizar Login}%Assoc
\end{Requisito}

\vfill
\begin{Requisito}
    \ReqNome{Realizar logout}%Nome
    \ReqTipo{funcional}%Tipo
    %\ReqLabel{loginsocial}%Label
    \ReqDescr{O usu\'ario deve fazer logout ao terminar de utilizar o sistema, impedindo que pessoas
n\~ao autorizadas tenham acesso a certas fun\c{c}\~oes do sistema.}%Desc.
    \ReqPrioridade{alta}%Prioridade
    \ReqStatus{aprovado}%Status
    \ReqEstabilidade{alta}%Estabilidade
    \ReqOrigem{usuario}%Origem
    \ReqRationale{M\'etodo que previne que pessoas n\~ao autorizadas tenham acesso \'as funcionalidades
do sistema, em especial as que alteram o banco de dados.}%Rationale
    \ReqAssoc{Realizar Login}%Assoc
\end{Requisito}

\vfill
\begin{Requisito}
    \ReqNome{Listar linhas de \noexpand\^onibus mais prov\noexpand\'aveis}%Nome
    \ReqTipo{funcional}%Tipo
    %\ReqLabel{loginsocial}%Label
    \ReqDescr{O sistema deve listar as linhas de \^onibus mais prov\'aveis \`a escolha do usu\'ario, de acordo com sua localiza\c{c}\~ao.}%Desc.
    \ReqPrioridade{alta}%Prioridade
    \ReqStatus{aprovado}%Status
    \ReqEstabilidade{alta}%Estabilidade
    \ReqOrigem{usuario}%Origem
    \ReqRationale{M\'etodo que reduz o tempo necess\'ario para que o usu\'ario escolha o \^onibus, tanto para que possa realizar uma avalia\c{c}\~ao quanto para que possa obter informa\c{c}\~oes sobre a linha.}%Rationale
    \ReqAssoc{Coletar posicionamento do usu\'ario}%Assoc
\end{Requisito}

\vfill
\begin{Requisito}
    \ReqNome{Selecionar linha de \noexpand\^onibus}%Nome
    \ReqTipo{funcional}%Tipo
    %\ReqLabel{cadUsu}%Label
    \ReqDescr{O usu\'ario seleciona a linha de \^onibus que deseja avaliar.}%Desc.
    \ReqPrioridade{alta}%Prioridade
    \ReqStatus{aprovado}%Status
    \ReqEstabilidade{alta}%Estabilidade
    \ReqOrigem{interna}%Origem
    \ReqRationale{O usu\'ario precisa conseguir selecionar a linha de \^onibus para poder avali\'a-la.}%Rationale
    \ReqAssoc{Listas linhas de \^onibus mais prov\'aveis.}%Assoc
\end{Requisito}

\vfill
\begin{Requisito}
    \ReqNome{Coletar posicionamento do usu\noexpand\'ario}%Nome
    \ReqTipo{funcional}%Tipo
    %\ReqLabel{cadUsu}%Label
    \ReqDescr{O ``sistema'' deve coletar a informa\c{c}\~ao de geolocaliza\c{c}\~ao do usu\'ario, usando GPS e outros m\'etodos como Triangula\c{c}\~ao via 3G [usar API de geolocaliza\c{c}\~ao do HTML5].}%Desc.
    \ReqPrioridade{alta}%Prioridade
    \ReqStatus{aprovado}%Status
    \ReqEstabilidade{alta}%Estabilidade
    \ReqOrigem{interna}%Origem
    \ReqRationale{Para que o sistema possa exibir a lista de poss\'{\i}veis linhas de \^onibus que o usu\'ario deve estar, incluindo o sentido da linha, \'e fundamental conhecer a localiza\c{c}\~ao do usu\'ario e seu sentido de deslocamento.}%Rationale
    \ReqAssoc{Listas linhas de \^onibus mais prov\'aveis.}%Assoc
\end{Requisito}

\vfill
\begin{Requisito}
    \ReqNome{Realizar avalia\noexpand\c\noexpand{c}\noexpand\~ao global}%Nome
    \ReqTipo{funcional}%Tipo
    %\ReqLabel{cadUsu}%Label
    \ReqDescr{O usu\'ario avalia globalmente o servi\c{c}o de transporte p\'ublico com uma nota numa escala cont\'{\i}nua, utilizando um \textit{slider}.}%Desc.
    \ReqPrioridade{alta}%Prioridade
    \ReqStatus{aprovado}%Status
    \ReqEstabilidade{alta}%Estabilidade
    \ReqOrigem{interna}%Origem
    \ReqRationale{É importante uma avalia\c{c}\~ao global, simples e direta, para o caso do usu\'ario n\~ao querer preencher todas as avalia\c{c}\~oes espec\'{\i}ficas. Al\'em disso, o uso da escala cont\'{\i}nua \'e importante pois far\'a com que a avalia\c{c}\~ao seja uma vari\'avel quantitativa, o que permitir\'a exprimir estat\'{\i}sticas simples como m\'edia e desvio padr\~ao.}%Rationale
    \ReqAssoc{}%Assoc
\end{Requisito}

\vfill
\begin{Requisito}
    \ReqNome{Realizar avalia\noexpand\c\noexpand{c}\noexpand\~oes espec\noexpand\'\noexpand{\noexpand\i}ficas}%Nome
    \ReqTipo{funcional}%Tipo
    %\ReqLabel{cadUsu}%Label
    \ReqDescr{O usu\'ario avalia cada um dos crit\'erios espec\'{\i}ficos definidos como uma nota numa escala cont\'{\i}nua para cada um, utilizando um \textit{slider}.}%Desc.
    \ReqPrioridade{alta}%Prioridade
    \ReqStatus{aprovado}%Status
    \ReqEstabilidade{alta}%Estabilidade
    \ReqOrigem{interna}%Origem
    \ReqRationale{Os crit\'erios escolhidos ser\~ao de car\'ater subjetivo, crit\'erios estes que n\~ao seriam mensur\'aveis utilizando-se tecnologias de automa\c{c}\~ao. Al\'em disso, o uso da escala cont\'{\i}nua \'e importante pois far\'a com que a avalia\c{c}\~ao seja uma vari\'avel quantitativa, o que permitir\'a exprimir estat\'{\i}sticas simples como m\'edia e desvio padr\~ao.}%Rationale
    \ReqAssoc{}%Assoc
\end{Requisito}

\vfill
\begin{Requisito}
    \ReqNome{Adicionar reclama\noexpand\c{c}\noexpand\~ao}%Nome
    \ReqTipo{funcional}%Tipo
    %\ReqLabel{cadUsu}%Label
    \ReqDescr{Para cada "nota" negativa que o usu\'ario d\'a ser\'a mostrado a ele uma caixa de texto para preenchimento de uma reclama\c{c}\~ao [opcional].}%Desc.
    \ReqPrioridade{alta}%Prioridade
    \ReqStatus{aprovado}%Status
    \ReqEstabilidade{alta}%Estabilidade
    \ReqOrigem{interna}%Origem
    \ReqRationale{Ser\'a permitido ao usu\'ario adicionar uma descri\c{c}\~ao do motivo pelo qual ele deu uma nota negativa para aquele determinado crit\'erio, permitindo assim identificar motivos de avalia\c{c}\~oes negativas para cada crit\'erio, o que ajuda na melhoria do servi\c{c}o prestado.}%Rationale
    \ReqAssoc{}%Assoc
\end{Requisito}

\vfill
\begin{Requisito}
    \ReqNome{Enviar reclama\noexpand\c{c}\noexpand\~ao}%Nome
    \ReqTipo{funcional}%Tipo
    %\ReqLabel{cadUsu}%Label
    \ReqDescr{Cada reclama\c{c}\~ao escrita dever\'a estar associada a um endere\c{c}o \'unico a ser enviado, via twitter, \`a \sptrans.}%Desc.
    \ReqPrioridade{alta}%Prioridade
    \ReqStatus{aprovado}%Status
    \ReqEstabilidade{alta}%Estabilidade
    \ReqOrigem{interna}%Origem
    \ReqRationale{É fundamental que o processo de avalia\c{c}\~ao e \textit{accountability} seja o mais p\'ublicos e transparentes poss\'{\i}veis. Dessa forma, conforme apresentado anteriormente \`a \sptrans, as reclama\c{c}\~oes ser\~ao postadas publicamente, com a utiliza\c{c}\~ao de algumas \textit{hashtags} espec\'{\i}ficas pr\'e-definidas para serem respondidas pelo \'org\~ao p\'ublicamente.}%Rationale
    \ReqAssoc{}%Assoc
\end{Requisito}

\clearpage
\subsection{Tabela Resumo dos Requisitos}\label{subsec:tabResReqF}
   \begin{table}[H]
        \centering
        \caption{Requisitos funcionais}
        \label{tab:reqFunc}
        \PrintRequisitosFunc
    \end{table}
