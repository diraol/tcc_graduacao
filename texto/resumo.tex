\lettrine{A}{ presente monografia} descreve a implementação do aplicativo \trilhasp, um aplicativo que se propõe a melhorar o fluxo de informações entre usuário e prestador de serviço público de transporte, tanto fornecendo aos gestores avaliações do sistema realizadas pelos usuários quanto disponibilizando aos usuários informações para uma tomada de decisão mais consciente ao utilizar o sistema de transporte.
Com o \trilhasp~os usuários do sistema público de transporte poderão avaliar o serviço, segundo critérios qualitativos como ``o ônibus estava muito lotado'', ``fui bem atendido pelo motorista'' e ``o ônibus estava sujo''. A opção por estes critérios qualitativos se deu pois identificou-se que eles influenciam na decisão do usuário sobre o modo de transporte preferido e não são facilmente mensurados por meio de tecnologias de automação, como é o caso da velocidade e frequência dos ônibus. Essas avaliações, que podem ser positivas ou negativas numa escala contínua, permitirão a criação de indicadores por ônibus que poderão ser utilizados pelas autoridades para melhorar o serviço e também influenciar no sistema de remuneração das empresas prestadoras de serviço.
O módulo ``Mapa'' mostrará aos usuários um mapa com todos os usuários conectados, o que permitirá ao usuário optar por ir ou não para o ponto de ônibus num determinado horário usando a informação de demanda e ``lotação'' do ponto naquele momento, o que pode melhorar a distribuição da demanda no sistema, e levar a uma melhora do serviço prestado.
Por fim, o módulo ``game'' tem por objetivo tanto atrair e reter os usuários no aplicativo quanto ser uma solução educativa, levando ao usuário informações e vivência verossímeis à realidade do sistema de transporte, como custo de um ônibus, necessidade de manutenção, etc.
O projeto foi desenvolvido utilizando tecnologias livres e também terá seu código fonte distribuído livremente.

\vspace{\onelineskip}

\vfill

\noindent

\textbf{Palavras-chaves}: Software livre, Transporte Urbano, Sistema de Posicionamento Global, Ônibus