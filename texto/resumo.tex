%Resumo - Bruno Fernandes
%$Rev$ - $Author$ 
%$Id$ - $Date$
% rev3 - 11/12/13

\lettrine{O}{presente projeto} desenvolve uma unidade de gerenciamento eletrônico (conhecida também como ECU - \textit{Eletronic Control Unit}) para um motor a combustão interna. Gerenciando corretamente ignição, injeção, relés e válvula borboleta, a unidade deverá ser capaz de controlar adequadamente a rotação do motor, mesmo em condições de carga. A execução do projeto envolve o desenvolvimento de um \textit{hardware} (placa de circuito impresso), com todas as entradas e saídas necessárias para interligar sensores e atuadores ao sistema, além de fornecer uma interface de comunicação para monitoramento de parâmetros em tempo real. Além do \textit{hardware}, o projeto envolve desenvolvimento de \textit{firmware}, correspondente à programação dos microcontroladores presentes no sistema, e desenvolvimento de \textit{software}, correspondente a uma aplicação que, rodando em computador externo com sistema operacional \textit{Windows}, deverá se comunicar com o \textit{hardware} a fim de possibilitar a visualização de parâmetros e variáveis de erro. A mesma aplicação deverá também fornecer uma opção para se controlar o motor a partir do computador, simulando, neste caso, a leitura do pedal de aceleração do veículo. O projeto tem como meta final a sua implantação em um motor Volkswagen 2.0L aplicado a um veículo modelo Polo Sedan 2004, de tal forma que se possa avaliar o desempenho do motor em condições de carga, através do uso de um dinamômetro inercial. A elaboração e execução serão feitas em cooperação com o departamento de eletrônica automotiva da Faculdade de Tecnologia de Sandro André, o qual cederá o veículo e o dinamômetro para testes. É importante ressaltar também que o projeto teve início com o desenvolvimento em conjunto da parte conceitual e teórica ao projeto Otto \cite{andre2012} em 2012, desenvolvido por André Masakazu Ferreira Soares e Vitor Saiki Scarpinetti, justificando o fato das partes introdutória e teórica serem semelhantes em ambos os projetos.

	\textbf{Palavras-chaves}: Unidade de gerenciamento eletrônico. Injeção eletrônica. Ignição eletrônica. Válvula borboleta. Eletrônica automotiva. Motor a combustão interna. Controle de rotação.

%End of chapter
