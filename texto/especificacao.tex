\chapter{Especificação}\label{chp: Especificação}
Definiu-se que, por limitações de tempo de desenvolvimento, o projeto será desenvolvido como um projeto Web, utilizando o conceito \gls{mobfirst}, assim evita-se a necessidade de desenvolvimento para plataformas específicas como \textbf{Android} ou \textbf{iOS}.

\section{Tecnologias envolvidas}\label{sec: TecEnvolv}
Por motivos de conhecimento prévio, definido que o projeto será desenvolvido sob o \textit{framework} \gls{django}, que utiliza a linguagem de programação \textbf{Python} e modelo de desenvolvimento \gls{mvt}.

Para o \textit{frontend} do projeto será utilizado \textbf{HTML5}, \textbf{CSS3} e \textbf{JavaScript}, além de algumas bibliotecas auxiliares como \textbf{Twitter Bootstrap} (versão 3).

Com relação à base de dados, até o presente momento o \gls{sgbd} será o \textbf{PostgreSQL} com a extensão \textbf{PostGis} para informações georreferenciadas.

\section{Requisitos Funcionais}\label{sec: RF}
%Requisitos Funcionais, n\~ao funcionais, regras de neg\'ocio e restri\c{c}\~oes encontram-se no anexo \ref{Anexo Requisitos}
%%\temporario{Descreve os requisitos no n\'{\i}vel que permita desenvolver o software que satisfa\c{c}a os requisitos do produto e os testes que mostrem o atendimento destes requisitos. O conte\'udo das se\c{c}\~oes podem ser desenvolvidos como documentos independentes.\\
%%Neste documento, esta se\c{c}\~ao est\'a organizada para atender \'as necessidades da An\'alise Orientada a Objetos.}

%%02
%\vfill
%\begin{Requisito}
%    \ReqTipo{funcional}%Tipo
%    \ReqNome{Remover Material}%Nome
%%%    \ReqLabel{ReqRemMat}%Label
%    \ReqDescr{O sistema recebe a mensagem do usu\'ario solicitando a remo\c{c}\~ao de determinado(s) item(ns) do banco de dados do sistema.}%Desc.
%    \ReqPrioridade{media}%Prioridade
%    \ReqStatus{proposto}%Status
%    \ReqEstabilidade{media}%Estabilidade
%    \ReqOrigem{cliente}%Origem
%    \ReqRationale{Esta funcionalidade permite a remo\c{c}\~ao de determinado(s) item(ns) do banco de dados pelo usu\'ario. É utilizada em caso de descontinuidade de um produto, quando a loja para de comercializar algum produto, entre outros.}%Rationale
%    \ReqAssoc{Banco de dados estruturado}%Assoc
%\end{Requisito}

%%01
\vfill
\begin{Requisito}
    \ReqTipo{funcional}%Tipo
    \ReqNome{Cadastrar usu\'ario}%Nome
    %\ReqLabel{cadUsu}%Label
    \ReqDescr{O usu\'ario realiza seu cadastro na plataforma.}%Desc.
    \ReqPrioridade{alta}%Prioridade
    \ReqStatus{aprovado}%Status
    \ReqEstabilidade{alta}%Estabilidade
    \ReqOrigem{interna}%Origem
    \ReqRationale{Permite realizar um ``tracking'' do hist\'orico dos usu\'arios para oferecer informa\c{c}\~oes mais precisas, al\'em de permitir que o usu\'ario utilize o ``game'', que demanda um ac\'umulo de participa\c{c}\~ao na plataforma.}%Rationale
    \ReqAssoc{}%Assoc
\end{Requisito}

%%02
\vfill
\begin{Requisito}
    \ReqTipo{funcional}%Tipo
    \ReqNome{Realizar login}%Nome
    %\ReqLabel{login}%Label
    \ReqDescr{O usu\'ario deve fazer login para ter acesso ao sistema, impedindo que pessoas n\~ao autorizadas tenham acesso a certas fun\c{c}\~oes do sistema.}%Desc.
    \ReqPrioridade{alta}%Prioridade
    \ReqStatus{aprovado}%Status
    \ReqEstabilidade{alta}%Estabilidade
    \ReqOrigem{usuario}%Origem
    \ReqRationale{M\'etodo que previne que pessoas n\~ao autorizadas tenham acesso \'as funcionalidades do sistema, em especial as que alteram o banco de dados.}%Rationale
    \ReqAssoc{Realizar login social}%Assoc
\end{Requisito}

%%03
\vfill
\begin{Requisito}
    \ReqTipo{funcional}%Tipo
    \ReqNome{Realizar login social}%Nome
    %\ReqLabel{loginsocial}%Label
    \ReqDescr{Usu\'ario utiliza seu login de redes sociais para se conectar \`a plataforma}%Desc.
    \ReqPrioridade{alta}%Prioridade
    \ReqStatus{aprovado}%Status
    \ReqEstabilidade{alta}%Estabilidade
    \ReqOrigem{usuario}%Origem
    \ReqRationale{Para aumentar a facilidade de acesso \`a plataforma \'e fundamental permitir que os usu\'arios acessem o sistema utilizando sistemas de autentica\c{c}\~ao cruzada com redes sociais, principalmente \textit{Facebook}, \textit{Twitter} e \textit{Google}.}%Rationale
    \ReqAssoc{Realizar Login}%Assoc
\end{Requisito}

%%04
\vfill
\begin{Requisito}
    \ReqTipo{funcional}%Tipo
    \ReqNome{Realizar logout}%Nome
    %\ReqLabel{loginsocial}%Label
    \ReqDescr{O usu\'ario deve fazer logout ao terminar de utilizar o sistema, impedindo que pessoas
n\~ao autorizadas tenham acesso a certas fun\c{c}\~oes do sistema.}%Desc.
    \ReqPrioridade{alta}%Prioridade
    \ReqStatus{aprovado}%Status
    \ReqEstabilidade{alta}%Estabilidade
    \ReqOrigem{usuario}%Origem
    \ReqRationale{M\'etodo que previne que pessoas n\~ao autorizadas tenham acesso \'as funcionalidades
do sistema, em especial as que alteram o banco de dados.}%Rationale
    \ReqAssoc{Realizar Login}%Assoc
\end{Requisito}

%%05
\vfill
\begin{Requisito}
    \ReqTipo{funcional}%Tipo
    \ReqNome{Listar linhas de \^onibus mais prov\'aveis}%Nome
    %\ReqLabel{loginsocial}%Label
    \ReqDescr{O sistema deve listar as linhas de \^onibus mais prov\'aveis \`a escolha do usu\'ario, de acordo com sua localiza\c{c}\~ao.}%Desc.
    \ReqPrioridade{alta}%Prioridade
    \ReqStatus{aprovado}%Status
    \ReqEstabilidade{alta}%Estabilidade
    \ReqOrigem{usuario}%Origem
    \ReqRationale{M\'etodo que reduz o tempo necess\'ario para que o usu\'ario escolha o \^onibus, tanto para que possa realizar uma avalia\c{c}\~ao quanto para que possa obter informa\c{c}\~oes sobre a linha.}%Rationale
    \ReqAssoc{Coletar posicionamento do usu\'ario}%Assoc
\end{Requisito}

%%06
\vfill
\begin{Requisito}
    \ReqTipo{funcional}%Tipo
    \ReqNome{Selecionar linha de \^onibus}%Nome
    %\ReqLabel{cadUsu}%Label
    \ReqDescr{O usu\'ario seleciona a linha de \^onibus que deseja avaliar.}%Desc.
    \ReqPrioridade{alta}%Prioridade
    \ReqStatus{aprovado}%Status
    \ReqEstabilidade{alta}%Estabilidade
    \ReqOrigem{interna}%Origem
    \ReqRationale{O usu\'ario precisa conseguir selecionar a linha de \^onibus para poder avali\'a-la.}%Rationale
    \ReqAssoc{Listas linhas de \^onibus mais prov\'aveis.}%Assoc
\end{Requisito}

%%07
\vfill
\begin{Requisito}
    \ReqTipo{funcional}%Tipo
    \ReqNome{Coletar posicionamento do usu\'ario}%Nome
    %\ReqLabel{cadUsu}%Label
    \ReqDescr{O ``sistema'' deve coletar a informa\c{c}\~ao de geolocaliza\c{c}\~ao do usu\'ario, usando GPS e outros m\'etodos como Triangula\c{c}\~ao via 3G [usar API de geolocaliza\c{c}\~ao do HTML5].}%Desc.
    \ReqPrioridade{alta}%Prioridade
    \ReqStatus{aprovado}%Status
    \ReqEstabilidade{alta}%Estabilidade
    \ReqOrigem{interna}%Origem
    \ReqRationale{Para que o sistema possa exibir a lista de poss\'{\i}veis linhas de \^onibus que o usu\'ario deve estar, incluindo o sentido da linha, \'e fundamental conhecer a localiza\c{c}\~ao do usu\'ario e seu sentido de deslocamento.}%Rationale
    \ReqAssoc{Listas linhas de \^onibus mais prov\'aveis.}%Assoc
\end{Requisito}

%%08
\vfill
\begin{Requisito}
    \ReqTipo{funcional}%Tipo
    \ReqNome{Realizar avalia\c{c}\~ao global}%Nome
    %\ReqLabel{cadUsu}%Label
    \ReqDescr{O usu\'ario avalia globalmente o servi\c{c}o de transporte p\'ublico com uma nota numa escala cont\'{\i}nua, utilizando um \textit{slider}.}%Desc.
    \ReqPrioridade{alta}%Prioridade
    \ReqStatus{aprovado}%Status
    \ReqEstabilidade{alta}%Estabilidade
    \ReqOrigem{interna}%Origem
    \ReqRationale{É importante uma avalia\c{c}\~ao global, simples e direta, para o caso do usu\'ario n\~ao querer preencher todas as avalia\c{c}\~oes espec\'{\i}ficas. Al\'em disso, o uso da escala cont\'{\i}nua \'e importante pois far\'a com que a avalia\c{c}\~ao seja uma vari\'avel quantitativa, o que permitir\'a exprimir estat\'{\i}sticas simples como m\'edia e desvio padr\~ao.}%Rationale
    \ReqAssoc{}%Assoc
\end{Requisito}

%%09
\vfill
\begin{Requisito}
    \ReqTipo{funcional}%Tipo
    \ReqNome{Realizar avalia\c{c}\~oes espec\'{\i}ficas}%Nome
    %\ReqLabel{cadUsu}%Label
    \ReqDescr{O usu\'ario avalia cada um dos crit\'erios espec\'{\i}ficos definidos como uma nota numa escala cont\'{\i}nua para cada um, utilizando um \textit{slider}.}%Desc.
    \ReqPrioridade{alta}%Prioridade
    \ReqStatus{aprovado}%Status
    \ReqEstabilidade{alta}%Estabilidade
    \ReqOrigem{interna}%Origem
    \ReqRationale{Os crit\'erios escolhidos ser\~ao de car\'ater subjetivo, crit\'erios estes que n\~ao seriam mensur\'aveis utilizando-se tecnologias de automa\c{c}\~ao. Al\'em disso, o uso da escala cont\'{\i}nua \'e importante pois far\'a com que a avalia\c{c}\~ao seja uma vari\'avel quantitativa, o que permitir\'a exprimir estat\'{\i}sticas simples como m\'edia e desvio padr\~ao.}%Rationale
    \ReqAssoc{}%Assoc
\end{Requisito}

%%10
\vfill
\begin{Requisito}
    \ReqTipo{funcional}%Tipo
    \ReqNome{Adicionar reclama\c{c}\~ao}%Nome
    %\ReqLabel{cadUsu}%Label
    \ReqDescr{Para cada "nota" negativa que o usu\'ario d\'a ser\'a mostrado a ele uma caixa de texto para preenchimento de uma reclama\c{c}\~ao [opcional].}%Desc.
    \ReqPrioridade{alta}%Prioridade
    \ReqStatus{aprovado}%Status
    \ReqEstabilidade{alta}%Estabilidade
    \ReqOrigem{interna}%Origem
    \ReqRationale{Ser\'a permitido ao usu\'ario adicionar uma descri\c{c}\~ao do motivo pelo qual ele deu uma nota negativa para aquele determinado crit\'erio, permitindo assim identificar motivos de avalia\c{c}\~oes negativas para cada crit\'erio, o que ajuda na melhoria do servi\c{c}o prestado.}%Rationale
    \ReqAssoc{}%Assoc
\end{Requisito}

%%11
\vfill
\begin{Requisito}
    \ReqTipo{funcional}%Tipo
    \ReqNome{Enviar reclama\c{c}\~ao}%Nome
    %\ReqLabel{cadUsu}%Label
    \ReqDescr{Cada reclama\c{c}\~ao escrita dever\'a estar associada a um endere\c{c}o \'unico a ser enviado, via twitter, \`a \sptrans.}%Desc.
    \ReqPrioridade{alta}%Prioridade
    \ReqStatus{aprovado}%Status
    \ReqEstabilidade{alta}%Estabilidade
    \ReqOrigem{interna}%Origem
    \ReqRationale{É fundamental que o processo de avalia\c{c}\~ao e \textit{accountability} seja o mais p\'ublicos e transparentes poss\'{\i}veis. Dessa forma, conforme apresentado anteriormente \`a \sptrans, as reclama\c{c}\~oes ser\~ao postadas publicamente, com a utiliza\c{c}\~ao de algumas \textit{hashtags} espec\'{\i}ficas pr\'e-definidas para serem respondidas pelo \'org\~ao p\'ublicamente.}%Rationale
    \ReqAssoc{}%Assoc
\end{Requisito}

\clearpage
\subsection{Tabela Resumo dos Requisitos}\label{subsec:tabResReq}
   \begin{table}[H]
        \centering
        \caption{Requisitos funcionais}
        \label{tab:reqFunc}
%        \PrintRequisitos
%        \begin{tabular}{|c|l|}
%            \hline
%            \textbf{Identifica\c{c}\~ao} & \textbf{Requisito}\\
%            \hline
%            \hyperlink{RF1}{RF1} & Cadastrar usu\'ario\\
%            \hline
%            \hyperlink{RF2}{RF2} & Realizar login\\
%            \hline
%            \hyperlink{RF3}{RF3} & Realizar login social\\
%            \hline
%            \hyperlink{RF4}{RF4} & Realizar logout\\
%            \hline
%         \end{tabular}
    \end{table}
%
%    \begin{table}[H]
%        \centering
%        \caption{Requisitos funcionais do m\'odulo de avalia\c{c}\~ao}
%        \label{tab:reqFuncAva}
%        \begin{tabular}{|c|l|}
%            \hline
%            \textbf{Identifica\c{c}\~ao} & \textbf{Requisito} \\
%            \hline
%            \hyperlink{RF5}{RF5} & Listar linhas de \^onibus mais prov\'aveis\\
%            \hline
%            \hyperlink{RF6}{RF6} & Selecionar linha de \^onibus\\
%            \hline
%            \hyperlink{RF7}{RF7} & Coletar posicionamento do usu\'ario\\
%            \hline
%            \hyperlink{RF8}{RF8} & Realizar avalia\c{c}\~ao global\\
%            \hline
%            \hyperlink{RF9}{RF9} & Realizar avalia\c{c}\~oes espec\'{\i}ficas\\
%            \hline
%            \hyperlink{RF10}{RF10} & Adicionar reclama\c{c}\~ao\\
%            \hline
%            \hyperlink{RF11}{RF11} & Enviar reclama\c{c}\~ao\\
%            \hline
%         \end{tabular}
%    \end{table}
%
%    \begin{table}[H]
%        \centering
%        \caption{Requisitos funcionais do m\'odulo de informa\c{c}\~oes}
%        \label{tab:reqFuncInfo}
%        \begin{tabular}{|c|l|}
%            \hline
%            \textbf{Identifica\c{c}\~ao} & \textbf{Requisito} \\
%            \hline
%            \hyperlink{RF1}{RF1} & Cadastrar Usu\'ario \\
%            \hline
%            \hyperlink{RF2}{RF2} & Executar Login Social\\
%            \hline
%            \hyperlink{RF3}{RF3} & Nome 3\\
%            \hline
%         \end{tabular}
%    \end{table}
%
%    \begin{table}[H]
%        \centering
%        \caption{Requisitos funcionais do m\'odulo de game}
%        \label{tab:reqFuncGame}
%        \begin{tabular}{|c|l|}
%            \hline
%            \textbf{Identifica\c{c}\~ao} & \textbf{Requisito} \\
%            \hline
%            \hyperlink{RF4}{RF4} & Nome 4\\
%            \hline
%            \hyperlink{RF5}{RF5} & Nome 5\\
%            \hline
%            \hyperlink{RF6}{RF6} & Nome 6\\
%            \hline
%         \end{tabular}
%    \end{table}
%

\section{Requisitos Não Funcionais}\label{sec: RNF}
\StartReqNFunc
%\vfill
%\begin{Requisito}
%    \ReqTipo{nao funcional}%Tipo
%    \ReqNome{O sistema deverá exigir autenticação para acesso do usuário}%Nome
%%%%    \ReqLabel{Req}%Label
%    \ReqDescr{Para qualquer funcionalidade do sistema, este deve verificar se o usuário possui privilégios para acessar esta função.}%Desc
%    \ReqPrioridade{alta}%Prioridade
%    \ReqStatus{proposto}%Status
%    \ReqEstabilidade{media}%Estabilidade
%    \ReqOrigem{cliente}%Origem
%    \ReqRationale{Necessário para garantir que os dados do negócio do cliente estejam seguros no sistema}%Rationale
%    \ReqAssoc{Funcionalidades do sistema estruturadas}%Assoc
%\end{Requisito}
%
\vfill
%
\begin{Requisito}
    \ReqNome{Autentica\unexpanded{\c{c}}\unexpanded{\~a}o Obrigat\unexpanded{\'o}ria}%Nome
    \ReqTipo{nao funcional}%Tipo
%%%    \ReqLabel{Req}%Label
    \ReqDescr{Para qualquer funcionalidade do sistema, este deve verificar se o usuário está autenticado.}%Desc
    \ReqPrioridade{alta}%Prioridade
    \ReqStatus{aceito}%Status
    \ReqEstabilidade{alta}%Estabilidade
    \ReqOrigem{interna}%Origem
    \ReqRationale{Até este momento todos os serviços do projeto requerem autenticação dos usuários.}%Rationale
    \ReqAssoc{}%Assoc
\end{Requisito}
%
\vfill
%
\begin{Requisito}
    \ReqNome{Garantir acesso apenas a pessoas autorizadas}%Nome
    \ReqTipo{nao funcional}%Tipo
%%%    \ReqLabel{Req}%Label
    \ReqDescr{Para qualquer funcionalidade do sistema, este deve verificar se o usuário possui privilégios para acessar esta função.}%Desc
    \ReqPrioridade{alta}%Prioridade
    \ReqStatus{aceito}%Status
    \ReqEstabilidade{alta}%Estabilidade
    \ReqOrigem{interna}%Origem
    \ReqRationale{Necessário para garantir que nenhum dado ou função seja acessada indevidamente, além de garantir que toda ação dos usuários seja corretamente registrada.}%Rationale
    \ReqAssoc{}%Assoc
\end{Requisito}
%
\vfill
%
\begin{Requisito}
    \ReqNome{Senhas criptografadas na base}%Nome
    \ReqTipo{nao funcional}%Tipo
%%%    \ReqLabel{Req}%Label
    \ReqDescr{Senhas gravadas no banco de dados devem estar criptografadas, nunca em texto plano.}%Desc
    \ReqPrioridade{alta}%Prioridade
    \ReqStatus{aceito}%Status
    \ReqEstabilidade{alta}%Estabilidade
    \ReqOrigem{interna}%Origem
    \ReqRationale{Gravar senhas no banco de dados em texto plano é uma falha grave de segurança em qualquer sistema.}%Rationale
    \ReqAssoc{}%Assoc
\end{Requisito}
%
\vfill
%
\begin{Requisito}
    \ReqNome{Qualidade da senha}%Nome
    \ReqTipo{nao funcional}%Tipo
%%%    \ReqLabel{Req}%Label
    \ReqDescr{Todas as senhas definidas no sistema devem ter ao menos 6 dígitos, contendo ao menos um número, uma letra e um caracter não alfanumérico (.-\_\@\#! dentre outros).}%Desc
    \ReqPrioridade{alta}%Prioridade
    \ReqStatus{aceito}%Status
    \ReqEstabilidade{alta}%Estabilidade
    \ReqOrigem{interna}%Origem
    \ReqRationale{É fundamental que as senhas tenham um nível mínimo de segurança para impedir ataques de ``força bruta''.}%Rationale
    \ReqAssoc{}%Assoc
\end{Requisito}
%
\vfill
%
\begin{Requisito}
    \ReqNome{Qualidade das senhas administrativas}%Nome
    \ReqTipo{nao funcional}%Tipo
%%%    \ReqLabel{Req}%Label
    \ReqDescr{Todas as senhas administrativas definidas no sistema devem ter ao menos 12 dígitos, contendo ao menos um número, uma letra e um caracter não alfanuméricos (.-\_\@\#! dentre outros).}%Desc
    \ReqPrioridade{alta}%Prioridade
    \ReqStatus{aceito}%Status
    \ReqEstabilidade{alta}%Estabilidade
    \ReqOrigem{interna}%Origem
    \ReqRationale{É fundamental que as senhas tenham um nível mínimo de segurança para impedir ataques de ``força bruta'', em especial as com acesso privilegiado.}%Rationale
    \ReqAssoc{}%Assoc
\end{Requisito}
%
\vfill
\clearpage
\vfill
%
\begin{Requisito}
    \ReqNome{Realizar o log das a\unexpanded{\c{c}}\unexpanded{\~o}es dos usu\unexpanded{\'a}rios}%Nome
    \ReqTipo{nao funcional}%Tipo
%%%    \ReqLabel{Req}%Label
    \ReqDescr{Realizar o log (``tracking'') de todas as ações de todos os usuários, registrando-as na base de dados.}%Desc
    \ReqPrioridade{alta}%Prioridade
    \ReqStatus{aceito}%Status
    \ReqEstabilidade{alta}%Estabilidade
    \ReqOrigem{interna}%Origem
    \ReqRationale{Essas informações servirão para posterior análise e melhoria de usabilidade da aplicação.}%Rationale
    \ReqAssoc{}%Assoc
\end{Requisito}
%
\vfill
%
\begin{Requisito}
    \ReqNome{Realizar registro do timing das a\unexpanded{\c{c}}\unexpanded{\~o}es dos usu\unexpanded{\'a}rios}%Nome
    \ReqTipo{nao funcional}%Tipo
%%%    \ReqLabel{Req}%Label
    \ReqDescr{Registrar no log o quanto tempo cada usuário demorou para realizar cada ação, ou quanto dele ele ficou em cada página da aplicação.}%Desc
    \ReqPrioridade{alta}%Prioridade
    \ReqStatus{aceito}%Status
    \ReqEstabilidade{alta}%Estabilidade
    \ReqOrigem{interna}%Origem
    \ReqRationale{Essas informações servirão para posterior análise e melhoria de usabilidade da aplicação.}%Rationale
    \ReqAssoc{}%Assoc
\end{Requisito}
%
\vfill
%
\begin{Requisito}
    \ReqNome{N\unexpanded{\'u}mero m\unexpanded{\'a}ximo de avalia\unexpanded{\c{c}}\unexpanded{\~o}es espec\unexpanded{\'i}ficas}%Nome
    \ReqTipo{nao funcional}%Tipo
%%%    \ReqLabel{Req}%Label
    \ReqDescr{Apresentar no máximo 5 avaliações específicas aos usuários.}%Desc
    \ReqPrioridade{alta}%Prioridade
    \ReqStatus{aceito}%Status
    \ReqEstabilidade{alta}%Estabilidade
    \ReqOrigem{interna}%Origem
    \ReqRationale{Mais do que 5 questões faz com que o questionário fique muito extenso e demorado para ser respondido num dispositivo móvel, dentro de um ônibus em movimento, reduzindo o número de respostas.}%Rationale
    \ReqAssoc{}%Assoc
\end{Requisito}
%
\vfill
%
\begin{Requisito}
    \ReqNome{Reclama\unexpanded{\c{c}}\unexpanded{\~o}es apenas ap\unexpanded{\'o}s salvar avalia\unexpanded{\c{c}}\unexpanded{\~o}es}%Nome
    \ReqTipo{nao funcional}%Tipo
%%%    \ReqLabel{Req}%Label
    \ReqDescr{As telas de reclamação devem aparecer após o usuário salvar suas avaliações.}%Desc
    \ReqPrioridade{alta}%Prioridade
    \ReqStatus{aceito}%Status
    \ReqEstabilidade{alta}%Estabilidade
    \ReqOrigem{interna}%Origem
    \ReqRationale{As avaliações, realizadas com ``\textit{slidebar}'' são muito rápidas. Por isso, permitir que os usuários realizem todas essas avaliações antes de abrir as caixas de ''reclamação'' para que eles digitem as reclamações é fundamental.}%Rationale
    \ReqAssoc{}%Assoc
\end{Requisito}
%
\vfill
%
%\section{Tabela de Casos de Uso}\label{sec:TabCasosDeUso}
%
%%%%%%%%%%%%%%%%%%%MODELO A SER USADO%%%%%%%%%%%%%%%%
%\begin{casodeuso}
%\nomeCdU{} %Nome do Caso de Uso
%\descricaoCdU{} %Descrição
%\eventoiniciadorCdU{} %Evento iniciador
%\atoresCdU{} %Atores
%%Lista de pré-condições
%\precondicaoCdU{} %Pré-condição 1
%\precondicaoCdU{} %Pré-condição 2
%\precondicaoCdU{} %Pré-condição 3
%%Lista de eventos do caso de uso
%\eventosCdU{} %primeiro evento
%\eventosCdU{} %segundo evento
%\eventosCdU{} %terceiro evento
%Lista de pós-condições
%\poscondicaoCdU{} %Pós-condição 1
%\poscondicaoCdU{} %Pós-condição 2
%\poscondicaoCdU{} %Pós-condição 3
%%Lista de casos de uso de extensão
%\extensaoCdU{} %Caso de extensão 1
%\extensaoCdU{} %Caso de extensão 2
%\extensaoCdU{} %Caso de extensão 3
%%Lista de casos de uso de inclusão
%\inclusaoCdU{} %Caso de inclusão 1
%\inclusaoCdU{} %Caso de inclusão 2
%\inclusaoCdU{} %Caso de inclusão 3
%%Lista de requisitos relacionados/atendidos
%\requisitoCdU{} %Requisitos relacionados/atendidos
%\requisitoCdU{} %Requisitos relacionados/atendidos
%\requisitoCdU{} %Requisitos relacionados/atendidos
%\end{casodeuso}
%%%%%%%%%%%%%%%%%%%MODELO A SER USADO%%%%%%%%%%%%%%%%
%
%
%%%%%%%%%%%%%%%%%%%%%%%%%%%%%%%%%%%
%\begin{casodeuso}
%\nomeCdU{Cadastrar Usuário} %Nome do Caso de Uso
%\labelCdU{cad-usu} %Label do Caso de Uso
%\descricaoCdU{Cadastro de um novo usuário no sistema} %Descrição
%\eventoiniciadorCdU{Solicitação de Cadastro ou tentativa de login social com usuário não cadastrado} %Evento iniciador
%\atoresCdU{} %Atores
%%Lista de pré-condições
%\precondicaoCdU{} %Pré-condição 1
%\precondicaoCdU{} %Pré-condição 2
%\precondicaoCdU{} %Pré-condição 3
%%Lista de eventos do caso de uso
%\eventosCdU{} %primeiro evento
%\eventosCdU{} %segundo evento
%\eventosCdU{} %terceiro evento
%Lista de pós-condições
%\poscondicaoCdU{} %Pós-condição 1
%\poscondicaoCdU{} %Pós-condição 2
%\poscondicaoCdU{} %Pós-condição 3
%%Lista de casos de uso de extensão
%\extensaoCdU{} %Caso de extensão 1
%\extensaoCdU{} %Caso de extensão 2
%\extensaoCdU{} %Caso de extensão 3
%%Lista de casos de uso de inclusão
%\inclusaoCdU{} %Caso de inclusão 1
%\inclusaoCdU{} %Caso de inclusão 2
%\inclusaoCdU{} %Caso de inclusão 3
%%Lista de requisitos relacionados/atendidos
%\requisitoCdU{} %Requisitos relacionados/atendidos
%\requisitoCdU{} %Requisitos relacionados/atendidos
%\requisitoCdU{} %Requisitos relacionados/atendidos
%\end{casodeuso}
%%%%%%%%%%%%%%%%%%%MODELO A SER USADO%%%%%%%%%%%%%%%%
%

\section{Tabelas Resumo de Requisitos e Casos de Uso}
\subsection{Tabela de Requisitos Funcionais}\label{subsec:tabResReqF}
   \begin{table}[H]
        \centering
        \caption{Requisitos funcionais}
        \label{tab:reqFunc}
        \PrintRequisitosFunc
    \end{table}
    

\subsection{Tabela de Requisitos Não Funcionais}\label{subsec:tabResReqNF}
   \begin{table}[H]
        \centering
        \caption{Requisitos não funcionais}
        \label{tab:reqNFunc}
        \PrintRequisitosNFunc
    \end{table}

\section{Avaliações}\label{sec:avalia}
\subsection{Avaliações Específicas}\label{subsec:avaliaspec}
Descrever quais as perguntas específicas serão realizadas e o porque delas.

\subsection{Modelo de avaliação}\label{subsec:modavalia}
Descrever os motivos da escolha de uma escala linear contínua;

Descrever se o porque da escolha de uma escala numérica e/ou uma escala não numérica.
