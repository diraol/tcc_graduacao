\chapter{Especificação}\label{chp:Especificação}
O projeto foi desenvolvido para dispositivos móveis utilizando \textit{tecnologia de desenvolvimento híbrida}, visando facilitar o desenvolvimento e o deploy em diversas plataformas móveis. Neste capítulo descreveremos a especificação técnica do \textit{backend} e do aplicativo mobile do projeto.

\section{Tecnologias do \textit{Backend}}\label{sec:spec-backend}
Como solução de \textit{backend} foi utilizado o \textit{framework} \gls{django}, que utiliza a linguagem de programação \textbf{Python} e é organizado segundo a \textit{design pattern} \gls{mvt}, conforme pode ser visto nas Figuras \ref{fig:arqMVT} e \ref{fig:arqDjango}.%
%
\diagramaRetrato{arquitetura-app-django.png}{0.8}{Arquitetura MVT de um APP do framework Django}{arqMVT}{Jeff Croft em {\footnotesize\url{http://www.flickr.com/photos/jcroft/432038560/sizes/o/in/photostream/}}}{}%
%
\diagramaRetrato{django-arq2.eps}{1.1}{Arquitetura do framework Django}{arqDjango}{Autoria própria}{baseado em {\footnotesize\url{http://www.slideshare.net/AbhijeetShekhar1/django-39439148}}}

A estrutura de projeto do \textit{framework} é pensada de forma modular, na qual a aplicação é composta por ''\textit{apps}'' independentes que realizam funções específicas e são conectados no projeto, conforme pode ser visto na Figura \ref{fig:multiApps}.%
%
\diagramaRetrato{django-multi-apps.jpg}{0.45}{Arquitetura multi-aplicativos do Django}{multiApps}{Ian Ward em {\footnotesize\url{http://excess.org/article/2007/06/oclug-django-site/}}}{}

Considerando tal arquitetura, foram desenvolvidos os 3 módulos principais já descritos anteriormente, na Seção \ref{sec:Escopo} (\textbf{Avaliação}, \textbf{Game} e \textbf{GPS social}). Além destes, também foi necessária a criação de um módulo adicional (\textit{utils}) com a finalidade de suprir algumas integrações entre os \textit{apps} sem impactar no isolamento entre eles. A função principal deste módulo foi a de criar a \gls{api} \gls{rest} do projeto

A \gls{api} \gls{rest} foi implementada utilizando-se o pacote \gls{drf}\footnote{\url{http://www.django-rest-framework.org}}. Com o uso do \gls{drf} foi necessário desenvolver algumas classes serializadoras vinculadas aos modelos de dados que seriam expostos (\textit{models}), em seguida criar as respectivas \textit{views} e criar um \textit{Router} com as urls que seriam expostas. Sequer foi necessária a criação de templates, visto que o pacote já fornece um template padrão. O módulo expõe os dados no formato \gls{json}, para além da apresentação na interface web, sem que seja necessário qualquer desenvolvimento ou configuração.

Foi utilizado também o pacote \gls{psa} para permitir a autenticação dos usuários com seus logins de redes sociais. Neste primeiro momento foram disponibilizados os logins via \textit{Facebook}\footnote{\url{https://developers.facebook.com/docs/facebook-login/v2.2}} e via \textit{Google Social Login}\footnote{\url{https://developers.google.com/+/web/signin/}}, ambos utilizando o protocolo de autenticação OAuth 2.0\footnote{\url{http://oauth.net/2/}}. Neste primeiro momento não foi utilizada a rede social \textit{Twitter} pois a mesma não disponibiliza o email do usuário ao realizar o login, o que impede que possamos vincular a conta da rede social com usuários já cadastrados.

Como o aplicativo lida com dados georreferenciados, para \gls{sgbd} foi escolhido o \textbf{PostgreSQL} com a extensão \textbf{PostGis}, visto que esta é a solução mais amplamente utilizada no mercado e com melhor suporte, além de ser a recomendada pelos desenvolvedores do \gls{django}, de sua extensão ``geo'' e da biblioteca python (\textit{gdal (Geospatial Data Abstraction Library)}). A solução escolhida permite inclusive realizar consultas utilizando critérios de geolocalização como, por exemplo, ``Selecionar todos os registros cuja localização se encontra num raio de X metros do ponto Y'', o que é fundamental para a criação do módulo \textbf{mapa}.

O framework \gls{django} por si só não é um webserver - ele apenas possui um microserver para fins de teste e desenvolvimento; então faz-se necessário utilizar um webserver. No presente projeto optou-se pelo \gls{nginx}, o servidor web que tem crescido no ritmo mais acelerado dos que possuem pelo menos 1\% do mercado. Seu ritmo anual médio de crescimento de 2010 a 2014 43\%, e hoje ele já ocupa a segunda colocação como servidor mais utilizado na internet com 22,6\% do mercado, atrás apenas do Apache, que tem 59\% do mercado mas que tem perdido, em média, 3,71\% de \textit{market share} ao ano, conforme pode ser observado na Figura \ref{fig:nginxmarkershare}.

\diagramaRetrato{market_share_webservers.png}{0.55}{\textit{Market Share} de servidores web}{nginxmarkershare}{W3Techs - Web Technology Surveys {\footnotesize\url{http://w3techs.com/technologies/history_overview/web_server/ms/y}}}{}

Além dessa grande exposição, e de possuir uma boa documentação\footnote{\url{http://nginx.org/en/docs/}}, outra vantagem é que ele consegue cumprir a função de webserver, proxy reverso e também possui recursos de \textit{load balance}, tornado-o uma ótima alternativa em termos de escalabilidade.

O \gls{nginx} por si só não consegue "fornecer" diretamente a aplicação django, para tanto é preciso ainda mais um elemento, que é o servidor de aplicações. Para tanto, um dos que apresenta melhor desempenho nos dias de hoje para servir projetos que utilizam a linguagem de programação Python é o \gls{uwsgi}\footnote{\url{https://ivan-site.com/2012/09/benchmark-uwsgi-vs-gunicorn-for-async-workers/}}$^,$\footnote{\url{http://blog.kgriffs.com/2012/12/18/uwsgi-vs-gunicorn-vs-node-benchmarks.html}}$^,$\footnote{\url{http://www.peterbe.com/plog/fcgi-vs-gunicorn-vs-uwsgi}}. Dessa maneira, optamos pelo \gls{uwsgi} como servidor de aplicação trabalhando em conjunto com o \gls{nginx} como servidor web.

Fica como sugestão para o futuro do projeto a utilização do servidor de cache \textit{varnish}\footnote{\url{https://www.varnish-cache.org/}}, conforme recomendado pela equipe do hosting DigitalOcean\footnote{\url{https://www.digitalocean.com/community/tutorials/how-to-scale-django-beyond-the-basics}}, para conseguir escalar o projeto sem precisar necessariamente de mais recursos de máquina.

\section{Modelagem de Dados}\label{sec:diagrama-er}
Na figura \ref{fig:DiagER} encontra-se a modelagem de dados realizada inicialmenten no projeto. Esta modelagem não leva em consideração as especificidades do \textit{framework} utilizado, e eventuais diferenças serão expostas na descrição dos \textit{apps}.
    \diagramaRetrato{diagramas_er_bds.eps}{1.2}{Diagrama Entidade Relacionamento}{DiagER}{Autoria Própria}{}

\section{Implementação}\label{sec:estrutura-app}
Nesta seção iremos descrever a rotina de implementação da infraestrutura de backend como um todo e a implementação de cada um dos \textit{apps}, assim como do projeto \gls{django} que integra os \textit{apps}. Cada \textit{app} será descrito considerando o \textit{design pattern} \gls{mvt}.

A implementação completa dos \textit{models}, das \textit{views}, \textit{serializers} e \textit{urls} pode ser encontrada no Anexo \ref{anexo:sources} ou no repositório oficial do projeto\footnote{Repositório Oficial do Projeto: \url{http://github.com/diraol/trilhasp}}

\subsection{Setup Inicial}
Para o \textit{setup} inicial do servidor foi criado um \textit{shell script} que realiza toda a instalação. Ele foi desenvolvido e testado para o sistema operacional \textbf{Debian Jessie (8.0)}.

Este setup inicial contempla a instalação do \gls{django}, do \gls{sgbd} \textbf{PostgreSQL} (v9.4), com sua extensão de dados espaciais \textbf{PostGis} (v2.1), além dos pacotes necessários para a utilização do recurso de ambientes virtuais (\textit{Virtualenv}) do \textbf{Python}, o que facilita o encapsulamento e a manutenabilidade da aplicação num servidor.

Requisitos para a utilização de ambientes virtuais: %
\begin{enumerate*}[label=\itshape\alph*\upshape)] 
    \item \mbox{\textit{python-setuptools}};
    \item \mbox{\textit{python-pip}};
    \item \mbox{\textit{python-dev}}
\end{enumerate*}.

Requisitos para a base de dados espaciais e sua integração com o Python: %
\begin{enumerate*}[label=\itshape\alph*\upshape)]
    \item\mbox{\textit{postgresql-9.4}};
    \item\mbox{\textit{postgresql-contrib-9.4}};
    \item\mbox{\textit{postgresql-9.4-postgis}};
    \item\mbox{\textit{postgresql-server-dev-9.4}};
    \item\mbox{\textit{libpq-dev}};
    \item\mbox{\textit{binutils}};
    \item\mbox{\textit{libproj-dev}};
    \item\mbox{\textit{gdal-bin}};
    \item\mbox{\textit{python-gdal}};
    \item\mbox{\textit{python-psycopg2}}
\end{enumerate*}.

Após instalados os pacotes, o script irá realizar a criação de um usuário \textit{trilhasp} no PostgreSQL, será requisitado ao usuário para inserir um \textit{password}, será criada a base de dados dentro do \gls{sgbd}, com encoding UTF-8, serão instaladas as extensões de georreferenciamento nesta base.

Em seguida é instalado o pacote \textit{virtualenv} do \textbf{Python}, com o uso da ferramenta de gerenciamento de pacotes \textit{pip}, é criado um novo ambiente virtual (numa pasta chamada \textbf{venv}), que é ativado e no qual são instalados os demais \textit{requirements} de python, que ficam definidos no arquivo \textit{requirements.txt}.

Finalizando o setup inicial do \gls{django}, o script irá executar os comandos para criação das bases de dados, tomando como referência os modelos (\textit{models}) de todas as aplicações listadas nas configurações do django.

\subsection{Configuração do Django}
A configuração do django se dá basicamente no arquivo \textit{settings.py} do projeto....
%TODO

%%%%%%%%%%%%%%%%%%%%%%%%%%%%%%%%%%%%%%%%%%%%%%%%%%%%%%%%%%%%%%%%%%%%%%%%%%%%%
%%                              APP AVALIAÇÃO                              %%
%%%%%%%%%%%%%%%%%%%%%%%%%%%%%%%%%%%%%%%%%%%%%%%%%%%%%%%%%%%%%%%%%%%%%%%%%%%%%
\subsection{APP Avaliação}
Começaremos pelo  \textit{app} responsável pela avaliação do sistema de transporte.

\subsubsection{Camada de Acesso a Dados (\textit{model})}\label{subsubsec:eval-camada-model}
Foram definidos os seguintes modelos, que representam classes, e que possuem os respectivos atributos:
\begin{description}
    \item[BusCompanies] - Empresa de ônibus
        \begin{itemize}
            \item \textbf{company\_name} Tipo: CharField - Nome da empresa
            \item \textbf{logo} Tipo: ImageField - Logo da empresa
        \end{itemize}
    \item[BusLine] - Linha de ônibus
        \begin{itemize}
            \item \textbf{bus\_line\_code} Tipo: CharField - Código da linha de ônibus
            \item \textbf{going\_bus\_name} Tipo: CharField - Nome do trajeto de ida
            \item \textbf{return\_bus\_name} Tipo: CharField - Nome do trajeto de volta
            \item \textbf{active} Tipo: BooleanField - Estado da linha (ativa ou desativa)
            \item \textbf{company\_name} Tipo: ForeignKey - Empresa responsável pela linha
        \end{itemize}
        Chave primária composta: \textit{bus\_line\_code} e \textit{active}
    \item[Buses] - Ônibus único
        \begin{itemize}
            \item \textbf{bus\_unique\_number} Tipo: IntegerField - Número único de cada ``carro'' (ônibus), pintado na lateral do mesmo.
            \item \textbf{bus\_line\_code} Tipo: ForeignKey - Código da linha que este ônibus percorre
            \item \textbf{active} Tipo: BooleanField - Estado da linha (ativa ou desativa)
        \end{itemize}
        Chave primária composta: \textit{bus\_unique\_number} e \textit{active}
    \item[EVALAnswerModel] - Modelos de respostas
        \begin{itemize}
            \item \textbf{answer} Tipo: CharField - Texto explicativo deste tipo de resposta
            \item \textbf{lower\_limit\_text} Tipo: CharField - Label do valor superior
            \item \textbf{upper\_limit\_text} Tipo: CharField - Label do valor inferior
            \item \textbf{middle\_text} Tipo: CharField - Label do valor central
            \item \textbf{lower\_limit\_value} Tipo: IntegerField - Valor do limite inferior
            \item \textbf{upper\_limit\_value} Tipo: IntegerField - Valor do limite superior
            \item \textbf{middle\_value} Tipo: IntegerField - Valor central
        \end{itemize}
    \item[EVALQuestion] - Perguntas
        \begin{itemize}
            \item \textbf{question} Tipo: CharField - Texto da questão, que será mostrado ao usuário
            \item \textbf{answer} Tipo: ForeignKey - Tipo da resposta para esta pergunta
            \item \textbf{enabled} Tipo: Estado da pergunta (ativa ou desativa)
        \end{itemize}
    \item[EVALAnswer] - Respostas dos usuários
        \begin{itemize}
            \item \textbf{question} Tipo: ForeignKey - Questão
            \item \textbf{user} Tipo: ForeignKey - Usuário respondente
            \item \textbf{timestamp} Tipo: DateTimeField - Data e Horário da resposta
            \item \textbf{answer\_value} Tipo: IntegerField - Valor numérico da resposta
            \item \textbf{answer\_text} Tipo: CharField - Texto inserido pelo usuário (em caso de resposta abaixo do valor central)
            \item \textbf{bus\_unique\_number} Tipo: ForeignKey - Número único do ônibus
            \item \textbf{geolocation} Tipo: PointField - Geolocalização do usuário no momento da resposta
        \end{itemize}
\end{description}

\subsubsection{Serializadores}
Para a criação da \gls{api} o \gls{drf} requer que o desenvolvedor crie um (ou mais) serializador(es) para cada classe que deseja expor e, em seguida, crie as views que utilizam esses serializadores no lugar dos ``\textit{models}''.

Assim, para este \textit{app} foram criados os seguintes serializadores: %
\begin{itemize}
    \item \textit{BusCompaniesSerializer};
    \item \textit{BusLineSerializer};
    \item \textit{BusesSerializer};
    \item \textit{EVALAnswerModelSerializer};
    \item \textit{EVALQuestionSerializer};
    \item \textit{EVALAnswerSerializer}
\end{itemize}
Todos eles foram criados como extensões da classe \textit{serializers.HyperlinkedModelSerializer}, o que faz com que a \gls{api} exposta faça o link entre objetos que são chave-estrangeira e o objeto que está sendo mostrado.

\subsubsection{Camada de lógica de negócio (\textit{views})}\label{subsubsec:eval-camada-view}
Esta camada irá expor os serializadores definidos anteriormente. As \textit{views} criadas foram: %
\begin{itemize}
    \item \textit{BusCompanyViewSet};
    \item \textit{BusLineViewSet};
    \item \textit{BusesViewSet};
    \item \textit{EVALAnswerModelViewSet};
    \item \textit{EVALQuestionViewSet};
    \item \textit{EVALAnswerViewSet}
\end{itemize}
Estas classes foram definidas como extensões da classe ViewSet do \gls{drf}, o que já contempla automaticamente a criação de \textit{views} para lista de resultados e também para um resultado único em cada visão. Além disso, é na definição destas classes que podemos especificar as permissões de acesso, caso desejamos uma diferente da default configurada.

\subsubsection{Camada de Template e URLs}
Como não foi criada nenhuma ``página'' publicamente acessível, não foi necessária a criação de template - o \gls{drf} utiliza um template default para a \gls{api} no browser. Quanto às URLs, elas serão definidas no arquivo de URLs do projeto \gls{django}.

\subsubsection{Camada de Adminsitração}
O \gls{django} oferece por padrão um módulo de administração da aplicação, bastando ativá-lo no arquivo de configurações. Em seguida, para cada \textit{app} é necessário registrar quais são as classes (do modelo) que estadão disponíveis para administração, e, eventualmente, fazer alguma personalização na forma como a mesma é exposta.

No caso deste \textit{app} todas as classes foram expostas e ainda adicionou-se uma configuração para que a informação georreferenciada fosse apresentada num widget de mapa, ao invés de se apresentar uma coordenada apenas. Esta configuração encontra-se no arquivo \textit{admin.py} do \textit{app} e também está disponível no Anexo \ref{anexo:sources}.

\clearpage
%%%%%%%%%%%%%%%%%%%%%%%%%%%%%%%%%%%%%%%%%%%%%%%%%%%%%%%%%%%%%%%%%%%%%%%%%%%%%
%%                              APP GPSSocial                              %%
%%%%%%%%%%%%%%%%%%%%%%%%%%%%%%%%%%%%%%%%%%%%%%%%%%%%%%%%%%%%%%%%%%%%%%%%%%%%%
\subsection{APP GPSSocial}
Agora veremos o \textit{app} responsável pelo mapa de GPS Social.

\subsubsection{Camada de Acesso a Dados (\textit{model})}\label{subsubsec:gps-camada-model}
Foram definidos os seguintes modelos, que representam classes, e que possuem os respectivos atributos:
\begin{description}
    \item[GEOLastPosition] - Tabela com a última posição de cada usuário
        \begin{itemize}
            \item \textbf{user} Tipo: ForeignKey - Usuário;
            \item \textbf{geolocation} Tipo: PointField - Localização do usuário;
            \item \textbf{timestamp} Tipo: DateTimeField - Horário do registro da localização.
        \end{itemize}
    \item[GEOHistoryPosition] - Histórico de localização de todos os usuários
        \begin{itemize}
            \item \textbf{user} Tipo: ForeignKey - Usuário;
            \item \textbf{geolocation} Tipo: PointField - Localização do usuário;
            \item \textbf{timestamp} Tipo: DateTimeField - Horário do registro da localização.
        \end{itemize}
\end{description}
Optou-se por criar uma tabela com a última localização dos usuários para melhorar o desempenho nas consultas para o GPS social, reduzindo o número de registros a serem consultados para esta funcionalidade.

\subsubsection{Serializadores}
Para este \textit{app} foram criados os seguintes serializadores: %
\begin{itemize}
    \item \textit{GEOLastPositionSerializer};
    \item \textit{GEOLastAnonPositionSerializer};
    \item \textit{GEOHistoryPositionSerializer}.
\end{itemize}
Os serializadores \textit{GEOLastPositionSerializer} e \textit{GEOLastAnonPositionSerializer} são extensões da classe \mbox{\textit{GeoFeatureModelSerializer}} (do módulo \gls{drf}-geo), e o serializador \mbox{\textit{GEOHistoryPositionSerializer}} é uma extensão da classe \mbox{\textit{HyperlinkedModelSerializer}}.

Aqui nota-se que há mais serializadores que classes no modelo. Isto ocorreu pois foi criado um serializador para oferecer as informações dos usuários de forma anonimizada, garantindo a privacidade dos usuários.

\subsubsection{Camada de lógica de negócio (\textit{views})}
Esta camada irá expor os serializadores definidos anteriormente. As \textit{views} criadas foram: %
\begin{itemize}
    \item \textit{GEOLastPositionViewSet};
    \item \textit{LastUserPosition};
    \item \textit{LastUsersAtPosition};
    \item \textit{GEOHistoryPositionViewSet}
\end{itemize}
Aqui percebemos quatro classes. Duas classes padrões (\mbox{\textit{GEOLastPositionViewSet}} e \mbox{\textit{GEOHistoryPositionViewSet}}), que basicamente expõe os dados dos respectivos modelos, e duas classes mais específicas. Em ambos os casos, se o usuário não é administrador ele não tem como saber a qual usuário cada registro está vinculado (a informação é anonimizada).

A classe \textit{LastUserPosition}, que estará vinculada a uma URL que permitirá obter a última posição de um determinado usuário, por seu username. 

E a classe \textit{LastUsersAtPosition}, que retorna os registros (anonimizados) que estão a uma distância pré-determinada de um determinado ponto (latitude e longitue) e que tenha sido atualizado há menos de uma hora.

\subsubsection{Camada de Template e URLs}
Como não foi criada nenhuma ``página'' publicamente acessível, não foi necessária a criação de template - o \gls{drf} utiliza um template default para a \gls{api} no browser. Quanto às URLs, elas serão definidas no arquivo de URLs do projeto \gls{django}.

\subsubsection{Camada de Adminsitração}
Neste \textit{app} foram expostas as duas classes do ``modelo''.

\clearpage
%%%%%%%%%%%%%%%%%%%%%%%%%%%%%%%%%%%%%%%%%%%%%%%%%%%%%%%%%%%%%%%%%%%%%%%%%%%%%
%%                                 APP Game                                %%
%%%%%%%%%%%%%%%%%%%%%%%%%%%%%%%%%%%%%%%%%%%%%%%%%%%%%%%%%%%%%%%%%%%%%%%%%%%%%
\subsection{APP Game}
Agora veremos o \textit{app} responsável pelo game.
%TODO %TODO %TODO %TODO %TODO %TODO %TODO %TODO %TODO %TODO %TODO %TODO %TODO %TODO %TODO
\subsubsection{Camada de Acesso a Dados (\textit{model})}
Foram definidos os seguintes modelos, que representam classes, e que possuem os respectivos atributos:
\begin{description}
    \item[GameCoinModel] - 
        \begin{itemize}
            \item \textbf{name} Tipo: CharField - ;
            \item \textbf{value} Tipo: PositiveIntegerField - ;
            \item \textbf{enabled} Tipo: BooleanField - .
        \end{itemize}
    \item[GameFinance] - 
        \begin{itemize}
            \item \textbf{user} Tipo: ForeignKey - ;
            \item \textbf{coin\_model} Tipo: ForeignKey - ;
            \item \textbf{amount} Tipo: PositiveIntegerField - .
        \end{itemize}
    \item[BusBrand] - 
        \begin{itemize}
            \item \textbf{label} Tipo: CharField - ;
            \item \textbf{name} Tipo: CharField - ;
            \item \textbf{logo} Tipo: ImageField - ;
            \item \textbf{enabled} Tipo: BooleanField - .
        \end{itemize}
    \item[GameBusModel] - 
        \begin{itemize}
            \item \textbf{name} Tipo: CharField - ;
            \item \textbf{bus\_brand} Tipo: ManyToManyField - ;
            \item \textbf{efficiency} Tipo: PositiveIntegerField - ;
            \item \textbf{price} Tipo: PositiveIntegerField - .
        \end{itemize}
    \item[GameBusAvailability] - 
        \begin{itemize}
            \item \textbf{bus\_model} Tipo: ForeignKey - ;
            \item \textbf{available\_buses} Tipo: PositiveIntegerField - .
        \end{itemize}
    \item[GamePersonalBusFleet] - 
        \begin{itemize}
            \item \textbf{user} Tipo: ForeignKey - ;
            \item \textbf{bus\_model} Tipo: ForeignKey - ;
            \item \textbf{amount} Tipo: PositiveIntegerField - ;
            \item \textbf{last\_payment} Tipo: DateTimeField - .
        \end{itemize}
\end{description}
Optou-se por criar uma tabela com a última localização dos usuários para melhorar o desempenho nas consultas para o GPS social, reduzindo o número de registros a serem consultados para esta funcionalidade.

\subsubsection{Serializadores}
Para este \textit{app} foram criados os seguintes serializadores: %
\begin{itemize}
    \item \textit{};
    \item \textit{};
    \item \textit{}.
\end{itemize}
Os serializadores \textit{GEOLastPositionSerializer} e \textit{GEOLastAnonPositionSerializer} são extensões da classe \mbox{\textit{GeoFeatureModelSerializer}} (do módulo \gls{drf}-geo), e o serializador \mbox{\textit{GEOHistoryPositionSerializer}} é uma extensão da classe \mbox{\textit{HyperlinkedModelSerializer}}.

Aqui nota-se que há mais serializadores que classes no modelo. Isto ocorreu pois foi criado um serializador para oferecer as informações dos usuários de forma anonimizada, garantindo a privacidade dos usuários.

\subsubsection{Camada de lógica de negócio (\textit{views})}
Esta camada irá expor os serializadores definidos anteriormente. As \textit{views} criadas foram: %
\begin{itemize}
    \item \textit{};
    \item \textit{};
    \item \textit{};
    \item \textit{}
\end{itemize}
Aqui percebemos quatro classes. Duas classes padrões (\mbox{\textit{GEOLastPositionViewSet}} e \mbox{\textit{GEOHistoryPositionViewSet}}), que basicamente expõe os dados dos respectivos modelos, e duas classes mais específicas. Em ambos os casos, se o usuário não é administrador ele não tem como saber a qual usuário cada registro está vinculado (a informação é anonimizada).

A classe \textit{LastUserPosition}, que estará vinculada a uma URL que permitirá obter a última posição de um determinado usuário, por seu username. 

E a classe \textit{LastUsersAtPosition}, que retorna os registros (anonimizados) que estão a uma distância pré-determinada de um determinado ponto (latitude e longitue) e que tenha sido atualizado há menos de uma hora.

\subsubsection{Camada de Template e URLs}
Como não foi criada nenhuma ``página'' publicamente acessível, não foi necessária a criação de template - o \gls{drf} utiliza um template default para a \gls{api} no browser. Quanto às URLs, elas serão definidas no arquivo de URLs do projeto \gls{django}.

\subsubsection{Camada de Adminsitração}
Neste \textit{app} foram expostas as duas classes do ``modelo''.


























\section{Aplicativo Mobile}\label{sec:spec-appmobile}

Já para o desenvolvimento Para o \textit{frontend} do projeto será utilizado \textbf{HTML5}, \textbf{CSS3} e \textbf{JavaScript}, além de algumas bibliotecas auxiliares como \textbf{Twitter Bootstrap} (versão 3).

\section{Requisitos Funcionais}
Nesta se\c{c}\~ao s\~ao apresentados os Requisitos Funcionais do \trilhasp~que devem ser satisfeitos. Estes requisitos devem ainda ser indicados no pr\'oprio código fonte do projeto, assim como nos testes que verifiquem o requisito.
\StartReqFunc
%
\vfill
%
\begin{Requisito}
    \ReqNome{Cadastrar usu\unexpanded{\'a}rio}%Nome
    \ReqLabel{cad-usu}%Label
    \ReqTipo{funcional}%Tipo
    \ReqDescr{O usu\'ario realiza seu cadastro na plataforma.}%Desc.
    \ReqPrioridade{alta}%Prioridade
    \ReqStatus{aprovado}%Status
    \ReqEstabilidade{alta}%Estabilidade
    \ReqOrigem{interna}%Origem
    \ReqRationale{Permite realizar um ``tracking'' do hist\'orico dos usu\'arios para oferecer informa\c{c}\~oes mais precisas, al\'em de permitir que o usu\'ario utilize o ``game'', que demanda um ac\'umulo de participa\c{c}\~ao na plataforma.}%Rationale
    \ReqAssoc{}%Assoc
\end{Requisito}
%
\vfill
%
\begin{Requisito}
    \ReqNome{Realizar login}%Nome
    \ReqLabel{login}%Label
    \ReqTipo{funcional}%Tipo
    \ReqDescr{O usu\'ario deve fazer login para ter acesso ao sistema, impedindo que pessoas n\~ao autorizadas tenham acesso a certas fun\c{c}\~oes do sistema.}%Desc.
    \ReqPrioridade{alta}%Prioridade
    \ReqStatus{aprovado}%Status
    \ReqEstabilidade{alta}%Estabilidade
    \ReqOrigem{usuario}%Origem
    \ReqRationale{M\'etodo que previne que pessoas n\~ao autorizadas tenham acesso \'as funcionalidades do sistema, em especial as que alteram o banco de dados.}%Rationale
    \ReqAssoc{Realizar login social}%Assoc
\end{Requisito}
%
\vfill
%
\begin{Requisito}
    \ReqNome{Realizar login social}%Nome
    \ReqLabel{login-social}%Label
    \ReqTipo{funcional}%Tipo
    \ReqDescr{Usu\'ario utiliza seu login de redes sociais para se conectar \`a plataforma}%Desc.
    \ReqPrioridade{alta}%Prioridade
    \ReqStatus{aprovado}%Status
    \ReqEstabilidade{alta}%Estabilidade
    \ReqOrigem{usuario}%Origem
    \ReqRationale{Para aumentar a facilidade de acesso \`a plataforma \'e fundamental permitir que os usu\'arios acessem o sistema utilizando sistemas de autentica\c{c}\~ao cruzada com redes sociais, principalmente \textit{Facebook}, \textit{Twitter} e \textit{Google}. Ao realizar o login social a aplicação deve verificar o email do usuário e, caso já esteja cadastrado no sistema, deve integrar os usuários.}%Rationale
    \ReqAssoc{Realizar Login}%Assoc
\end{Requisito}
%
\vfill
%
\begin{Requisito}
    \ReqNome{Realizar logout}%Nome
    \ReqLabel{logout}%Label
    \ReqTipo{funcional}%Tipo
    \ReqDescr{O usu\'ario deve fazer logout ao terminar de utilizar o sistema, impedindo que pessoas
n\~ao autorizadas tenham acesso a certas fun\c{c}\~oes do sistema.}%Desc.
    \ReqPrioridade{alta}%Prioridade
    \ReqStatus{aprovado}%Status
    \ReqEstabilidade{alta}%Estabilidade
    \ReqOrigem{usuario}%Origem
    \ReqRationale{M\'etodo que previne que pessoas n\~ao autorizadas tenham acesso \'as funcionalidades
do sistema, em especial as que alteram o banco de dados.}%Rationale
    \ReqAssoc{Realizar Login}%Assoc
\end{Requisito}
%
\vfill
%
\begin{Requisito}
    \ReqNome{Recuperar senha}%Nome
    \ReqLabel{rec-senha}%Label
    \ReqTipo{funcional}%Tipo
    \ReqDescr{O usuário com senha no sistema pode redefinir sua senha acessando um endereço enviado por email para o email de cadastro. A senha não poderá ser enviada por email.}%Desc.
    \ReqPrioridade{alta}%Prioridade
    \ReqStatus{aprovado}%Status
    \ReqEstabilidade{alta}%Estabilidade
    \ReqOrigem{usuario}%Origem
    \ReqRationale{Apesar de o padrão previsto ser que os usuários acessem o sistema utilizando login de redes sociais, ao menos os administradores possuirão logins ``locais''. Dessa forma, é importante haver a possibilidade de redefinir a senha caso a mesma seja esquecida.}%Rationale
    \ReqAssoc{Realizar Login}%Assoc
\end{Requisito}
%
\vfill
%
\begin{Requisito}
    \ReqNome{Coletar posicionamento do usu\unexpanded{\'a}rio}%Nome
    \ReqLabel{posic-usu}%Label
    \ReqTipo{funcional}%Tipo
    \ReqDescr{O ``sistema'' deve coletar a informa\c{c}\~ao de geolocaliza\c{c}\~ao do usu\'ario, usando GPS e outros m\'etodos como Triangula\c{c}\~ao via 3G [usar API de geolocaliza\c{c}\~ao do HTML5] e salvar essa informação a cada 2 minutos nas duas tabelas do módulo \textbf{GPS Social}.}%Desc.
    \ReqPrioridade{alta}%Prioridade
    \ReqStatus{aprovado}%Status
    \ReqEstabilidade{alta}%Estabilidade
    \ReqOrigem{interna}%Origem
    \ReqRationale{Para ter o registro de localização dos usuários, tanto para mostrar o mapa em tempo real de demanda nos pontos quanto para ter um histórico da demanda em cada ponto, por dia e horário, para futuras análises e relatórios do sistema e também para registro da localização no momento da avaliação do ônibus.}%Rationale
    \ReqAssoc{}%Assoc
\end{Requisito}
%
\vfill
%
\begin{Requisito}
    \ReqNome{Mostrar usu\unexpanded{\'a}rios conectados no mapa}%Nome
    \ReqLabel{mapa-demanda}%Label
    \ReqTipo{funcional}%Tipo
    \ReqDescr{Apresentar um mapa, centralizado na localidade do usuário, mostrando os pontos de ônibus da região e a quantidade de usuários aguardando em cada ponto.}%Desc.
    \ReqPrioridade{alta}%Prioridade
    \ReqStatus{aprovado}%Status
    \ReqEstabilidade{alta}%Estabilidade
    \ReqOrigem{interna}%Origem
    \ReqRationale{Permite que o usuário tenha uma noção da demanda instantânea do sistema e possa tomar uma decisão mais consciente de quando ir para o ponto de ônibus.}%Rationale
    \ReqAssoc{Coletar posicionamento do usu\noexpand\'ario}%Assoc
\end{Requisito}
%
\vfill
%
\begin{Requisito}
    \ReqNome{Identificar o \unexpanded{\^o}nibus}%Nome
    \ReqLabel{ident-linha}%Label
    \ReqTipo{funcional}%Tipo
    \ReqDescr{O usu\'{a}rio seleciona a linha de \^{o}nibus que deseja avaliar tirando uma foto do \gls{qrcode} presente no \^{o}nibus.}%Desc.
    \ReqPrioridade{alta}%Prioridade
    \ReqStatus{aprovado}%Status
    \ReqEstabilidade{alta}%Estabilidade
    \ReqOrigem{interna}%Origem
    \ReqRationale{O usu\'{a}rio precisa conseguir selecionar o \^{o}nibus para poder avali\'{a}-lo.}%Rationale
    \ReqAssoc{}%Assoc
\end{Requisito}
%
\vfill
%
\begin{Requisito}
    \ReqNome{Realizar avalia\unexpanded{\c{c}}\unexpanded{\~a}o global}%Nome
    \ReqLabel{avalia-glob}%Label
    \ReqTipo{funcional}%Tipo
    \ReqDescr{O usu\'ario avalia globalmente o servi\c{c}o de transporte p\'ublico com uma nota numa escala cont\'{\i}nua, utilizando um \textit{slider}.}%Desc.
    \ReqPrioridade{alta}%Prioridade
    \ReqStatus{aprovado}%Status
    \ReqEstabilidade{alta}%Estabilidade
    \ReqOrigem{interna}%Origem
    \ReqRationale{É importante uma avalia\c{c}\~ao global, simples e direta, para o caso do usu\'ario n\~ao querer preencher todas as avalia\c{c}\~oes espec\'{\i}ficas. Al\'em disso, o uso da escala cont\'{\i}nua \'e importante pois far\'a com que a avalia\c{c}\~ao seja uma vari\'avel quantitativa, o que permitir\'a exprimir estat\'{\i}sticas simples como m\'edia e desvio padr\~ao.}%Rationale
    \ReqAssoc{}%Assoc
\end{Requisito}
%
\vfill
\clearpage
\vfill
%
\begin{Requisito}
    \ReqNome{Realizar avalia\unexpanded{\c{c}}\unexpanded{\~o}es espec\unexpanded{\'i}ficas}%Nome
    \ReqLabel{avalia-espec}%Label
    \ReqTipo{funcional}%Tipo
    \ReqDescr{O usu\'ario avalia cada um dos crit\'erios espec\'{\i}ficos definidos como uma nota numa escala cont\'{\i}nua para cada um, utilizando um \textit{slider}.}%Desc.
    \ReqPrioridade{alta}%Prioridade
    \ReqStatus{aprovado}%Status
    \ReqEstabilidade{alta}%Estabilidade
    \ReqOrigem{interna}%Origem
    \ReqRationale{Os crit\'erios escolhidos ser\~ao de car\'ater subjetivo, crit\'erios estes que n\~ao seriam mensur\'aveis utilizando-se tecnologias de automa\c{c}\~ao. Al\'em disso, o uso da escala cont\'{\i}nua \'e importante pois far\'a com que a avalia\c{c}\~ao seja uma vari\'avel quantitativa, o que permitir\'a exprimir estat\'{\i}sticas simples como m\'edia e desvio padr\~ao.}%Rationale
    \ReqAssoc{}%Assoc
\end{Requisito}
%
\vfill
%
\begin{Requisito}
    \ReqNome{Adicionar reclama\unexpanded{\c{c}}\unexpanded{\~a}o}%Nome
    \ReqLabel{adic-recl}%Label
    \ReqTipo{funcional}%Tipo
    \ReqDescr{Para cada ``nota'' negativa que o usu\'ario d\'a ser\'a mostrado a ele uma caixa de texto para preenchimento de uma reclama\c{c}\~ao [opcional].}%Desc.
    \ReqPrioridade{alta}%Prioridade
    \ReqStatus{aprovado}%Status
    \ReqEstabilidade{alta}%Estabilidade
    \ReqOrigem{interna}%Origem
    \ReqRationale{Ser\'a permitido ao usu\'ario adicionar uma descri\c{c}\~ao do motivo pelo qual ele deu uma nota negativa para aquele determinado crit\'erio, permitindo assim identificar motivos de avalia\c{c}\~oes negativas para cada crit\'erio, o que ajuda na melhoria do servi\c{c}o prestado.}%Rationale
    \ReqAssoc{}%Assoc
\end{Requisito}
%
\vfill
%
\begin{Requisito}
    \ReqNome{Enviar reclama\unexpanded{\c{c}}\unexpanded{\~a}o}%Nome
    \ReqLabel{envia-recl}%Label
    \ReqTipo{funcional}%Tipo
    \ReqDescr{Cada reclama\c{c}\~ao escrita dever\'a estar associada a um endere\c{c}o \'unico a ser enviado, via twitter, \`a \sptrans.}%Desc.
    \ReqPrioridade{alta}%Prioridade
    \ReqStatus{aprovado}%Status
    \ReqEstabilidade{alta}%Estabilidade
    \ReqOrigem{interna}%Origem
    \ReqRationale{É fundamental que o processo de avalia\c{c}\~ao e \textit{accountability} seja o mais p\'ublicos e transparentes poss\'{\i}veis. Dessa forma, conforme apresentado anteriormente \`a \sptrans, as reclama\c{c}\~oes ser\~ao postadas publicamente, com a utiliza\c{c}\~ao de algumas \textit{hashtags} espec\'{\i}ficas pr\'e-definidas para serem respondidas pelo \'org\~ao p\'ublicamente.}%Rationale
    \ReqAssoc{}%Assoc
\end{Requisito}
%
\vfill
%
\begin{Requisito}
    \ReqNome{Configurar recompensas por avalia\unexpanded{\c{c}}\unexpanded{\~a}o realizada}%Nome
    \ReqLabel{config-recomp}%Label
    \ReqTipo{funcional}%Tipo
    \ReqDescr{Os administradores do sistema devem poder configurar quantas moedas ser\~ao dadas aos usu\'arios ap\'os estes realizarem avalia\c{c}\~oes do Servi\c{c}o de \^onibus. O administrador deve poder definir a recomepnsa ``por pergunta'' ou ``por conjunto de perguntas''.}%Desc.
    \ReqPrioridade{alta}%Prioridade
    \ReqStatus{aprovado}%Status
    \ReqEstabilidade{alta}%Estabilidade
    \ReqOrigem{interna}%Origem
    \ReqRationale{A modifica\c{c}\~ao do valor da recompensa por avali\c{c}\~ao deve estar dispon\'{\i}vel apenas para administradores do sistema, e deve ser de f\'acil realiza\c{c}\~ao.}%Rationale
    \ReqAssoc{}%Assoc
\end{Requisito}
%
\vfill
%
\begin{Requisito}
    \ReqNome{Verificar saldo de ``moedas''}%Nome
    \ReqLabel{ver-saldo}%Label
    \ReqTipo{funcional}%Tipo
    \ReqDescr{O usuário irá verificar o saldo que possui de cada tipo moeda.}%Desc.
    \ReqPrioridade{alta}%Prioridade
    \ReqStatus{aprovado}%Status
    \ReqEstabilidade{alta}%Estabilidade
    \ReqOrigem{interna}%Origem
    \ReqRationale{O usuário precisa conseguir avaliar seu saldo de moedas.}%Rationale
    \ReqAssoc{}%Assoc
\end{Requisito}
%
\vfill
%
\begin{Requisito}
    \ReqNome{Verificar sua frota pessoal}%Nome
    \ReqLabel{ver-frota}%Label
    \ReqTipo{funcional}%Tipo
    \ReqDescr{O usuário irá verificar quantos ônibus de cada tipo ele possui. Devem ser mostradas a idade de cada carro e seu rendimento médio}%Desc.
    \ReqPrioridade{alta}%Prioridade
    \ReqStatus{aprovado}%Status
    \ReqEstabilidade{alta}%Estabilidade
    \ReqOrigem{interna}%Origem
    \ReqRationale{O usuário precisa conhecer sua frota pessoal.}%Rationale
    \ReqAssoc{}%Assoc
\end{Requisito}
%
\vfill
%
\begin{Requisito}
    \ReqNome{Verificar o rendimento de cada tipo de \unexpanded{\^o}nibus}%Nome
    \ReqLabel{ver-rend}%Label
    \ReqTipo{funcional}%Tipo
    \ReqDescr{O usuário deve conseguir verificar o rendimento (em moedas) de cada tipo ônibus de sua frota, por período de tempo.}%Desc.
    \ReqPrioridade{alta}%Prioridade
    \ReqStatus{aprovado}%Status
    \ReqEstabilidade{alta}%Estabilidade
    \ReqOrigem{interna}%Origem
    \ReqRationale{É importante o usuário avaliar cada carro de sua frota, considerando idade e rendimento, para saber se ele irá se desfazer (vender), fazer algum tipo de manutenção/melhoria ou comprar novos veículos.}%Rationale
    \ReqAssoc{}%Assoc
\end{Requisito}
%
\vfill
%
\begin{Requisito}
    \ReqNome{Verificar o custo de cada tipo de \unexpanded{\^o}nibus}%Nome
    \ReqLabel{ver-custo}%Label
    \ReqTipo{funcional}%Tipo
    \ReqDescr{Os administradores do sistema devem poder configurar quantas moedas ser\~ao dadas aos usu\'arios ap\'os estes realizarem avalia\c{c}\~oes do Servi\c{c}o de \^onibus. O administrador deve poder definir a recomepnsa ``por pergunta'' ou ``por conjunto de perguntas''.}%Desc.
    \ReqPrioridade{alta}%Prioridade
    \ReqStatus{aprovado}%Status
    \ReqEstabilidade{alta}%Estabilidade
    \ReqOrigem{interna}%Origem
    \ReqRationale{A modifica\c{c}\~ao do valor da recompensa por avali\c{c}\~ao deve estar dispon\'{\i}vel apenas para administradores do sistema, e deve ser de f\'acil realiza\c{c}\~ao.}%Rationale
    \ReqAssoc{}%Assoc
\end{Requisito}
%
\vfill
%
\begin{Requisito}
    \ReqNome{Comprar \unexpanded{\^o}nibus}%Nome
    \ReqLabel{compra-onibus}%Label
    \ReqTipo{funcional}%Tipo
    \ReqDescr{Permitir que o usuário compre novos ônibus para sua frota com suas moedas acumuladas.}%Desc.
    \ReqPrioridade{alta}%Prioridade
    \ReqStatus{aprovado}%Status
    \ReqEstabilidade{alta}%Estabilidade
    \ReqOrigem{interna}%Origem
    \ReqRationale{O usuário deve ter controle de sua frota.}%Rationale
    \ReqAssoc{}%Assoc
\end{Requisito}
%
\vfill
%
\begin{Requisito}
    \ReqNome{Definir custo dos \unexpanded{\^o}nibus}%Nome
    \ReqLabel{def-custo}%Label
    \ReqTipo{funcional}%Tipo
    \ReqDescr{Permitir aos administradores do site definir o preço de cada ônibus.}%Desc.
    \ReqPrioridade{alta}%Prioridade
    \ReqStatus{aprovado}%Status
    \ReqEstabilidade{alta}%Estabilidade
    \ReqOrigem{interna}%Origem
    \ReqRationale{Os custos podem ser ajustados para permitir maior ou menor facilidade aos usuáiros adquirirem determinados tipos de ônibus.}%Rationale
    \ReqAssoc{}%Assoc
\end{Requisito}
\vfill
\clearpage
\section{Requisitos Não Funcionais}\label{sec: RNF}

\StartReqNFunc
%\vfill
%\begin{Requisito}
%    \ReqTipo{nao funcional}%Tipo
%    \ReqNome{O sistema deverá exigir autenticação para acesso do usuário}%Nome
%%%%    \ReqLabel{Req}%Label
%    \ReqDescr{Para qualquer funcionalidade do sistema, este deve verificar se o usuário possui privilégios para acessar esta função.}%Desc
%    \ReqPrioridade{alta}%Prioridade
%    \ReqStatus{proposto}%Status
%    \ReqEstabilidade{media}%Estabilidade
%    \ReqOrigem{cliente}%Origem
%    \ReqRationale{Necessário para garantir que os dados do negócio do cliente estejam seguros no sistema}%Rationale
%    \ReqAssoc{Funcionalidades do sistema estruturadas}%Assoc
%\end{Requisito}

\vfill
\begin{Requisito}
    \ReqNome{Autentica\noexpand\c{c}\noexpand\~ao Obrigat\noexpand\'oria}%Nome
    \ReqTipo{nao funcional}%Tipo
%%%    \ReqLabel{Req}%Label
    \ReqDescr{Para qualquer funcionalidade do sistema, este deve verificar se o usuário está autenticado.}%Desc
    \ReqPrioridade{alta}%Prioridade
    \ReqStatus{aceito}%Status
    \ReqEstabilidade{alta}%Estabilidade
    \ReqOrigem{interna}%Origem
    \ReqRationale{Até este momento todos os serviços do projeto requerem autenticação dos usuários.}%Rationale
    \ReqAssoc{}%Assoc
\end{Requisito}

\vfill
\begin{Requisito}
    \ReqNome{Garantir acesso apenas a pessoas autorizadas}%Nome
    \ReqTipo{nao funcional}%Tipo
%%%    \ReqLabel{Req}%Label
    \ReqDescr{Para qualquer funcionalidade do sistema, este deve verificar se o usuário possui privilégios para acessar esta função.}%Desc
    \ReqPrioridade{alta}%Prioridade
    \ReqStatus{aceito}%Status
    \ReqEstabilidade{alta}%Estabilidade
    \ReqOrigem{interna}%Origem
    \ReqRationale{Necessário para garantir que nenhum dado ou função seja acessada indevidamente, além de garantir que toda ação dos usuários seja corretamente registrada.}%Rationale
    \ReqAssoc{}%Assoc
\end{Requisito}

\vfill
\begin{Requisito}
    \ReqNome{Senhas criptografadas na base}%Nome
    \ReqTipo{nao funcional}%Tipo
%%%    \ReqLabel{Req}%Label
    \ReqDescr{Senhas gravadas no banco de dados devem estar criptografadas, nunca em texto plano.}%Desc
    \ReqPrioridade{alta}%Prioridade
    \ReqStatus{aceito}%Status
    \ReqEstabilidade{alta}%Estabilidade
    \ReqOrigem{interna}%Origem
    \ReqRationale{Gravar senhas no banco de dados em texto plano é uma falha grave de segurança em qualquer sistema.}%Rationale
    \ReqAssoc{}%Assoc
\end{Requisito}

\vfill
\begin{Requisito}
    \ReqNome{Qualidade da senha}%Nome
    \ReqTipo{nao funcional}%Tipo
%%%    \ReqLabel{Req}%Label
    \ReqDescr{Todas as senhas definidas no sistema devem ter ao menos 6 dígitos, contendo ao menos um número, uma letra e um caracter não alfanumérico (.-\_\@\#! e outros).}%Desc
    \ReqPrioridade{alta}%Prioridade
    \ReqStatus{aceito}%Status
    \ReqEstabilidade{alta}%Estabilidade
    \ReqOrigem{interna}%Origem
    \ReqRationale{É fundamental que as senhas tenham um nível mínimo de segurança para impedir ataques de "força bruta".}%Rationale
    \ReqAssoc{}%Assoc
\end{Requisito}

\vfill
\begin{Requisito}
    \ReqNome{Qualidade das senhas administrativas}%Nome
    \ReqTipo{nao funcional}%Tipo
%%%    \ReqLabel{Req}%Label
    \ReqDescr{Todas as senhas administrativas definidas no sistema devem ter ao menos 10 dígitos, contendo ao menos dois números, duas letras e dois caracteres não alfanuméricos (.-\_\@\#! e outros).}%Desc
    \ReqPrioridade{alta}%Prioridade
    \ReqStatus{aceito}%Status
    \ReqEstabilidade{alta}%Estabilidade
    \ReqOrigem{interna}%Origem
    \ReqRationale{É fundamental que as senhas tenham um nível mínimo de segurança para impedir ataques de "força bruta", em especial as com acesso privilegiado.}%Rationale
    \ReqAssoc{}%Assoc
\end{Requisito}

\vfill
\begin{Requisito}
    \ReqNome{Realizar o log das a\noexpand\c{c}\noexpand\~oes dos usu\noexpand\'arios}%Nome
    \ReqTipo{nao funcional}%Tipo
%%%    \ReqLabel{Req}%Label
    \ReqDescr{Realizar o log (``tracking'') de todas as ações de todos os usuários, registrando-as na base de dados.}%Desc
    \ReqPrioridade{alta}%Prioridade
    \ReqStatus{aceito}%Status
    \ReqEstabilidade{alta}%Estabilidade
    \ReqOrigem{interna}%Origem
    \ReqRationale{Essas informações servirão para posterior análise e melhoria de usabilidade da aplicação.}%Rationale
    \ReqAssoc{}%Assoc
\end{Requisito}

\vfill
\begin{Requisito}
    \ReqNome{Realizar registro do timing das a\noexpand\c{c}\noexpand\~oes dos usu\noexpand\'arios}%Nome
    \ReqTipo{nao funcional}%Tipo
%%%    \ReqLabel{Req}%Label
    \ReqDescr{Registrar no log o quanto tempo cada usuário demorou para realizar cada ação, ou quanto dele ele ficou em cada página da aplicação.}%Desc
    \ReqPrioridade{alta}%Prioridade
    \ReqStatus{aceito}%Status
    \ReqEstabilidade{alta}%Estabilidade
    \ReqOrigem{interna}%Origem
    \ReqRationale{Essas informações servirão para posterior análise e melhoria de usabilidade da aplicação.}%Rationale
    \ReqAssoc{}%Assoc
\end{Requisito}

\vfill
\begin{Requisito}
    \ReqNome{N\noexpand\'umero m\noexpand\'aximo de avalia\noexpand\c{c}\noexpand\~oes espec\noexpand\'ificas}%Nome
    \ReqTipo{nao funcional}%Tipo
%%%    \ReqLabel{Req}%Label
    \ReqDescr{Apresentar no máximo 5 avaliações específicas aos usuários.}%Desc
    \ReqPrioridade{alta}%Prioridade
    \ReqStatus{aceito}%Status
    \ReqEstabilidade{alta}%Estabilidade
    \ReqOrigem{interna}%Origem
    \ReqRationale{Mais do que 5 questões faz com que o questionário fique muito extenso e demorado para ser respondido num dispositivo móvel, dentro de um ônibus em movimento, reduzindo o número de respostas.}%Rationale
    \ReqAssoc{}%Assoc
\end{Requisito}

\vfill
\begin{Requisito}
    \ReqNome{Reclama\noexpand\c{c}\noexpand\~oes ap\noexpand\'os salvar avalia\noexpand\c{c}\noexpand\~oes}%Nome
    \ReqTipo{nao funcional}%Tipo
%%%    \ReqLabel{Req}%Label
    \ReqDescr{As telas de reclamação devem aparecer após o usuário salvar suas avaliações.}%Desc
    \ReqPrioridade{alta}%Prioridade
    \ReqStatus{aceito}%Status
    \ReqEstabilidade{alta}%Estabilidade
    \ReqOrigem{interna}%Origem
    \ReqRationale{As avaliações, realizadas com ``\textit{slidebar}'' são muito rápidas. Por isso, permitir que os usuários realizem todas essas avaliações antes de abrir as caixas de "reclamação" para que eles digitem as reclamações é fundamental.}%Rationale
    \ReqAssoc{}%Assoc
\end{Requisito}

\vfill
\begin{Requisito}
    \ReqNome{Reclama\noexpand\c{c}\noexpand\~oes ap\noexpand\'os salvar avalia\noexpand\c{c}\noexpand\~oes}%Nome
    \ReqTipo{nao funcional}%Tipo
%%%    \ReqLabel{Req}%Label
    \ReqDescr{As telas de reclamação devem aparecer após o usuário salvar suas avaliações.}%Desc
    \ReqPrioridade{alta}%Prioridade
    \ReqStatus{aceito}%Status
    \ReqEstabilidade{alta}%Estabilidade
    \ReqOrigem{interna}%Origem
    \ReqRationale{As avaliações, realizadas com ``\textit{slidebar}'' são muito rápidas. Por isso, permitir que os usuários realizem todas essas avaliações antes de abrir as caixas de "reclamação" para que eles digitem as reclamações é fundamental.}%Rationale
    \ReqAssoc{}%Assoc
\end{Requisito}

\clearpage
\subsection{Tabela Resumo dos Requisitos Não Funcionais}\label{subsec:tabResReqNF}
   \begin{table}[H]
        \centering
        \caption{Requisitos não funcionais}
        \label{tab:reqNFunc}
        \PrintRequisitosNFunc
    \end{table}

\section{Avaliações}\label{sec:avalia}
\subsection{Avaliações Específicas}\label{subsec:avaliaspec}
Aos usuários serão facultadas as seguintes avaliações específicas:
\begin{description}
    \item[Lotação] \hfill \\
    Como o usuário avalia a lotação do ônibus, de muito lotado a vazio
    \item[Conforto Térmico] \hfill \\
    Como o usuário avalia o conforto térmico do ônibus, de muito quente a muito frio
    \item[Higiene] \hfill \\
    Como o usuário avalia a limpeza do ônibus, de muito sujo a muito limpo
    \item[Atendimento do Motorista] \hfill \\
    Como o usuário avalia o atendimento do motorista, de péssimo a ótimo
    \item[Atendimento do Cobrador] \hfill \\
    Como o usuário avalia o atendimento do cobrador, de péssimo a ótimo    
\end{description}

Estas são as avaliações que foram pensadas inicialmente, mas da forma como o projeto foi desenvolvido é possível adicionar, habilitar, desabilitar ou remover avaliações dinamicamente por meio do painel de administração do \gls{django}.