À minha família, que me deu todas as oportunidades para que eu chegasse a este momento e sempre esteve ao meu lado, me apoiando e confiando em mim. Obrigado Maria Júlia Rabatone Oliveira, Luiz Francisco Oliveira, Pedro Rabatone Oliveira e Paula Rabatone Oliveira.

À minha grande companheira Haydée Svab, que também me co-orientou neste trabalho, por todo seu amore e apoio incondicional, com elogios, abraços, broncas e orientações. Sou muito feliz por tê-la em minha vida, e que muitas outras experiências venham para somarmos e construirmos juntos.

À Daniela B. Silva e ao Pedro Markun, em nome da comunidade Transparência Hacker, que me abriu todo um mundo de colaboração e transformação de nossa sociedade, por meio do uso de tecnologia, mas sem tê-la como objetivo e única via.

Ao Leonardo Alexandre Ferreira Leite, Thiago Costa Paiva e Tássio Naia, em nome do PoliGNU, Grupo de Estudos de Software Livre da Poli-USP, que ajudei a fundar em 2009 e aonde pude aprender muito sobre compartilhamento, troca, colaboração, Software Livre, tecnologia, companheirismo e dedicação.

Às professores Cíntia Borges Margi, Maria Eugênia Boscov e ao professor Felipe Pait, por todo apoio dado ao PoliGNU e ao PoliGen, e por ajudarem a construir uma Escola melhor para todas e todos.

Ao Giuliano Salcas Olguin, Orientador Pedagógico da Escola Politénica, que de chefe se tornou um grande amigo, com quem tive muitas oportunidades de aprendizado e troca.

À Ângela Teresa Buscema, Assistente Técnico Acadêmico da Escola Politénica, uma verdadeira mãe de todos os politécnicos e politénicas.

Ao Guttember Nunes (Gutem), por todo apoio dado na reta final deste trabalho.

Ao Prof. Dr. João José Neto (JJ), por todas as horas de conversa, por todo conhecimento oferecido durante minha graduação, e por ser um grande exemplo.

E ao meu orientador, Prof. Dr. Reginaldo Arakaki, por todo aprendizado oferecido e pela confiança em meu trabalho e potencial.