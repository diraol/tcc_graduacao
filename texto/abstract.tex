\lettrine{T}{his work} describes the implementation of the \trilhasp~application, that aims to improve the flow of information between users and service providers of the public transport system, so it can positively change the urban mobility of the city. It will allow the users to evaluate the service and also make available to the users crowdsourced information about other users so each user can take better decisions on how and when to use the system.%
%
~\trilhasp~users will be able to evaluate the transport service based on qualitative criteria such as ``How crowded was the bus'' or ``Were you well attended by the bus driver and the collector?''. This qualitative criteria was chosen because it was identified that they influence the users choice of transport mode and they are not as easy to measure with technology and automation as speed of bus frequency. This evaluations can be positive or negative on a continuous scale and they will allow the creation of indexes per bus vehicle so they can be used to improve the service and also be a criteria of service remuneration.%
%
~The ``map'' module will show a map with all currently connected users. This will allow them to take a more precise decision on how to go to take the bus. If the bus stop are too crowded the user can just postpone the trip and do other things before going to the bus stop, like going to a gym or finish the current work. This could influence the demand in a way that makes the system more rational.%
%
~There is also a ``game'' module, that aims both to engage the user on the app and also educate the user about the costs and needs of a transport system.%
%
~This project was developed with free software and it's own source code is available as a Free Software (as in freedom).

\vspace{\onelineskip}

\vfill

\noindent

Software livre, Transporte Urbano, Sistema de Posicionamento Global
\textbf{Keywords}: Free software, Urban Transportation, Global Positioning System