\section{Objetivos}\label{sec:objetivos}

\subsection{Geral}\label{subsec:objGerais}
O objetivo geral do projeto é desenvolver um software que corrobore para a melhoria da mobilidade nas cidades pelo incremento da qualidade e da eficiência do sistema público de transporte urbano, mais especificamente rodoviário (ônibus). Será desenvolvido um produto considerando as condições de contorno da Região Metropolitana de São Paulo, que é o maior conglomerado urbano do Brasil com aproximadamente 20 milhões de habitantes. 

\subsection{Específicos}\label{subsec:objEspec}
São objetivos específicos do projeto: gerar dados e informações de valor para os atores envolvidos no sistema, a saber, operadores dos sistemas de transporte público, usuários e SPTrans; possibilitar a troca de informações triangular entre os três atores do sistema; corroborar para a eficiência do sistema pois com mais informação haverá mais controle de operação; promover transparência já que as informações geradas serão em sua maioria dados públicos sujeitos à Lei 12.527/2011; e estimular a conscientização para o usuário acerca do funcionamento do sistema público de transporte por ônibus.
	
\section{Escopo e critério de sucesso}\label{sec:Escopo}
	Ao final deste projeto o software contará com uma série de módulos, a saber:
	\begin{itemize}%[\itshape a\upshape)]
		\item módulo de avaliação do sistema de transporte;
		\item módulo de “gamificação” que terá as funções de oferecer um jogo que atraia e incentive a participação dos usuários e também sirva para um propósito pedagógico de conscientizar os usuários sobre a gestão de um sistema de transporte;
		\item módulo de “gps social”, que levantará informações de demanda do sistema de transporte e as fornecerá aos usuários em tempo real por meio de um mapa; e
		\item módulo de informações do sistema, que irá coletar as informações oferecidas pelo prestador de serviços (ex.: SPTrans) e disponibilizar à população – informações como tempo de espera de um ônibus, linhas que passam por um determinado ponto.
	\end{itemize}.

\section{Não-escopo}\label{sec:NãoEscopo}
	Não está no escopo de implementação deste projeto um ou mais módulos que permitam a integração do sistema com prestadores de serviço não ligados ao sistema de transporte, apesar de a arquitetura do sistema ser modelada para aceitar tais recursos num momento futuro.

%\section{Stakeholders}\label{sec:stakeholders}
%	Na matriz abaixo encontram-se os \emph{stakeholders} do projeto e seus respectivos poder e influência no projeto:
%
%	\bigskip
%	\begin{table}[H]
%	\centering
%	\caption{Stakeholders, relacionamento, interesse e poder}
%    \begin{tabular}{lccc}
%      \toprule
%      \headerCell{Parte interessada} &
%      \begin{minipage}{0.2\textwidth}
%        \espacoVert
%        \headerCell{Relacionamento com o Projeto}
%        \espacoVert
%      \end{minipage} &
%      \headerCell{Interesse (A/B)} &
%      \headerCell{Poder (A/B)}\\
%        
%      \midrule
%         
%			\textbf{Diego Rabatone Oliveira} &
%			Respons\'avel &
%			A &
%			A\\
%			
%			\textbf{Reginaldo Arakaki} &
%			Orientador &
%			A &
%			A\\
%
%			\textbf{Haydée Svab} &
%			Co-orientadora &
%			A &
%			A\\
%
%			\textbf{Prefeitura de São Paulo} &
%			Interessada &
%			B &
%			B\\
%			
%			\textbf{SPTrans} &
%			Interessada &
%			B &
%			B\\
%
%			\bottomrule
%	\end{tabular}
%	\end{table}

%\bigskip

\section{Restrições}\label{sec:restrições}
Todo desenvolvimento será realizado utilizando licenças livres e possiveis integrações com softwares terceiros devem levar este fato em consideração, ou seja, será preciso que haja compatibilidade de licenças.

