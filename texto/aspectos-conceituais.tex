\chapter{Aspectos Conceituais}\label{chp:aspectosConceituais}
Os principais aspectos conceituais abordados neste projeto são:
\begin{description}
    \item[\gls{api}] \cite{apifoldoc} \hfill \\
        \gls{api} é um conjunto de rotinas e padrões estabelecidos por um \textit{software} para a utilização das suas funcionalidades por aplicativos que não pretendem considerar os detalhes da implementação do \textit{software}, mas apenas usar seus serviços.\\
        Este conceito será aplicado no \textit{backend} do projeto como interface de comunicação do \textit{app mobile} com o \textit{backend} e também como forma de disponibilizar os dados coletados pelo projeto para, por exemplo, a produção de relatórios para a \sptrans, ou disponibilizar os dados publicamente para que seja possível produzir relatórios e análises independentes.%
%
	\item[\gls{rest}] \cite{Fielding2000} \hfill \\
	    \gls{rest} é um estilo arquitetural de \textit{software} que consiste numa série planejada de restrições aplicadas a componentes, conectores e elementos de dados em sistemas de hipermídia distribuídos. \\
	    Este conceito será aplicado na implementação da \gls{api} do projeto.%
%
	\item[BigData] \hfill \\
	    Como a aplicação irá lidar com uma quantidade de usuários simultâneos da ordem de centenas de milhares, caso ganhe escala, será necessária uma infraestrutura que possa lidar com essa quantidade de informações, tanto do ponto de vista de armazenamento como do ponto de vista de transações.\\
	    Neste sentido, o maior desafio neste projeto deverá ser na aquisição da localização dos usuários, registro da mesma e, principalmente, na disponibilização dessa informação atualizada aos usuários no \textbf{módulo ``gps social''}.%
%
	\item[Gamificação] \hfill \\
	    Segundo \apudonline[p.32]{zichermann2010game}{Mastrocola2012}:
	    \begin{citacao}
	    \textit{Gamification} é fundamentalmente reescrever as regras de jogos para \textit{design} de produtos e marketing. Da rede social de geo localização \textit{FourSquare} até o \textit{social game Farmville}, e da Nike até a Marinha americana, elementos de \textit{games} como pontos, troféus, níveis, recompensas e rankings estão sendo usados em número cada vez maior.
	    \end{citacao}
	    Este conceito será aplicado na criação de um jogo que se integre ao módulo de avaliação para incentivar os usuários a avaliarem o sistema de trânsito e também com o objetivo de educar os usuários quanto às necessidades e custos na manutenção de uma frota de ônibus de passageiros.%
%
	\item[GPS Social] \hfill \\
	    O sentido de GPS Social aqui utilizado diz respeito à simultânea coleta e disponibilização de grande volume de dados de geolocalização, advindos de uma grande base de usuários, de forma que os dados sejam utilizados ``em tempo real'' pelas pessoas influenciando seus comportamentos de deslocamento. A primeira, senão a única, aplicação que foi encontrada e que faz uso intensivo desse conceito é o aplicativo \textbf{Waze}\footnote{``Waze é o maior aplicativo de navegação e trânsito do mundo baseado em uma comunidade. Junte-se aos outros motoristas em sua área e compartilhe informações de trânsito das vias em tempo real, fazendo todos economizarem tempo e combustível em seus deslocamentos diários.'' - \url{http://waze.com} - Acesso em: 01/02/2014}, que recentemente começou a adotar o termo ``Aplicativo de navegação e tráfego baseado em comunidade'' (\textit{community-based traffic and navigation app}).\\
	    Um ponto importante a ser destacado neste estilo de aplicação é que, diferente do conceito tradicional das redes sociais, nestes aplicativos, os dados de geolocalização disponibilizados não dependem de relacionamentos entre os usuários, mas unicamente da localização dos mesmos. \\
	    Uma outra leitura possível para o termo, mas que não se encaixa na aplicação proposta neste trabalho, é mais ligada ao conceito tradicional de rede social e serve para conectar usuários que estão numa mesma localidade. Exemplos de aplicativos que seguem este conceito são:%
	    %
	    \begin{itemize}
        \item Tinder\footnote{``O Tinder é um jeito divertido de se conectar com pessoas novas e interessantes próximas de você. Passe as fotos para a direita para curtir ou para a esquerda para passar. Se alguém curtir você de volta, vocês combinam! Converse com sua combinação ou tire uma foto para compartilhar um Momento com todas as suas combinações de uma vez. É uma nova maneira de se expressar e compartilhar com seus amigos.'' {\url{https://play.google.com/store/apps/details?id=com.tinder&hl=pt_BR}} - Acesso em: 01/11/2014}
%
        \item Swipe\footnote{``Anonimamente encontrara outras pessoas próximas que gostam de você. Se você também gosta delas, fazemos a ligação e então vocês podem conversar gratuitamente! Conheça novas pessoas sem ser propagado por pessoas que você não está interessado em conhecer ou conversar com elas. Este é o maior e mais livre aplicativo para conhecer novas pessoas e fazer novos amigos para paquerar, namorar, bater papo, amizade e diversão!'' {\url{https://play.google.com/store/apps/details?id=com.chirpme.swipe&hl=pt_BR}} - Acesso em: 01/11/2014}
%
        \item Brenda\footnote{``Brenda é o mais popular aplicativo de encontros para mulheres lésbicas, bissexuais ou curiosas. É rápido, fácil de usar, sem complicações e amigável às usuárias. A maioria dos recursos é totalmente gratuita, incluindo conversas ilimitadas.'' {\url{https://play.google.com/store/apps/details?id=com.benderapp.brenda&hl=pt_BR}} - Acesso em: 01/11/2014}
%
        \item Grindr\footnote{``O aplicativo de rede social exclusivamente para gays, bissexuais e curiosos mais famoso do mundo - agora mais sexy e mais rápido do que nunca. Com mais de 7 milhões de homens em 192 países, o Grindr encontra rapazes por perto. Você pode conversar e conhecer - em qualquer lugar do mundo. Encontre o rapaz perfeito agora mesmo.'' {\url{https://play.google.com/store/apps/details?id=com.grindrapp.android&hl=pt_BR}} - Acesso em: 01/11/2014}
%
	    \end{itemize}
	    Assim, o conceito de ``gps social'' será utilizado para a realização de estimativas de lotação dos pontos de ônibus e também poderá ser utilizado, futuramente, para estimar a quantidade de passageiros dentro do ônibus.%
%
	\item[Percepção do usuário/consumidor]\cite{Lai1995,Almeida2011,Almeida2007,andrade2008constructos} \hfill \\
	    Como um dos pontos fundamentais propostos para a aplicação é permitir ao usuário avaliar o serviço de transporte, em especial baseado em critérios qualitativos que não são facilmente mensuráveis. Definiu-se então que será realizada uma avaliação geral e cinco avaliações específicas, considerados critérios importantes na tomada de decisão de um usuário de ônibus, a saber:
	    \begin{itemize}
	    \item Lotação
	    \item Conforto Térmico
	    \item Higiene/Limpeza
	    \item Atendimento do Motorista
	    \item Atendimento do Cobrador
	    \end{itemize}
	    Optou-se por cinco questões específicas por ser um número não muito grande com o objetivo de não desincentivar o usuário a responder. Este número, porém, pode variar, mas recomenda-se que não passe de 15 \cite{teller2013}. As cinco questões serão colocadas numa única página, de maneira que, de início, o usuário já tenha conhecimento da quantidade de respostas que terá que dar.\\
	    Além disso, as avaliações serão realizadas numa escala contínua. Dessa forma será possível extrair um resultado que permitirá a elaboração de estatísticas descritivas para além de simples frequências. Caso se tivesse optado por utilizar uma escala Likert seria possível apenas realizar análises das frequências das respostas, restringindo as análises possíveis e dificultando a geração de índices e indicadores.\cite{favero2009}%
%
    \item[Desenvolvimento Móvel Híbrido] \hfill \\
        Ou também conhecido como \textit{Mobile Hybrid}, é definido por \citeonline[p.6]{Leadership2012} como:
        \begin{citacao}
            \textit{The hybrid approach combines native development with web technology. Using this approach, developers write significant portions of their application in cross-platform web technologies, while maintaining direct access to native APIs when required.}\\
        \textit{The native portion of the application uses the operating system APIs to create an embedded HTML rendering engine that serves as a bridge between the browser and the device APIs. This bridge enables the hybrid app to take full advantage of all the features that modern devices have to offer.}
        \end{citacao}
        Este será o conceito utilizado para o desenvolvimento do aplicativo mobile do projeto. As funcionalidades nativas que serão utilizadas serão:%
            \begin{enumerate*}[label=\itshape\alph*\upshape)]
                \item geolocalização (gps);
                \item câmera; e
                \item execução da aplicação em segundo plano.
            \end{enumerate*}%
%
    \item[\gls{mvt}] \hfill \\
        É a \gls{dp} utilizada pelo \textit{framework} \gls{django}. Segundo \citeonline{djangobook}, o \textit{framework} segue boa parte do padrão \gls{mvc}, proposto por \citeonline{gamma95}.\\
        O padrão \gls{mvc}, no \gls{django}, é implementado da seguinte maneira:
        \begin{itemize}
	        \item[M](``\textit{Model}''): camada responsável pelo acesso aos dados. É gerenciada pelo módulo de base de dados do \gls{django};
	        \item[V](``\textit{View}''): camada responsável pela seleção de quais dados serão apresentados e como eles serão apresentados. É gerenciada pelos módulos \textit{views} e \textit{template}; e
	        \item[C](``\textit{Controller}''): camada que direciona uma requisição a uma \textit{View}, baseado nas regras definidas no arquivo de configuração de endereços (\textit{urls}), chamando a função apropriada para tratar aquela requisição.
        \end{itemize}
        Como a camada ``C'' é gerenciada pelo próprio \textit{framework} e a maior parte das ações ocorrem nos módulos ``\textit{model}'', ``\textit{view}'' e ``\textit{template}'', defini-se que o \gls{django} como um \textit{framework} \gls{mvt}, no qual:
        \begin{itemize}
	        \item[M](``\textit{Model}''): camada de acesso aos dados. Esta camada contém tudo sobre os dados do projeto, como:
            \begin{enumerate*}[label=\itshape\alph*\upshape)]
                \item Como acessar os dados;
                \item Como validar os dados;
                \item Qual o comportamento do dado; e
                \item O relacionamento entre os dados.
            \end{enumerate*};
            \item[V](``\textit{View}''): camada de lógica de negócio. Ela contém toda lógica de acesso à camada de dados e direciona para o template adequado. Pode-se pensá-la como a ponte entre a camada de dados e a de templates; e
            \item[T](``\textit{Template}''): camada de apresentação. Esta camada contém decisões relativas à apresentação como, por exemplo, a maneira pela qual algo deve ser mostrado em uma página web ou outro tipo de documento.
        \end{itemize}
\end{description}