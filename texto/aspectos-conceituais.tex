\chapter{Aspectos Conceituais}\label{chp:aspectosConceituais}

Os aspectos conceituais a serem tratados neste projeto são:
\begin{description}
	\item[\gls{rest}] \cite{Fielding2000} \hfill \\
	    \gls{rest} é um estilo arquitetural de software que consiste numa série planejada de restrições apliacadas a componentes, conectores e elementos de dados em sistemas de hipermedia distribuídos. \\
	    Este conceito será aplicado, inicialmente, na aquisição dos dados da \sptrans.
	\item[Big Data] \hfill \\
	    Como a aplicação irá lidar com uma quantidade de usuários simultâneos da ordem de centenas de milhares, será necessária uma infraestrutura de base de dados que possa lidar com essa quantidade de informações. Além disso, dos conceitos envoltos no tema \bigdata~serão necessárias realizar rápidas e intensivas consultas a esta base.
	\item[Gamificação] \cite{vieira2012exploratory,Mastrocola2012} \hfill \\
	    Serão utilizados conceitos de gamificação para tentar aumentar o engajamento do usuário na plataforma e ainda adicionar um caráter educativo.
	\item[GPS Social]\cite{Miller2013,Gal-Tzur2014a,Filippi2013,Gal-Tzur2014,Nunes2014} \hfill \\
	    O conceito de \gpssocial~ganhou projeção com o conhecido aplicativo de GPS \textbf{Waze}. Sua principal ideia é que as informações de geolocalização de todos os usuários da plataforma estão disponíveis instantâneamente aos outros usuários de forma que a origem das informações oferecidas pela plataforma não são advindas de coletas feitas pela própria plataforma, mas sim pelos usuários da mesma, de uma maneira \textit{distribuída}. \\
	    Este conceito será utilizado para a realização de estimativas de lotação dos ônibus e dos pontos de ônibus, ajudando os usuários na tomada de decisão sobre quando e aonde tomar um ônibus.
	\item[Percepção do usuário/consumidor]\cite{Lai1995,Almeida2011,Almeida2007,andrade2008constructos} \hfill \\
	    Como um dos pontos fundamentais propostos para a aplicação é permitir ao usuário avaliar o serviço de transporte, em especial baseado em critérios mais subjetivos e não facilmente mensuráveis por meio de instrumentação eletrônica, será necessário estudar qual a melhor forma de se realizar tal avaliação.
%	\item[\gls{its}]\cite{silva2000sistemas,williams2008intelligent} \hfill \\
%	\item[Medidas de Desempenho no Transporte]\cite{Cellos2012}
\end{description}

%No que tange à \textbf{API Rest}, será realizada uma rápida revisão bibliográfica com a finalidade de definir as melhores técnicas referentes tanto à utilização de API REST servidas por terceiros (no caso a SPTrans) quanto à implementação de uma API própria para servir, principalmente à SPTrans, os dados de avaliação inseridos pelos usuários. Com relação à \textbf{Big Data}, a revisão bibliográfica será feita no sentido de entender como capturar e armazenar os dados gerados pelos usuários, além de buscar os melhores softwares livres para esta implementação. Relativo à \textbf{Gamificação}, a revisão versará sobre o uso de gamificação ligada à área de Transporte Público. Sobre \textbf{GPS Social}, a revisão versará sobre o uso de tecnologias sociais e de massa na geração de informações e influência nos sistemas públicos de transporte. Com relação à \textbf{Percepção do usuário/consumidor}, a revisão será feita no sentido de buscar a melhor forma de permitir que usuários que avaliem serviços, levantando questões sobre como formular as perguntas ou sobre a utilização de escala \textit{Likert} ou escala contínua para as respostas. Este último item se conecta diretamente a \textbf{Medidas de desempenho no transporte}, que tem por finalidade entender quais são os principais critérios a serem avaliados num sistema de transporte. Por fim, as leituras sobre \textbf{ITS} objetivam entender como integrar todas as informações coletadas de maneira dinânima e em tempo real com o próprio sistema de transporte, além de levantar quais tecnologias já existem atualmente.
%
%\section{API-REST}\label{sec:apiRest}
%\section{Big Data}\label{sec:bigData}
%\section{Gamificação}\label{sec:gamificação}
%\section{GPS Social}\label{sec:gpsSocial}
%\section{Percepção do Usuário}\label{sec:percepçãoUsuário}
%\section{Intelligent Transport Systems}\label{sec:its}
\gls{its} significa .... \gls{its}
%\section{Medidas de Desempenho no Transporte}\label{sec:medidasDeDesempenho}
