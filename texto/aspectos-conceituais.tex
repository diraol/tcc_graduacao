\chapter{Aspectos Conceituais}\label{chp:aspectosConceituais}
Os principais aspectos conceituais abordados neste projeto são:
\begin{description}
    \item[\gls{api}] \cite{apifoldoc} \hfill \\
        \gls{api} é um conjunto de rotinas e padrões estabelecidos por um software para a utilização das suas funcionalidades por aplicativos que não pretendem envolver-se em detalhes da implementação do software, mas apenas usar seus serviços.\\
        Este conceito será aplicado no \textit{backend} do projeto como interface de comunicação do \textit{app mobile} com o \textit{backend} e também como forma de disponibilizar os dados coletados pelo projeto para, por exemplo, expor relatórios à SPTrans, ou disponibilizar os dados publicamente para que seja possível produzir relatórios e análises e análises independentes.%
%
	\item[\gls{rest}] \cite{Fielding2000} \hfill \\
	    \gls{rest} é um estilo arquitetural de software que consiste numa série planejada de restrições apliacadas a componentes, conectores e elementos de dados em sistemas de hipermedia distribuídos. \\
	    Este conceito será aplicado na implementação da \gls{api} do projeto.%
%
	\item[BigData] \hfill \\
	    Como a aplicação irá lidar com uma quantidade de usuários simultâneos da ordem de centenas de milhares, caso ganhe escala, será necessária uma infraestrutura que possa lidar com essa quantidade de informações, tanto do ponto de vista de armazenamento como do ponto de vista de transações.\\
	    Neste sentido, o maior desafio neste projeto deverá ser na aquisição da localização dos usuários, registro da mesma e, principalmente, na disponibilização dessa informação atualizada aos usuários no \textbf{módulo ``gps social''}.%
%
	\item[Gamificação] \hfill \\
	    Segundo \apudonline{zichermann2010game}{Mastrocola2012}:
	    \begin{citacao}
	    \textit{Gamification} é fundamentalmente reescrever as regras de jogos para \textit{design} de produtos e marketing. Da rede social de geo localização \textit{FourSquare} até o \textit{social game Farmville}, e da Nike até a Marinha americana, elementos de \textit{games} como pontos, troféus, níveis, recompensas e rankings estão sendo usados em número cada vez maior.
	    \end{citacao}
	    Este conceito será aplicado na tentativa de criação de um game que integre-se ao módulo de avaliação para incentivar os usuários a avaliarem o sistema de trânsito e também com o objetivo de educar os usuários quanto às necessidades e custos na manutenção de uma frota de transporte.%
%
	\item[GPS Social] \hfill \\
	    O sentido de GPS Social aqui utilizado diz respeito á simultânea coleta e disponibilização de grande volume de dados de geolocalização, advindos de uma grande base de usuários, de forma que os dados sejam utilizados ``em tempo real'' pelos usuários influenciando seus comportamentos de deslocamento. A primeira, senão a única, aplicação que foi encontrada e que faz uso intensivo desse conceito é o aplicativo \textbf{Waze}\footnote{\textit{Waze is the world's largest community-based traffic and navigation app. Join other drivers in your area who share real-time traffic and road info, saving everyone time and gas money on their daily commute.} - \url{http://waze.com}}, que recentemente começou a adotar o termo ``community-based traffic and navigation app''.\\
	    Um ponto importante a ser destacado neste estilo de aplicação é que, diferente do conceito tradicional das redes sociais, nestes aplicativos os dados de geolocalização disponibilizados não dependem de relacionamentos entre os usuários, mas unicamente da localização dos mesmos. \\
	    Uma outra leitura possível para o termo, mas que não se encaixa na aplicação proposta neste trabalho, é mais ligada ao conceito tradicional de rede social e serve para conectar usuários que estão numa mesma localidade. Exemplos de aplicativos que seguem este conceito são:
	    \begin{itemize}
	    \item Tinder\footnote{\textit{``Tinder is the fun way to connect with new and interesting people around you. Swipe right to like or left to pass. If someone likes you back, it’s a match! Chat with a match or snap a photo to share a Moment with all of your matches at once. Moment is a new way to express yourself and share with friends.''}\url{https://play.google.com/store/apps/details?id=com.tinder&hl=pt_BR}}
	    \item Swipe\footnote{\textit{``Chat instantly with people near you right from the palm of your hand. Unlock special rewards just for using the app. Play the newest version and increase your opportunity for more real-time interactions with real people!''}\url{https://play.google.com/store/apps/details?id=com.chirpme.swipe&hl=pt_BR}}
	    \item Brenda\footnote{\textit{``Brenda is the most popular dating app for lesbian, bi or curious women. It's fast, easy to use, no hassle and user friendly. Most features are completely free, including unlimited chat''}\url{https://play.google.com/store/apps/details?id=com.benderapp.brenda&hl=pt_BR}}
	    \item Grindr\footnote{\textit{``Grindr is a simple app that uses your mobile device’s location-based services to show you the guys closest to you who are also on Grindr. How much of your info they see is entirely your call.''}\url{https://play.google.com/store/apps/details?id=com.grindrapp.android&hl=pt_BR}}
	    \end{itemize}
	    Este conceito será utilizado para a realização de estimativas de lotação dos pontos de ônibus e também pode ser utilizado futuramente para estimar a quantidade de usuários dentro do ônibus.%
%
\clearpage
	\item[Percepção do usuário/consumidor]\cite{Lai1995,Almeida2011,Almeida2007,andrade2008constructos} \hfill \\
	    Como um dos pontos fundamentais propostos para a aplicação é permitir ao usuário avaliar o serviço de transporte, em especial baseado em critérios qualitativos que não são facilmente mensuráveis por instrumentação eletrônica. Definiu-se então que será realizada uma avaliação geral e 5 avaliações específicas, consideradas critérios importantes na tomada de decisão de um usuário de ônibus, e estas são:
	    \begin{itemize}
	    \item Lotação
	    \item Conforto Térmico
	    \item Higiene/Limpeza
	    \item Atendimento do Motorista
	    \item Atendimento do Cobrador
	    \end{itemize}
	    Optou-se por cinco questões específicas por ser um número não muito grande que não desincentivaria o usuário a responder. As cinco questões serão colocadas numa única página, de maneira que o usuário já tenha conhecimento de início a quantidade de respostas que terá que dar.\\
	    Além disso, as avaliações serão realizadas numa escala contínua. Dessa forma será possível extrair um resultado que permitirá a elaboração de estatísticas descritivas para além de simples frequências. Caso tivesse-se optado por utilizar uma escala Likert de 5 pontos seria possível apenas realizar análises das frequências das respostas, permitindo assim análises menos elaboradas e dificultando a geração de índices e indicadores.\cite{favero2009}%
%
    \item[Desenvolvimento Móvel Híbrido] \hfill \\
        Ou também conhecido como \textit{Mobile Hybrid}, é definido por \citeonline[p.6]{Leadership2012} como:
        \begin{citacao}
            \textit{The hybrid approach combines native development with web technology. Using this approach, developers write significant portions of their application in cross-platform web technologies, while maintaining direct access to native APIs when required.}\\
        \textit{The native portion of the application uses the operating system APIs to create an embedded HTML rendering engine that serves as a bridge between the browser and the device APIs. This bridge enables the hybrid app to take full advantage of all the features that modern devices have to offer.}
        \end{citacao}
        Este conceito será o utilizado para o desenvolvimento do aplicativo mobile do projeto. As funcionalidades nativas que serã utilizadas serão a de geolocalização (gps), o acesso à câmera do dispositivo e manter a aplicação rodando em \textit{background}.%
\clearpage
    \item[\gls{mvt}] \hfill \\
        É uma \textit{design pattern} utilizada pelo \textit{framework} \gls{django}. Segundo \citeonline{djangobook}, o \textit{framework} segue boa parte do que é prposto pelo padrão \gls{mvc} proposto por \citeonline{gamma95}, de maneira que ele poderia considerar-se que ele segue o mesmo.\\
        O padrão \gls{mvc}, no \gls{django}, é implementado da seguinte maneira:
        \begin{itemize}
	        \item[M](``\textit{Model}''): camada responsável pelo acesso aos dados, é gerenciada pelo módulo de base de dados do \gls{django}
	        \item[V](``\textit{View}''): camada responsável pela seleção de quais dados serão apresentados e como eles serão apresentados, é gerenciada pelos módulos \textit{views} e \textit{template}.
	        \item[C](``\textit{Controller}''): camada que delega a uma \textit{view} dependendo da entrada do usuário, é gerenciada pelo framework seguindo as regras expostas no \textit{URLConf} do projeto (regras de endereçamento das urls) e chamando a função python apropriada para aquela URL.
        \end{itemize}
        Como a camada ``C'' é gerenciada pelo próprio framework e a maior parte das ações ocorrem nos módulos ``\textit{model}'', ``\textit{view}'' e ``\textit{template}'', definiu-se que o \gls{django} seria referenciado como um \textit{framework} \gls{mvt}, no qual:
        \begin{itemize}
	        \item[M](``\textit{Model}''): acamada de acesso aos dados. Esta camada contém tudo sobre os dados do projeto, como:
            \begin{enumerate*}[label=\itshape\alph*\upshape)]
                \item Como acessar os dados;
                \item Como validar os dados;
                \item Qual o comportamento do dado; e
                \item o relacionamento entre os dados.
            \end{enumerate*}
	        \item[T](``\textit{Template}''): a camada de apresentação. Esta camada contém decisões relativas à apresentação como, por exemplo, a maneira na qual algo deve ser mostrado em uma página web ou outro tipo de documento;
            \item[V](``\textit{View}''): a camada de lógica de negócio. Ela contém toda lógica de acesso à camada de dados e direciona para o template adequado. Pode-se pensá-la como a ponte entre a camada de dados e a de templates.
        \end{itemize}        

\end{description}