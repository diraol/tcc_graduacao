\chapter{Metodologia}\label{chp: metodologia}

O Planejamento do projeto e sua concepção seguirão partes do \textbf{PMBOK}, seguindo o planejamento da disciplina \textbf{PCS2511 - Gerência, Qualidade e Tecnologia de Software}, oferecido em paralelo à primeira disciplina de Trabalho de Formatura.

Já com relação a implementação e testes, o projeto será desenvolvido seguindo o conceito de \gls{tdd}. Destaca-se que será fundamental a manutenção de comentários descritivos, principalmente nos testes, mas também no código fonte em si. Dessa maneira, o código fonte servirá como documentação principal do projeto.

Vale ressaltar que o \textit{framework} escolhido (\gls{django}) já possui todo suporte necessário para o desenvolvimento guiado por testes. Outro ponto importante de se destacar é que o \gls{django} também incentiva o desenvolvimento modular - no caso voltado a \textit{apps}.

O desenvolvimento contará também com versionamento de código utilizando \textbf{git} e gestão de \textit{issues} e \textit{milestones} numa única plataforma integrada de gestão de projetos.

Por fim, é fundamental ressaltar que a \sptrans~já oferece diversos recursos aos usuários, como localização de ônibus em tempo real, e que essas informações também são fornecidas via API\footnote{\url{http://www.sptrans.com.br/desenvolvedores/APIOlhoVivo/Documentacao.aspx}}. Dessa maneira pretende-se utilizar o máximo que for possível esta API para fornecer serviços básicos aos usuários, sem que tais tecnologias precisem ser reimplementadas, reduzindo muito a necessidade de armazenamento e tratamento de dados do sistema de transporte dentro da própria plataforma.