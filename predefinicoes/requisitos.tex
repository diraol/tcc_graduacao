%%%%%%%%%%%%%%%%%%%%%%%%%%%%%%%%%%%%%%%%%%%%%%%%%%%%%%%%%%%%%%%%%%%%%%%%%%%%%%%
%%                                REQUISITOS                                 %%
%%%%%%%%%%%%%%%%%%%%%%%%%%%%%%%%%%%%%%%%%%%%%%%%%%%%%%%%%%%%%%%%%%%%%%%%%%%%%%%
%%%%%%%%%%%%%%%%%%%%%%%%%%%%%%%%%%%%%%%
%%              USO                  %%
%%%%%%%%%%%%%%%%%%%%%%%%%%%%%%%%%%%%%%%
%
%\begin{Requisito}
%    \ReqTipo{<ARGUMENTO 1>}
%    \ReqNome{<ARGUMENTO 2>}
%    \ReqLabel{<LABEL>}
%    \ReqDescr{<ARGUMENTO 3>}
%    \ReqPrioridade{<ARGUMENTO 4>}
%    \ReqStatus{<ARGUMENTO 5>}
%    \ReqEstabilidade{<ARGUMENTO 6>}
%    \ReqOrigem{<ARGUMENTO 7>}
%    \ReqRationale{<ARGUMENTO 8>}
%    \ReqAssoc{<ARGUMENTO 9>}
%\end{Requisito}
%
%%%%%%%%%%%%%%%%%%%%%%%%%%%%%%%%%%%%%%%
%%     DESCRIÇÃO DOS ARGUMENTOS      %%
%%%%%%%%%%%%%%%%%%%%%%%%%%%%%%%%%%%%%%%
%% *** Nenhum argumento pode conter caracteres especiais, como o 'underscore' ***
% <LABEL>: Label a ser usada para referencias internas do documento
% <ARGUMENTO 1>: TIPO: ´Funcional´ (funcional) ou ´Nao Funcional´ (nao funcional)
% <ARGUMENTO 2>: Nome do requisito
% <ARGUMENTO 3>: Descrição do Requisito
% <ARGUMENTO 4>: Prioridade [alta, media, baixa]
% <ARGUMENTO 5>: Status [proposto, aprovado, incorporado, validado]
% <ARGUMENTO 6>: Estabilidade [alta, media, baixa]
% <ARGUMENTO 7>: Origem [usuario, interna, externa]
% <ARGUMENTO 8>: Rationale - texto extenso
% <ARGUMENTO 9>: Requisitos Associados - por enquanto tem que listar na mão
%%%%%%%%%%%%%%%%%%%%%%%%%%%%%%%%%%%%%%%%%%%%%%%%%%%%%%%%%%%%%%%%%%%%%%%%%%%%%%%
%
%
% CONTADORES DOS REQUISITOS
\newcounter{indiceFuncional}
\newcounter{indiceNaoFuncional}
\newcounter{clearPageCounter}
%
%
% Requisitos Funcionais
\newcommand{\StartReqFunc}{%
    % New job (that is, file)
    \newwrite\reqFunc%
    \immediate\openout\reqFunc=tabRequisitosFunc.aux%
    \setcounter{clearPageCounter}{0}%
}
%
%% Usamos unexpanded... mas eu não sei porque. No meu
%% computador não funciona se removo o unexpanded. Acho
%% que deve ter algo haver com o `\\` dentro da expressão.
%% (Aliás, isso é muito provável, já que o significado
%% de `\\` é diferente no texto e em um ambiente tabular;
%% sua expansão tem que ser postergada para o momento em
%% que a tabela está sendo impressa (`PrintVolunteers`).
\newcommand{\AppendRequisitoFunc}[3]{%
    \immediate\write\reqFunc{###1##rf-#2## & \expandafter#3 \unexpanded{\\}}%
}
%
\newcommand{\PrintRequisitosFunc}{%
	\begin{tabular}{cl}%
	\toprule
	    \headerCell{Identificação} & \headerCell{Requisito} \\%
	\midrule%
	\immediate\closeout\reqFunc\input{tabRequisitosFunc2.aux}%
	\bottomrule%
	\end{tabular}%
}
%
%
%
% Requisitos Não Funcionais
\newcommand{\StartReqNFunc}{%
    % New job (that is, file)
    \newwrite\reqNFunc%
    \immediate\openout\reqNFunc=tabRequisitosNFunc.aux%
    \setcounter{clearPageCounter}{0}%
}
%
%% Usamos unexpanded... mas eu não sei porque. No meu
%% computador não funciona se removo o unexpanded. Acho
%% que deve ter algo haver com o `\\` dentro da expressão.
%% (Aliás, isso é muito provável, já que o significado
%% de `\\` é diferente no texto e em um ambiente tabular;
%% sua expansão tem que ser postergada para o momento em
%% que a tabela está sendo impressa (`PrintVolunteers`).
\newcommand{\AppendRequisitoNFunc}[3]{%
    \immediate\write\reqNFunc{###1##rnf-#2##  & \expandafter#3 \unexpanded{\\}}%
}
%
\newcommand{\PrintRequisitosNFunc}{%
	\begin{tabular}{cl}%
	\toprule
	    \headerCell{Identificação} & \headerCell{Requisito} \\%
	\midrule%
	\immediate\closeout\reqNFunc\input{tabRequisitosNFunc2.aux}%
	\bottomrule%
	\end{tabular}%
}
%
%
%
%%%%%
\newenvironment{Requisito}{}{%
    \CriaRequisito%
}
%
%
% FUNÇÕES INTERNAS
\newcommand{\CmdReqTipo}{}%
\newcommand{\CmdReqLabel}{}%
\newcommand{\CmdReqNome}{}%
\newcommand{\CmdReqDescr}{}%
\newcommand{\CmdReqPrioridade}{}%
\newcommand{\CmdReqStatus}{}%
\newcommand{\CmdReqEstabilidade}{}%
\newcommand{\CmdReqOrigem}{}%
\newcommand{\CmdReqRationale}{}%
\newcommand{\CmdReqAssoc}{}%
%
%
%
%FUNÇÕES USER FRIENDLY
\newcommand{\ReqTipo}[1]{%
    \ifstrequal{#1}{funcional}{%
        \stepcounter{indiceFuncional}%
        %\ReqLabel{RF\theindiceFuncional}%
        \renewcommand{\CmdReqTipo}{\multicolumn{3}{|l|}{\textbf{No:} RF\theindiceFuncional~ (\hypertarget{rf-\CmdReqLabel}{rf-\CmdReqLabel})} & (X) Funcional & ( ) Não Funcional \\}
        \AppendRequisitoFunc{RF\theindiceFuncional}{\CmdReqLabel}{\CmdReqNome}
    }{%
        \stepcounter{indiceNaoFuncional}%
        \ReqLabel{RNF\theindiceNaoFuncional}%
        \renewcommand{\CmdReqTipo}{\multicolumn{3}{|l|}{\textbf{No:} RNF\theindiceNaoFuncional~ (\hypertarget{rnf-\CmdReqLabel}{rnf-\CmdReqLabel})} & ( ) Funcional & (X) Não Funcional \\}
        \AppendRequisitoNFunc{RNF\theindiceNaoFuncional}{\CmdReqLabel}{\CmdReqNome}
    }
}
\newcommand{\ReqLabel}[1]{\renewcommand{\CmdReqLabel}{#1}}%
\newcommand{\ReqNome}[1]{\renewcommand{\CmdReqNome}{#1}}%
\newcommand{\ReqDescr}[1]{\renewcommand{\CmdReqDescr}{#1}}%
\newcommand{\ReqPrioridade}[1]{%
    \ifstrequal{#1}{alta}{%
        \renewcommand{\CmdReqPrioridade}{(X) Alta & ( ) Média & ( ) Baixa & \\}
    }{%
        \ifstrequal{#1}{media}{%
            \renewcommand{\CmdReqPrioridade}{( ) Alta & (X) Média & ( ) Baixa & \\}
        }{%
            \renewcommand{\CmdReqPrioridade}{( ) Alta & ( ) Média & (X) Baixa & \\}
        }
    }
}
\newcommand{\ReqStatus}[1]{
    \ifstrequal{#1}{proposto}{%
        \renewcommand{\CmdReqStatus}{(X) Proposto & ( ) Aprovado & ( ) Incorporado & ( ) Válido \\}
    }{
        \ifstrequal{#1}{aprovado}{%
            \renewcommand{\CmdReqStatus}{( ) Proposto & (X) Aprovado & ( ) Incorporado & ( ) Válido \\}
        }{
            \ifstrequal{#1}{incorporado}{%
                \renewcommand{\CmdReqStatus}{( ) Proposto & ( ) Aprovado & (X) Incorporado & ( ) Válido \\}
            }{
                \renewcommand{\CmdReqStatus}{( ) Proposto & ( ) Aprovado & ( ) Incorporado & (X) Válido \\}
            }%
        }%
    }
}%
\newcommand{\ReqEstabilidade}[1]{
    \ifstrequal{#1}{alta}{%
        \renewcommand{\CmdReqEstabilidade}{(X) Alta & ( ) Média & ( ) Baixa & \\}
    }{%
        \ifstrequal{#1}{media}{%
            \renewcommand{\CmdReqEstabilidade}{( ) Alta & (X) Média & ( ) Baixa & \\}
        }{%
            \renewcommand{\CmdReqEstabilidade}{( ) Alta & ( ) Média & (X) Baixa & \\}
        }
    }
}
\newcommand{\ReqOrigem}[1]{
    \ifstrequal{#1}{usuario}{%
        \renewcommand{\CmdReqOrigem}{(X) Usuário & ( ) Interna & ( ) Externa & \\}%
    }{%
        \ifstrequal{#1}{interna}{%
            \renewcommand{\CmdReqOrigem}{( ) Usuário & (X) Interna & ( ) Externa & \\}%
        }{
            \renewcommand{\CmdReqOrigem}{( ) Usuário & ( ) Interna & (X) Externa & \\}%
        }%
    }%
}
\newcommand{\ReqRationale}[1]{\renewcommand{\CmdReqRationale}{#1}}%
\newcommand{\ReqAssoc}[1]{\renewcommand{\CmdReqAssoc}{#1}}%
%
%
%Função Requisito - utilizada indiretamente pelo ambiente Requisito
\newcommand{\CriaRequisito}{
    \begin{table}[!h]
        \begin{tabular}{|lllll|}%
            \hline%
                \CmdReqTipo%
            \hline%
                \multicolumn{5}{|l|}{%
                    \begin{minipage}{1\textwidth}%
                        \espacoVert%
                        \textbf{Requisito:} \CmdReqNome%#2%
                        \espacoVert%
                    \end{minipage}%
                } \\%
            \hline%
                \multicolumn{5}{|l|}{%
                    \begin{minipage}{1\textwidth}%
                        \espacoVert%
                        \textbf{Descrição:} \CmdReqDescr%#3%
                        \espacoVert%
                    \end{minipage}%
                } \\%
            \hline%
                \textbf{Prioridade:} & \CmdReqPrioridade%
            \hline%
                \textbf{Status:} & \CmdReqStatus%
            \hline%
                \textbf{Estabilidade:} & \CmdReqEstabilidade%
            \hline%
                \textbf{Origem:} & \CmdReqOrigem%
            \hline%
                \multicolumn{5}{|l|}{%
                    \begin{minipage}{1\textwidth}%
                        \espacoVert%
                        \textbf{Rationale:} \CmdReqRationale%#8%
                        \espacoVert%
                    \end{minipage}%
                } \\%
            \hline%
                \multicolumn{5}{|l|}{%
                    \begin{minipage}{1\textwidth}%
                        \espacoVert%
                        \textbf{Requisitos Associados:} \CmdReqAssoc%#9%
                        \espacoVert%
                     \end{minipage}%
                 } \\%
             \hline%
        \end{tabular}
     \end{table}     
    \stepcounter{clearPageCounter}
    \ifnum\value{clearPageCounter} = 3%
        \clearpage%
        \setcounter{clearPageCounter}{0}%
    \fi
}% FIM DO COMMAND CriaRequisito
%
