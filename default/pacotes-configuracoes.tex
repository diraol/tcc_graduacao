% ---
% Configurações do pacote backref
% Usado sem a opção hyperpageref de backref
\renewcommand{\backrefpagesname}{Citado na(s) página(s):~}
% Texto padrão antes do número das páginas
\renewcommand{\backref}{}
% Define os textos da citação
\renewcommand*{\backrefalt}[4]{
  \ifcase #1 %
    Nenhuma citação no texto.%
  \or
    Citado na página #2.%
  \else
    Citado #1 vezes nas páginas #2.%
  \fi}%
% -----------------------------------------------------------------

%Corrigindo erro de URLs muito longas não quebrando linha
 \tolerance 1414
 \hbadness 1414
 \emergencystretch 1.5em
 \hfuzz 0.3pt
 \widowpenalty=10000
 \vfuzz \hfuzz
 \raggedbottom
% -----------------------------------------------------------------
 
%% Configurações do pacote listings
\lstdefinestyle{customc}{
  belowcaptionskip=1\baselineskip,
  breaklines=true,
  frame=L,
  %xleftmargin=\parindent,
  xleftmargin=0pt,
  language=Python,
  showstringspaces=false,
  basicstyle=\footnotesize\ttfamily,
  keywordstyle=\bfseries\color{green!40!black},
  commentstyle=\itshape\color{purple!40!black},
  identifierstyle=\color{blue},
  stringstyle=\color{orange},
}
\lstset{
    escapechar=@,
    style=customc,
    inputencoding=utf8,
    extendedchars=true,
    literate={á}{{\'a}}1 {à}{{\`a}}1 {ã}{{\~a}}1 {é}{{\'e}}1 {ê}{{\^e}}1 {ë}{{\"e}}1 {í}{{\'i}}1 {ç}{{\c{c}}}1 {Ç}{{\c{C}}}1 {õ}{{\~o}}1 {ó}{{\'o}}1 {ô}{{\^o}}1 {ú}{{\'u}}1 {_}{{\_}}1,
}

% ---
% Formatação de código-fonte
% ---
% Altera o nome padrão do rótulo usado no comando \autoref{}
\renewcommand{\lstlistingname}{Código}

% Altera o rótulo a ser usando no elemento pré-textual "Lista de código"
\renewcommand{\lstlistlistingname}{Lista de códigos}

% Configura a ``Lista de Códigos'' conforme as regras da ABNT (para abnTeX2)
\begingroup\makeatletter
\let\newcounter\@gobble\let\setcounter\@gobbletwo
  \globaldefs\@ne \let\c@loldepth\@ne
  \newlistof{listings}{lol}{\lstlistlistingname}
  \newlistentry{lstlisting}{lol}{0}
\endgroup

\lstset{numberbychapter=false}
%\counterwithout{loldepth}{chapter}

\renewcommand{\cftlstlistingaftersnum}{\hfill--\hfill}

\let\oldlstlistoflistings\lstlistoflistings
\renewcommand{\lstlistoflistings}{%
   \begingroup%
   \let\oldnumberline\numberline%
   \renewcommand{\numberline}{\lstlistingname\space\oldnumberline}%
   \oldlstlistoflistings%
   \endgroup}
   
\lstdefinelanguage{Python}{
 keywords={typeof, null, catch, switch, in, int, str, float, self},
 keywordstyle=\color{ForestGreen}\bfseries,
 ndkeywords={boolean, throw, import},
 ndkeywords={return, class, if ,elif, endif, while, do, else, True, False , catch, def},
 ndkeywordstyle=\color{BrickRed}\bfseries,
 identifierstyle=\color{black},
 sensitive=false,
 comment=[l]{\#},
 escapechar=@,
 morecomment=[s]{/*}{*/},
 commentstyle=\color{purple}\ttfamily,
 stringstyle=\color{red}\ttfamily,
 basicstyle=\ttfamily\ABNTEXfontereduzida,
 breakatwhitespace=false,         % sets if automatic breaks should only happen at whitespace
 extendedchars=true,              % lets you use non-ASCII characters; for 8-bits encodings only, does not work with UTF-8
 inputencoding=utf8,
 literate={á}{{\'a}}1 {ã}{{\~a}}1 {é}{{\'e}}1 {è}{{\`{e}}}1 {ê}{{\^{e}}}1 {ë}{{\¨{e}}}1 {É}{{\'{E}}}1 {Ê}{{\^{E}}}1 {û}{{\^{u}}}1 {ú}{{\'{u}}}1 {â}{{\^{a}}}1 {à}{{\`{a}}}1 {á}{{\'{a}}}1 {ã}{{\~{a}}}1 {Á}{{\'{A}}}1 {Â}{{\^{A}}}1 {Ã}{{\~{A}}}1 {ç}{{\c{c}}}1 {Ç}{{\c{C}}}1 {õ}{{\~{o}}}1 {ó}{{\'{o}}}1 {ô}{{\^{o}}}1 {Õ}{{\~{O}}}1 {Ó}{{\'{O}}}1 {Ô}{{\^{O}}}1 {î}{{\^{i}}}1 {Î}{{\^{I}}}1 {í}{{\'{i}}}1 {Í}{{\~{Í}}}1,
  %if you want to add more keywords to the set
  keepspaces=true,                 % keeps spaces in text, useful for keeping indentation of code (possibly needs columns=flexible)
  frame=single,                    % adds a frame around the code
  breaklines=true,                 % sets automatic line breaking
  alsoother={0123456789_},
  numberbychapter=false,
  numbers=left,                    % where to put the line-numbers; possible values are (none, left, right)
  numbersep=5pt,                   % how far the line-numbers are from the code
  % the style that is used for the line-numbers
  numberstyle=\tiny\color{blue!70!black!70!}\sffamily, 
  rulecolor=\color{blue!70!black!70!},         % if not set, the frame-color may be changed on line-breaks within not-black text (e.g. comments (green here))
  showspaces=false,                % show spaces everywhere adding particular underscores; it overrides 'showstringspaces'
  showstringspaces=false,          % underline spaces within strings only
  showtabs=false,                  % show tabs within strings adding particular underscores
  stepnumber=4,                    % the step between two line-numbers. If it's 1, each line will be numbered
  tabsize=4,                       % sets default tabsize to 2 spaces
  title=\lstname,                  % show the filename of files included with \lstinputlisting; also try caption instead of title
  framexleftmargin=10pt,
  framexleftmargin=15pt
}   
   
% Cria uma nova customização para a linguagem Prolog
%\lstloadlanguages{Prolog}
%\lstdefinestyle{prologCustom}{
%  backgroundcolor=\color{white},   % choose the background color; you must add \usepackage{color} or \usepackage{xcolor}
%  % the size of the fonts that are used for the code
%  backgroundcolor=\color{theshade},
%  captionpos=t,                    % sets the caption-position to bottom
%  commentstyle=\color{mygreen},    % comment style
%  deletekeywords={...},            % if you want to delete keywords from the given language
%  escapeinside={\%*}{*)},          % if you want to add LaTeX within your code
%  keywordstyle=\color{blue},       % keyword style
%  language=Prolog,                 % the language of the code
%  stringstyle=\color{mymauve}\itshape,     % string literal style
%  morekeywords={*, :-},

%  keepspaces=true,                 % keeps spaces in text, useful for keeping indentation of code (possibly needs columns=flexible)
%  frame=single,                    % adds a frame around the code
%  breaklines=true,                 % sets automatic line breaking
%  alsoother={0123456789_},
%  basicstyle=\ttfamily\ABNTEXfontereduzida,
%  breakatwhitespace=false,         % sets if automatic breaks should only happen at whitespace
%  extendedchars=true,              % lets you use non-ASCII characters; for 8-bits encodings only, does not work with UTF-8
%  inputencoding=utf8,
%  literate={á}{{\'a}}1 {ã}{{\~a}}1 {é}{{\'e}}1 {è}{{\`{e}}}1 {ê}{{\^{e}}}1 {ë}{{\¨{e}}}1 {É}{{\'{E}}}1 {Ê}{{\^{E}}}1 {û}{{\^{u}}}1 {ú}{{\'{u}}}1 {â}{{\^{a}}}1 {à}{{\`{a}}}1 {á}{{\'{a}}}1 {ã}{{\~{a}}}1 {Á}{{\'{A}}}1 {Â}{{\^{A}}}1 {Ã}{{\~{A}}}1 {ç}{{\c{c}}}1 {Ç}{{\c{C}}}1 {õ}{{\~{o}}}1 {ó}{{\'{o}}}1 {ô}{{\^{o}}}1 {Õ}{{\~{O}}}1 {Ó}{{\'{O}}}1 {Ô}{{\^{O}}}1 {î}{{\^{i}}}1 {Î}{{\^{I}}}1 {í}{{\'{i}}}1 {Í}{{\~{Í}}}1,
%  % if you want to add more keywords to the set
%  numberbychapter=false,
%  numbers=left,                    % where to put the line-numbers; possible values are (none, left, right)
%  numbersep=5pt,                   % how far the line-numbers are from the code
%  % the style that is used for the line-numbers
%  numberstyle=\tiny\color{theframe}\sffamily, 
%  rulecolor=\color{theframe},         % if not set, the frame-color may be changed on line-breaks within not-black text (e.g. comments (green here))
%  showspaces=false,                % show spaces everywhere adding particular underscores; it overrides 'showstringspaces'
%  showstringspaces=false,          % underline spaces within strings only
%  showtabs=false,                  % show tabs within strings adding particular underscores
%  stepnumber=4,                    % the step between two line-numbers. If it's 1, each line will be numbered
%  tabsize=4,                       % sets default tabsize to 2 spaces
%  title=\lstname,                  % show the filename of files included with \lstinputlisting; also try caption instead of title
%  framexleftmargin=10pt,
%  framexleftmargin=15pt
%}
%\lstset{escapechar=@,style=prologCustom}