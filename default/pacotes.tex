% ---
% Pacotes fundamentais
% ---
\usepackage{cmap}                               % Mapear caracteres especiais no PDF
\usepackage{lmodern}                            % Usa a fonte Latin Modern
\usepackage[T1]{fontenc}                        % Selecao de codigos de fonte.
\usepackage[utf8]{inputenc}                     % Codificacao do documento (conversão automática dos acentos)
\usepackage{lastpage}                           % Usado pela Ficha catalográfica
\usepackage{indentfirst}                        % Indenta o primeiro parágrafo de cada seção.
\usepackage[usenames,dvipsnames]{color}         % Controle das cores
\usepackage[usenames,dvipsnames,table]{xcolor}  % Para colorir textos
\usepackage{graphicx}                           % Inclusão de gráficos
% ---

% ---
% Pacotes adicionais
% ---
\usepackage{lipsum}                             % para geração de dummy text
%\usepackage[brazil, english]{babel}            % Gera tudo em Portugues-BR
%\usepackage{epsf}                              % Read EPS figures
\usepackage{amsthm,amsfonts,amsmath}            % Simbolos matematicos
\usepackage{multirow}                           % Tabelas com span de multiplas linhas
%\usepackage{verbatim}                          % Suporte a comentarios extensos
\usepackage{listings}                           % Inclusao de listagens de computador
%\usepackage{acronym}                           % Suporte a acronimos
\usepackage{enumerate}                         % Enumerate em alg. romandos
\usepackage{longtable}
\usepackage{lettrine}
\usepackage{pgfgantt}                           % para geração de gráfico de gantt
\usepackage{rotating}
\usepackage{tikz}                               % para desenhar
\usepackage{pdflscape}                          % usado para landscape de página
\usepackage{float}                              % Usado para posicionar imagens
\usepackage{enumitem}
%\usepackage[shortlabels]{enumitem}
\usepackage{spverbatim}
\usepackage{capa-epusp-abntex2}                 % Customizacao da capa e folha de rosto para poli-usp
\usepackage{pdfpages}                           % para importar pdfs para dentro do documento
\usepackage{xifthen}                            % provê o ifthen com o isempty
%\usepackage{paralist}                           % provê o inparaenum environment -> Causa erros no ITEMIZE
\usepackage{lipsum}
%\usepackage[subentrycounter,seeautonumberlist,nonumberlist=true]{glossaries}
\usepackage[xindy={language=portuguese},subentrycounter,seeautonumberlist,nonumberlist=true]{glossaries}
%\usepackage[acronym,toc]{glossaries}         % Load the package with the acronym option infos: http://bay.uchicago.edu/CTAN/macros/latex/contrib/glossaries/glossaries-user.html#acronymoptions
\usepackage{breakurl}
\usepackage[activate={true,nocompatibility},final,babel=true,tracking=true,kerning=true,spacing=true,factor=1100,stretch=10,shrink=10]{microtype}
% ---
% Pacotes de citações
% ---
%\usepackage[brazilian,hyperpageref]{backref}   % Paginas com as citações na bibl
\usepackage[alf,abnt-url-package=url]{abntex2cite}  % Citações padrão ABNT
