% ---
% Pacotes fundamentais
% ---
\usepackage{cmap}                               % Mapear caracteres especiais no PDF
\usepackage{lmodern}                            % Usa a fonte Latin Modern
\usepackage[T1]{fontenc}                        % Selecao de codigos de fonte.
\usepackage[utf8]{inputenc}                     % Codificacao do documento (conversão automática dos acentos)
\usepackage{lastpage}                           % Usado pela Ficha catalográfica
\usepackage{indentfirst}                        % Indenta o primeiro parágrafo de cada seção.
\usepackage[usenames,dvipsnames]{color}         % Controle das cores
\usepackage[usenames,dvipsnames,table]{xcolor}  % Para colorir textos
\usepackage{graphicx}                           % Inclusão de gráficos
\usepackage[activate={true,nocompatibility},final,babel=true,tracking=true,kerning=true,spacing=true,factor=1100,stretch=10,shrink=10]{microtype}
\usepackage{capa-epusp-abntex2}
% ---
% Pacotes adicionais
% ---
%\usepackage{lipsum}                            % para geração de dummy text
%\usepackage[brazil, english]{babel}            % Gera tudo em Portugues-BR
%\usepackage{verbatim}                          % Suporte a comentarios extensos
%\usepackage{acronym}                           % Suporte a acronimos
%\usepackage[shortlabels]{enumitem}
%\usepackage{lipsum}
\usepackage{amsthm,amsfonts,amsmath}            % Simbolos matematicos
\usepackage{array}                              % usado para centralizar células de tabelas
\usepackage{capa-epusp-abntex2}                 % Customizacao da capa e folha de rosto para poli-usp
\usepackage{enumerate}                          % Enumerate em alg. romandos
\usepackage{enumitem}
\usepackage{epsf}                               % Read EPS figures
\usepackage{epstopdf}
\usepackage{float}                              % Usado para posicionar imagens
\usepackage[noredefwarn,acronym,toc,subentrycounter,seeautonumberlist]{glossaries}
\usepackage{lettrine}
\usepackage{listings}                           % Inclusao de listagens de computador
\usepackage{longtable}
\usepackage{multirow}                           % Tabelas com span de multiplas linhas
\usepackage{pdflscape}                          % usado para landscape de página
\usepackage[final]{pdfpages}                    % Para fazer o include de arquivos PDF
\usepackage{pgfgantt}                           % para geração de gráfico de gantt
\usepackage{rotating}
\usepackage{spverbatim}
\usepackage{tikz}                               % para desenhar
\PassOptionsToPackage{hyphens}{url}             %\usepackage[hyphens]{url} % pacote para adicionar URLs no texto
\usepackage{hyperref}
\usepackage{xifthen}                            % provê o ifthen com o isempty
\usepackage{xparse}                             % For dualentry glossary + acronym personalized command
%\usepackage{underscore}                         % Para adicionar underscores nas urls
% Dúvidas se precisa
\usepackage{adjustbox}
%\usepackage{booktabs}
\usepackage{varwidth}
\usepackage{morewrites}                        % para evitar limitanção no número simultâneo de arquivos abertos pelo latex

% ---
% Pacotes de citações
% ---
\usepackage[brazilian,hyperpageref]{backref}   % Paginas com as citações na bibl
\usepackage[alf,abnt-url-package=url]{abntex2cite}  % Citações padrão ABNT
