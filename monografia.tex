%% abtex2-modelo-trabalho-academico.tex, v-1.8 laurocesar
%% Copyright 2012-2013 by abnTeX2 group at http://abntex2.googlecode.com/
%%
%% This work may be distributed and/or modified under the
%% conditions of the LaTeX Project Public License, either version 1.3
%% of this license or (at your option) any later version.
%% The latest version of this license is in
%%   http://www.latex-project.org/lppl.txt
%% and version 1.3 or later is part of all distributions of LaTeX
%% version 2005/12/01 or later.
%%
%% This work has the LPPL maintenance status `maintained'.
%%
%% The Current Maintainer of this work is the abnTeX2 team, led
%% by Lauro César Araujo. Further information are available on
%% http://abntex2.googlecode.com/
%%
%% This work consists of the files abntex2-modelo-trabalho-academico.tex,
%% abntex2-modelo-include-comandos and abntex2-modelo-references.bib
%%
 
% ------------------------------------------------------------------------
% ------------------------------------------------------------------------
% abnTeX2: Modelo de Trabalho Academico (tese de doutorado, dissertacao de
% mestrado e trabalhos monograficos em geral) em conformidade com
% ABNT NBR 14724:2011: Informacao e documentacao - Trabalhos academicos -
% Apresentacao
% ------------------------------------------------------------------------
% ------------------------------------------------------------------------

\documentclass[
  % -- opções da classe memoir --
  12pt,                                         % tamanho da fonte
  openright,                                   % capítulos começam em pág ímpar (insere página vazia caso preciso)
  openany,
  twoside,                                     % para impressão em verso e anverso. Oposto a oneside
  %oneside,
  a4paper,                                      % tamanho do papel.
  inline,
  % -- opções da classe abntex2 --
  %chapter=TITLE,                               % títulos de capítulos convertidos em letras maiúsculas
  %section=TITLE,                               % títulos de seções convertidos em letras maiúsculas
  %subsection=TITLE,                            % títulos de subseções convertidos em letras maiúsculas
  %subsubsection=TITLE,                         % títulos de subsubseções convertidos em letras maiúsculas
  % -- opções do pacote babel --
  english,                                      % idioma adicional para hifenização
  brazil,                                       % o último idioma é o principal do documento
  ]{abntex2}

% ---
% PACOTES
% ---
% Pacotes fundamentais
% ---
\usepackage{cmap}                               % Mapear caracteres especiais no PDF
\usepackage{lmodern}                            % Usa a fonte Latin Modern
\usepackage[T1]{fontenc}                        % Selecao de codigos de fonte.
\usepackage[utf8]{inputenc}                     % Codificacao do documento (conversão automática dos acentos)
\usepackage{lastpage}                           % Usado pela Ficha catalográfica
\usepackage{indentfirst}                        % Indenta o primeiro parágrafo de cada seção.
\usepackage[usenames,dvipsnames]{color}         % Controle das cores
\usepackage[usenames,dvipsnames,table]{xcolor}  % Para colorir textos
\usepackage{graphicx}                           % Inclusão de gráficos
\usepackage[activate={true,nocompatibility},final,babel=true,tracking=true,kerning=true,spacing=true,factor=1100,stretch=10,shrink=10]{microtype}
% ---
% Pacotes adicionais
% ---
%\usepackage{lipsum}                            % para geração de dummy text
%\usepackage[brazil, english]{babel}            % Gera tudo em Portugues-BR
%\usepackage{verbatim}                          % Suporte a comentarios extensos
%\usepackage{acronym}                           % Suporte a acronimos
%\usepackage[shortlabels]{enumitem}
%\usepackage{lipsum}
\usepackage{amsthm,amsfonts,amsmath}            % Simbolos matematicos
\usepackage{array}                              % usado para centralizar células de tabelas
\usepackage{capa-epusp-abntex2}                 % Customizacao da capa e folha de rosto para poli-usp
\usepackage{enumerate}                          % Enumerate em alg. romandos
\usepackage{enumitem}
\usepackage{epsf}                               % Read EPS figures
\usepackage{epstopdf}
\usepackage{float}                              % Usado para posicionar imagens
\usepackage[noredefwarn,acronym,toc,subentrycounter,seeautonumberlist]{glossaries}
\usepackage{lettrine}
\usepackage{listings}                           % Inclusao de listagens de computador
\usepackage{longtable}
\usepackage{multirow}                           % Tabelas com span de multiplas linhas
\usepackage{pdflscape}                          % usado para landscape de página
\usepackage[final]{pdfpages}                    % Para fazer o include de arquivos PDF
\usepackage{pgfgantt}                           % para geração de gráfico de gantt
\usepackage{rotating}
\usepackage{spverbatim}
\usepackage{tikz}                               % para desenhar
\PassOptionsToPackage{hyphens}{url}             %\usepackage[hyphens]{url} % pacote para adicionar URLs no texto
\usepackage{hyperref}
\usepackage{xifthen}                            % provê o ifthen com o isempty
\usepackage{xparse}                             % For dualentry glossary + acronym personalized command
%\usepackage{underscore}                         % Para adicionar underscores nas urls
% Dúvidas se precisa
\usepackage{adjustbox}
%\usepackage{booktabs}
\usepackage{varwidth}

% ---
% Pacotes de citações
% ---
\usepackage[brazilian,hyperpageref]{backref}   % Paginas com as citações na bibl
\usepackage[alf,abnt-url-package=url]{abntex2cite}  % Citações padrão ABNT

% ---
% CONFIGURAÇÕES DE PACOTES
% ---
% Configurações do pacote backref
% Usado sem a opção hyperpageref de backref
\renewcommand{\backrefpagesname}{Citado na(s) página(s):~}
% Texto padrão antes do número das páginas
\renewcommand{\backref}{}
% Define os textos da citação
\renewcommand*{\backrefalt}[4]{
  \ifcase #1 %
    Nenhuma citação no texto.%
  \or
    Citado na página #2.%
  \else
    Citado #1 vezes nas páginas #2.%
  \fi}%
% -----------------------------------------------------------------

%Corrigindo erro de URLs muito longas não quebrando linha
 \tolerance 1414
 \hbadness 1414
 \emergencystretch 1.5em
 \hfuzz 0.3pt
 \widowpenalty=10000
 \vfuzz \hfuzz
 \raggedbottom
% -----------------------------------------------------------------
 
%% Configurações do pacote listings
\lstdefinestyle{customc}{
  belowcaptionskip=1\baselineskip,
  breaklines=true,
  frame=L,
  %xleftmargin=\parindent,
  xleftmargin=0pt,
  language=bash,
  showstringspaces=false,
  basicstyle=\ttfamily\ABNTEXfontereduzida,
  keywordstyle=\bfseries\color{green!40!black},
  commentstyle=\itshape\color{purple!40!black},
  identifierstyle=\color{blue},
  stringstyle=\color{orange},
}
%% Configurações do pacote listings
\lstset{
    escapechar={!@!},
    style=customc,
    comment=[l]{\#},
    breakatwhitespace=false,         % sets if automatic breaks should only happen at whitespace
    extendedchars=true,              % lets you use non-ASCII characters; for 8-bits encodings only, does not work with UTF-8
    inputencoding=utf8,
    literate={á}{{\'a}}1 {ã}{{\~a}}1 {é}{{\'e}}1 {è}{{\`{e}}}1 {ê}{{\^{e}}}1 {ë}{{\¨{e}}}1 {É}{{\'{E}}}1 {Ê}{{\^{E}}}1 {û}{{\^{u}}}1 {ú}{{\'{u}}}1 {â}{{\^{a}}}1 {à}{{\`{a}}}1 {á}{{\'{a}}}1 {ã}{{\~{a}}}1 {Á}{{\'{A}}}1 {Â}{{\^{A}}}1 {Ã}{{\~{A}}}1 {ç}{{\c{c}}}1 {Ç}{{\c{C}}}1 {õ}{{\~{o}}}1 {ó}{{\'{o}}}1 {ô}{{\^{o}}}1 {Õ}{{\~{O}}}1 {Ó}{{\'{O}}}1 {Ô}{{\^{O}}}1 {î}{{\^{i}}}1 {Î}{{\^{I}}}1 {í}{{\'{i}}}1 {Í}{{\~{Í}}}1,
  %if you want to add more keywords to the set
}

% ---
% Formatação de código-fonte
% ---
% Altera o nome padrão do rótulo usado no comando \autoref{}
\renewcommand{\lstlistingname}{Código}

% Altera o rótulo a ser usando no elemento pré-textual "Lista de código"
\renewcommand{\lstlistlistingname}{Lista de códigos}

% Configura a ``Lista de Códigos'' conforme as regras da ABNT (para abnTeX2)
\begingroup\makeatletter
\let\newcounter\@gobble\let\setcounter\@gobbletwo
  \globaldefs\@ne \let\c@loldepth\@ne
  \newlistof{listings}{lol}{\lstlistlistingname}
  \newlistentry{lstlisting}{lol}{0}
\endgroup

\lstset{numberbychapter=false}

\renewcommand{\cftlstlistingaftersnum}{\hfill--\hfill}

\let\oldlstlistoflistings\lstlistoflistings
\renewcommand{\lstlistoflistings}{%
   \begingroup%
   \let\oldnumberline\numberline%
   \renewcommand{\numberline}{\lstlistingname\space\oldnumberline}%
   \oldlstlistoflistings%
   \endgroup}
   
\lstdefinelanguage{Python}{
 keywords={typeof, null, catch, switch, in, int, str, float, self},
 keywordstyle=\color{ForestGreen}\bfseries,
 ndkeywords={boolean, throw, import},
 ndkeywords={return, class, if ,elif, endif, while, do, else, True, False , catch, def},
 ndkeywordstyle=\color{BrickRed}\bfseries,
 identifierstyle=\color{black},
 sensitive=false,
 escapechar={!@!},
 morecomment=[s]{/*}{*/},
 commentstyle=\color{purple}\ttfamily,
 stringstyle=\color{red}\ttfamily,
  keepspaces=true,                 % keeps spaces in text, useful for keeping indentation of code (possibly needs columns=flexible)
  frame=single,                    % adds a frame around the code
  breaklines=true,                 % sets automatic line breaking
  alsoother={0123456789_},
  numberbychapter=false,
  numbers=left,                    % where to put the line-numbers; possible values are (none, left, right)
  numbersep=5pt,                   % how far the line-numbers are from the code
  % the style that is used for the line-numbers
  numberstyle=\tiny\color{blue!70!black!70!}\sffamily, 
  rulecolor=\color{blue!70!black!70!},         % if not set, the frame-color may be changed on line-breaks within not-black text (e.g. comments (green here))
  showspaces=false,                % show spaces everywhere adding particular underscores; it overrides 'showstringspaces'
  showstringspaces=false,          % underline spaces within strings only
  showtabs=false,                  % show tabs within strings adding particular underscores
  stepnumber=4,                    % the step between two line-numbers. If it's 1, each line will be numbered
  tabsize=4,                       % sets default tabsize to 2 spaces
  title=\lstname,                  % show the filename of files included with \lstinputlisting; also try caption instead of title
  framexleftmargin=10pt,
  framexleftmargin=15pt,
  comment=[l]{\#},
    breakatwhitespace=false,         % sets if automatic breaks should only happen at whitespace
    extendedchars=true,              % lets you use non-ASCII characters; for 8-bits encodings only, does not work with UTF-8
    inputencoding=utf8,
    literate={á}{{\'a}}1 {ã}{{\~a}}1 {é}{{\'e}}1 {è}{{\`{e}}}1 {ê}{{\^{e}}}1 {ë}{{\¨{e}}}1 {É}{{\'{E}}}1 {Ê}{{\^{E}}}1 {û}{{\^{u}}}1 {ú}{{\'{u}}}1 {â}{{\^{a}}}1 {à}{{\`{a}}}1 {á}{{\'{a}}}1 {ã}{{\~{a}}}1 {Á}{{\'{A}}}1 {Â}{{\^{A}}}1 {Ã}{{\~{A}}}1 {ç}{{\c{c}}}1 {Ç}{{\c{C}}}1 {õ}{{\~{o}}}1 {ó}{{\'{o}}}1 {ô}{{\^{o}}}1 {Õ}{{\~{O}}}1 {Ó}{{\'{O}}}1 {Ô}{{\^{O}}}1 {î}{{\^{i}}}1 {Î}{{\^{I}}}1 {í}{{\'{i}}}1 {Í}{{\~{Í}}}1,
} 

\lstdefinelanguage{bash}{
 keywords={typeof, null, catch, switch, in, int, str, float, self},
 keywordstyle=\color{ForestGreen}\bfseries,
 ndkeywords={boolean, throw, import},
 ndkeywords={return, class, if ,elif, endif, while, do, else, True, False , catch, def},
 ndkeywordstyle=\color{BrickRed}\bfseries,
 identifierstyle=\color{black},
 sensitive=false,
 escapechar={!@!},
 morecomment=[s]{/*}{*/},
 commentstyle=\color{purple}\ttfamily,
 stringstyle=\color{red}\ttfamily,
  keepspaces=true,                 % keeps spaces in text, useful for keeping indentation of code (possibly needs columns=flexible)
  frame=single,                    % adds a frame around the code
  breaklines=true,                 % sets automatic line breaking
  alsoother={0123456789_},
  numberbychapter=false,
  numbers=left,                    % where to put the line-numbers; possible values are (none, left, right)
  numbersep=5pt,                   % how far the line-numbers are from the code
  % the style that is used for the line-numbers
  numberstyle=\tiny\color{blue!70!black!70!}\sffamily, 
  rulecolor=\color{blue!70!black!70!},         % if not set, the frame-color may be changed on line-breaks within not-black text (e.g. comments (green here))
  showspaces=false,                % show spaces everywhere adding particular underscores; it overrides 'showstringspaces'
  showstringspaces=false,          % underline spaces within strings only
  showtabs=false,                  % show tabs within strings adding particular underscores
  stepnumber=4,                    % the step between two line-numbers. If it's 1, each line will be numbered
  tabsize=4,                       % sets default tabsize to 2 spaces
  title=\lstname,                  % show the filename of files included with \lstinputlisting; also try caption instead of title
  framexleftmargin=10pt,
  framexleftmargin=15pt,
  comment=[l]{\#},
    breakatwhitespace=false,         % sets if automatic breaks should only happen at whitespace
    extendedchars=true,              % lets you use non-ASCII characters; for 8-bits encodings only, does not work with UTF-8
    inputencoding=utf8,
    literate={á}{{\'a}}1 {ã}{{\~a}}1 {é}{{\'e}}1 {è}{{\`{e}}}1 {ê}{{\^{e}}}1 {ë}{{\¨{e}}}1 {É}{{\'{E}}}1 {Ê}{{\^{E}}}1 {û}{{\^{u}}}1 {ú}{{\'{u}}}1 {â}{{\^{a}}}1 {à}{{\`{a}}}1 {á}{{\'{a}}}1 {ã}{{\~{a}}}1 {Á}{{\'{A}}}1 {Â}{{\^{A}}}1 {Ã}{{\~{A}}}1 {ç}{{\c{c}}}1 {Ç}{{\c{C}}}1 {õ}{{\~{o}}}1 {ó}{{\'{o}}}1 {ô}{{\^{o}}}1 {Õ}{{\~{O}}}1 {Ó}{{\'{O}}}1 {Ô}{{\^{O}}}1 {î}{{\^{i}}}1 {Î}{{\^{I}}}1 {í}{{\'{i}}}1 {Í}{{\~{Í}}}1,
}

%\lstset{escapechar=@,style=prologCustom}

% ---
% Comandos personalizados
%%%%%%%%%%%%%%%%%%%%%%%%%%%%%%%%%%%%%%%%%%%%
%%        Comandos Gerais do Projeto      %%
%%%%%%%%%%%%%%%%%%%%%%%%%%%%%%%%%%%%%%%%%%%%

%% Comando gerador de Siglas
%%%% #1 = Sigla
%%%% #2 = Por extenso
\newcommand{\sigla}[2]{\emph{#1}\ifthenelse{\isempty{#2}}{}{: #2}}

%% Comando de texto temporário
\newcommand{\temporario}[1]{\textcolor{red}{#1}}

%%Destaque no texto
\newcommand{\destaque}[1]{\emph{#1}}

%%Espaçamento vertical interno de células de tabela
\newcommand{\espacoVert}{\vspace{3pt}}

%%Cor Padrão de cabeçalhos de tabelas
\newcommand{\headerColor}{RoyalBlue!70!black!80}
\newcommand{\headerFontStyle}{\sffamily\bfseries\color{white}}
\newcommand{\headerCell}[1]{
  %\multicolumn{1}{c|}{\cellcolor{\headerColor}\textcolor{white}{\sffamily\bfseries{#1}}}
  %\cellcolor{\headerColor}\textcolor{white}{\sffamily\bfseries{#1}}
  \multicolumn{1}{c|}{\sffamily\bfseries{#1}}%
}

% Para centralizar células de tabelas
\newcolumntype{P}[1]{>{\centering\arraybackslash}p{#1}}
\newcolumntype{M}[1]{>{\centering\arraybackslash}m{#1}}


\newcommand{\sptrans}{\textit{SPTrans}}
\hyphenation{SPTrans}
\newcommand{\gpssocial}{\textit{GPS Social}}
\newcommand{\bigdata}{\textit{Big Data}}
\newcommand{\trilhasp}{\textbf{\href{http://trilhasp.datapublika.com}{\#TrilhaSP}}}
\hyphenation{TrilhaSP}


%Needs xparse package - Dual Entry for glossary (glossary + acronym)
\DeclareDocumentCommand{\newdualentry}{ O{} O{} m m m m } {
  \newglossaryentry{gls-#3}{name={#5},text={#5\glsadd{#3}},
    description={#6},sort={#1},#1
  }
  \newacronym[see={[Glossário: ]{gls-#3}},#2]{#3}{#4}{#5\glsadd{gls-#3}}
}
%% Some shades
\definecolor{Dark}{gray}{0.2}
\definecolor{MedDark}{gray}{0.4}
\definecolor{Medium}{gray}{0.6}
\definecolor{Light}{gray}{0.8}

%Contador de numeração dos casos de uso
\newcounter{numcasodeuso}
%Sequencia de Eventos (Reiniciado a cada caso de uso)
\newcounter{indiceseqeventosCdU}
%Casos de Extensão (Reiniciado a cada caso de uso)
\newcounter{indiceextensaoCdU}
%Casos de Inclusão (Reiniciado a cada caso de uso)
\newcounter{indiceinclusaoCdU}

%Comando que cria um caso de uso e é usado internamente no ambiente
\newcommand{\criarcasodeuso}{%
    %\clearpage
    %\null \\
    %\hrule\ \\
    \hspace{-\parskip}\hspace{-\parindent}\textbf{Caso de uso UC\thenumcasodeuso :} \hypertarget{UC\thenumcasodeuso}{} \cmdNomeCasodeuso\\
    \null \\
    \textbf{Descrição:} \cmdDescCasodeuso\\
    \null \\
    \textbf{Evento iniciador:} \cmdEventoIniCasodeuso\\
    \null \\
    \textbf{Atores:} \cmdAtoresCasodeuso\\
    \null \\
    \textbf{Pré-condição:} \cmdPrecondCasodeuso\\
    \null \\
    \textbf{Sequência de eventos:} \cmdSeqEventosCasodeuso\\
    \null \\
    \textbf{Pós-condição:} \cmdPoscondCasodeuso\\
    \null \\
    \textbf{Extensão:} \cmdExtensaoCasodeuso\\
    \null \\
    \textbf{Inclusão:} \cmdInclusaoCasodeuso\\
    \null \\
    \textbf{Requisito:} \cmdRequisitoCasodeuso
    %\ \\ \hrule
    \clearpage
}

%Criação do ambiente Caso de Uso
\newenvironment{casodeuso}{%Zerando contadores
    \setcounter{indiceseqeventosCdU}{0}%
    \setcounter{indiceextensaoCdU}{0}%
    \setcounter{indiceinclusaoCdU}{0}%
    \stepcounter{numcasodeuso}%
}{\criarcasodeuso}

%Comandos internos
\newcommand{\cmdNomeCasodeuso}{}
%\newcommand{\cmdLabelCasodeuso}{}
\newcommand{\cmdDescCasodeuso}{}
\newcommand{\cmdEventoIniCasodeuso}{}
\newcommand{\cmdAtoresCasodeuso}{}
\newcommand{\cmdPrecondCasodeuso}{Nenhuma}
\newcommand{\cmdSeqEventosCasodeuso}{Nenhum}
\newcommand{\cmdPoscondCasodeuso}{Nenhuma}
\newcommand{\cmdExtensaoCasodeuso}{Nenhum}
\newcommand{\cmdInclusaoCasodeuso}{Nenhum}
\newcommand{\cmdRequisitoCasodeuso}{}
%%
%Comandos User Friendly de Casos de Uso
%
%Nome do Caso de Uso
\newcommand{\nomeCdU}[1]{\renewcommand{\cmdNomeCasodeuso}{#1}}
%Label
%\newcommand{\labelCdU}[1]{\renewcommand{\cmdLabelCasodeusp}{\label{#1}}}
%Descrição do Caso de Uso
\newcommand{\descricaoCdU}[1]{\renewcommand{\cmdDescCasodeuso}{#1}}
%Evento iniciador do caso de uso
\newcommand{\eventoiniciadorCdU}[1]{\renewcommand{\cmdEventoIniCasodeuso}{#1}}
%Atores do caso de uso
\newcommand{\atoresCdU}[1]{\renewcommand{\cmdAtoresCasodeuso}{#1}}
%Pré-condição do caso de uso
\newcommand{\precondicaoCdU}[1]{\renewcommand{\cmdPrecondCasodeuso}{#1}}
%Contagem de eventos
\newcommand{\contaeventoCdU}{%
    \stepcounter{indiceseqeventosCdU}
    \theindiceseqeventosCdU
}
%Sequencia de Eventos
\newcommand{\eventosCdU}[1]{
    \ifdefstring{\cmdSeqEventosCasodeuso}{Nenhum}{%
        \expandafter\renewcommand\expandafter\cmdSeqEventosCasodeuso\expandafter{%
            \\\null \hspace{.65cm}\contaeventoCdU. #1%
        }
    }{%
        \expandafter\renewcommand\expandafter\cmdSeqEventosCasodeuso\expandafter{%
            \cmdSeqEventosCasodeuso\\\null \hspace{.65cm}\contaeventoCdU. #1%
        }
    }
}
%Pós-condições
\newcommand{\poscondicaoCdU}[1]{\renewcommand{\cmdPoscondCasodeuso}{#1}}
%Contador de extensões do caso de uso
\newcommand{\contaextensaoCdU}{%
    \stepcounter{indiceextensaoCdU}
    \theindiceextensaoCdU
}
%Casos de uso de extensão
\newcommand{\extensaoCdU}[1]{
    \ifdefstring{\cmdExtensaoCasodeuso}{Nenhum}{%
        \expandafter\renewcommand\expandafter\cmdExtensaoCasodeuso\expandafter{\\\null \hspace{.65cm}\contaextensaoCdU. #1}
    }{%
        \expandafter\renewcommand\expandafter\cmdExtensaoCasodeuso\expandafter{%
            \cmdExtensaoCasodeuso\\\null \hspace{.65cm}\contaextensaoCdU. #1%
        }
    }
}
%Contador de casos de inclusão
\newcommand{\containclusaoCdU}{%
    \stepcounter{indiceinclusaoCdU}
    \theindiceinclusaoCdU
}
%Casos de uso de inclusão
\newcommand{\inclusaoCdU}[1]{
    \ifdefstring{\cmdInclusaoCasodeuso}{Nenhum}{%
        \expandafter\renewcommand\expandafter\cmdInclusaoCasodeuso\expandafter{\\\null \hspace{.65cm}\containclusaoCdU. #1}
    }{%
        \expandafter\renewcommand\expandafter\cmdInclusaoCasodeuso\expandafter{%
            \cmdInclusaoCasodeuso\\\null \hspace{.65cm}\containclusaoCdU. #1%
        }
    }
}
%Requisitos do caso de uso
\newcommand{\requisitoCdU}[1]{\renewcommand{\cmdRequisitoCasodeuso}{\hyperlink{#1}{#1}}}

%%%%%%%%%%%%%%%%%%%%%%%%%%%%%%%%%%%%%%%%%%%%%%%%%%%%%%%%%%%%%%%%%%%%%%%%%%%%%%%
%%                                DIAGRAMAS                                  %%
%%%%%%%%%%%%%%%%%%%%%%%%%%%%%%%%%%%%%%%%%%%%%%%%%%%%%%%%%%%%%%%%%%%%%%%%%%%%%%%
% PARÂMETROS:
%#1 = nome_da_imagem
%#2 = valor para escalar a imagem ('1' = 100% ; '0.5' = 50% ; etc)
%#3 = caption a ser utilizado na imagem/diagrama
%#4 = label a ser utilizada com a imagem/diagrama -> Não pode conter espaços!
%%%%%%
% USO
%Diagramas e imagens com página em formato retrato
%\diagramaRetrato{nome_imagem}{valor_escala}{texto_caption}{texto_label}{autoria}
\newcommand{\diagramaRetrato}[5]{
    \begin{figure}[H]%
        \centering%
        \caption{#3}%
        \label{fig:#4}%
        \includegraphics[scale=#2]{./imagens/#1}%
        \legend{Fonte: #5}%
    \end{figure}%
}

%Diagramas e imagens com página em formato paisagem
%\diagramaRetrato{nome_imagem}{valor_escala}{texto_caption}{texto_label}{autoria}
\newcommand{\diagramaPaisagem}[5]{
    \begin{landscape}
    \begin{figure}[H]%
        \centering%
        \caption{#3}%
        \label{fig:#4}%
        \includegraphics[scale=#2]{./imagens/#1}%
        \legend{Fonte: #5}%
    \end{figure}%
    \end{landscape}
}
%%%%%%%%%%%%%%%%%%%%%%%%%%%%%%%%%%%%%%%%%%%%%%%%%%%%%%%%%%%%%%%%%%%%%%%%%%%%%%%
%%                                REQUISITOS                                 %%
%%%%%%%%%%%%%%%%%%%%%%%%%%%%%%%%%%%%%%%%%%%%%%%%%%%%%%%%%%%%%%%%%%%%%%%%%%%%%%%
%%%%%%%%%%%%%%%%%%%%%%%%%%%%%%%%%%%%%%%
%%              USO                  %%
%%%%%%%%%%%%%%%%%%%%%%%%%%%%%%%%%%%%%%%
%
%\begin{Requisito}
%    \ReqTipo{<ARGUMENTO 1>}
%    \ReqNome{<ARGUMENTO 2>}
%    \ReqLabel{<LABEL>}
%    \ReqDescr{<ARGUMENTO 3>}
%    \ReqPrioridade{<ARGUMENTO 4>}
%    \ReqStatus{<ARGUMENTO 5>}
%    \ReqEstabilidade{<ARGUMENTO 6>}
%    \ReqOrigem{<ARGUMENTO 7>}
%    \ReqRationale{<ARGUMENTO 8>}
%    \ReqAssoc{<ARGUMENTO 9>}
%\end{Requisito}
%
%%%%%%%%%%%%%%%%%%%%%%%%%%%%%%%%%%%%%%%
%%     DESCRIÇÃO DOS ARGUMENTOS      %%
%%%%%%%%%%%%%%%%%%%%%%%%%%%%%%%%%%%%%%%
%% *** Nenhum argumento pode conter caracteres especiais, como o 'underscore' ***
% <LABEL>: Label a ser usada para referencias internas do documento
% <ARGUMENTO 1>: TIPO: ´Funcional´ (funcional) ou ´Nao Funcional´ (nao funcional)
% <ARGUMENTO 2>: Nome do requisito
% <ARGUMENTO 3>: Descrição do Requisito
% <ARGUMENTO 4>: Prioridade [alta, media, baixa]
% <ARGUMENTO 5>: Status [proposto, aprovado, incorporado, validado]
% <ARGUMENTO 6>: Estabilidade [alta, media, baixa]
% <ARGUMENTO 7>: Origem [usuario, interna, externa]
% <ARGUMENTO 8>: Rationale - texto extenso
% <ARGUMENTO 9>: Requisitos Associados - por enquanto tem que listar na mão
%%%%%%%%%%%%%%%%%%%%%%%%%%%%%%%%%%%%%%%%%%%%%%%%%%%%%%%%%%%%%%%%%%%%%%%%%%%%%%%
%
%
% CONTADORES DOS REQUISITOS
\newcounter{indiceFuncional}
\newcounter{indiceNaoFuncional}
\newcounter{clearPageCounter}
%
%
% Requisitos Funcionais
\newcommand{\StartReqFunc}{%
    % New job (that is, file)
    \newwrite\reqFunc%
    \immediate\openout\reqFunc=tabRequisitosFunc.aux%
    \setcounter{clearPageCounter}{0}%
}
%
%% Usamos unexpanded... mas eu não sei porque. No meu
%% computador não funciona se removo o unexpanded. Acho
%% que deve ter algo haver com o `\\` dentro da expressão.
%% (Aliás, isso é muito provável, já que o significado
%% de `\\` é diferente no texto e em um ambiente tabular;
%% sua expansão tem que ser postergada para o momento em
%% que a tabela está sendo impressa (`PrintVolunteers`).
\newcommand{\AppendRequisitoFunc}[3]{%
    \immediate\write\reqFunc{###1##rf-#2## & \expandafter#3 \unexpanded{\\}}%
}
%
\newcommand{\PrintRequisitosFunc}{%
	\begin{tabular}{cl}%
	\toprule
	    \headerCell{Identificação} & \headerCell{Requisito} \\%
	\midrule%
	\immediate\closeout\reqFunc\input{tabRequisitosFunc2.aux}%
	\bottomrule%
	\end{tabular}%
}
%
%
%
% Requisitos Não Funcionais
\newcommand{\StartReqNFunc}{%
    % New job (that is, file)
    \newwrite\reqNFunc%
    \immediate\openout\reqNFunc=tabRequisitosNFunc.aux%
    \setcounter{clearPageCounter}{0}%
}
%
%% Usamos unexpanded... mas eu não sei porque. No meu
%% computador não funciona se removo o unexpanded. Acho
%% que deve ter algo haver com o `\\` dentro da expressão.
%% (Aliás, isso é muito provável, já que o significado
%% de `\\` é diferente no texto e em um ambiente tabular;
%% sua expansão tem que ser postergada para o momento em
%% que a tabela está sendo impressa (`PrintVolunteers`).
\newcommand{\AppendRequisitoNFunc}[3]{%
    \immediate\write\reqNFunc{###1##rnf-#2##  & \expandafter#3 \unexpanded{\\}}%
}
%
\newcommand{\PrintRequisitosNFunc}{%
	\begin{tabular}{cl}%
	\toprule
	    \headerCell{Identificação} & \headerCell{Requisito} \\%
	\midrule%
	\immediate\closeout\reqNFunc\input{tabRequisitosNFunc2.aux}%
	\bottomrule%
	\end{tabular}%
}
%
%
%
%%%%%
\newenvironment{Requisito}{}{%
    \CriaRequisito%
}
%
%
% FUNÇÕES INTERNAS
\newcommand{\CmdReqTipo}{}%
\newcommand{\CmdReqLabel}{}%
\newcommand{\CmdReqNome}{}%
\newcommand{\CmdReqDescr}{}%
\newcommand{\CmdReqPrioridade}{}%
\newcommand{\CmdReqStatus}{}%
\newcommand{\CmdReqEstabilidade}{}%
\newcommand{\CmdReqOrigem}{}%
\newcommand{\CmdReqRationale}{}%
\newcommand{\CmdReqAssoc}{}%
%
%
%
%FUNÇÕES USER FRIENDLY
\newcommand{\ReqTipo}[1]{%
    \ifstrequal{#1}{funcional}{%
        \stepcounter{indiceFuncional}%
        %\ReqLabel{RF\theindiceFuncional}%
        \renewcommand{\CmdReqTipo}{\multicolumn{3}{|l|}{\textbf{No:} RF\theindiceFuncional~ (\hypertarget{rf-\CmdReqLabel}{rf-\CmdReqLabel})} & (X) Funcional & ( ) Não Funcional \\}
        \AppendRequisitoFunc{RF\theindiceFuncional}{\CmdReqLabel}{\CmdReqNome}
    }{%
        \stepcounter{indiceNaoFuncional}%
        \ReqLabel{RNF\theindiceNaoFuncional}%
        \renewcommand{\CmdReqTipo}{\multicolumn{3}{|l|}{\textbf{No:} RNF\theindiceNaoFuncional~ (\hypertarget{rnf-\CmdReqLabel}{rnf-\CmdReqLabel})} & ( ) Funcional & (X) Não Funcional \\}
        \AppendRequisitoNFunc{RNF\theindiceNaoFuncional}{\CmdReqLabel}{\CmdReqNome}
    }
}
\newcommand{\ReqLabel}[1]{\renewcommand{\CmdReqLabel}{#1}}%
\newcommand{\ReqNome}[1]{\renewcommand{\CmdReqNome}{#1}}%
\newcommand{\ReqDescr}[1]{\renewcommand{\CmdReqDescr}{#1}}%
\newcommand{\ReqPrioridade}[1]{%
    \ifstrequal{#1}{alta}{%
        \renewcommand{\CmdReqPrioridade}{(X) Alta & ( ) Média & ( ) Baixa & \\}
    }{%
        \ifstrequal{#1}{media}{%
            \renewcommand{\CmdReqPrioridade}{( ) Alta & (X) Média & ( ) Baixa & \\}
        }{%
            \renewcommand{\CmdReqPrioridade}{( ) Alta & ( ) Média & (X) Baixa & \\}
        }
    }
}
\newcommand{\ReqStatus}[1]{
    \ifstrequal{#1}{proposto}{%
        \renewcommand{\CmdReqStatus}{(X) Proposto & ( ) Aprovado & ( ) Incorporado & ( ) Válido \\}
    }{
        \ifstrequal{#1}{aprovado}{%
            \renewcommand{\CmdReqStatus}{( ) Proposto & (X) Aprovado & ( ) Incorporado & ( ) Válido \\}
        }{
            \ifstrequal{#1}{incorporado}{%
                \renewcommand{\CmdReqStatus}{( ) Proposto & ( ) Aprovado & (X) Incorporado & ( ) Válido \\}
            }{
                \renewcommand{\CmdReqStatus}{( ) Proposto & ( ) Aprovado & ( ) Incorporado & (X) Válido \\}
            }%
        }%
    }
}%
\newcommand{\ReqEstabilidade}[1]{
    \ifstrequal{#1}{alta}{%
        \renewcommand{\CmdReqEstabilidade}{(X) Alta & ( ) Média & ( ) Baixa & \\}
    }{%
        \ifstrequal{#1}{media}{%
            \renewcommand{\CmdReqEstabilidade}{( ) Alta & (X) Média & ( ) Baixa & \\}
        }{%
            \renewcommand{\CmdReqEstabilidade}{( ) Alta & ( ) Média & (X) Baixa & \\}
        }
    }
}
\newcommand{\ReqOrigem}[1]{
    \ifstrequal{#1}{usuario}{%
        \renewcommand{\CmdReqOrigem}{(X) Usuário & ( ) Interna & ( ) Externa & \\}%
    }{%
        \ifstrequal{#1}{interna}{%
            \renewcommand{\CmdReqOrigem}{( ) Usuário & (X) Interna & ( ) Externa & \\}%
        }{
            \renewcommand{\CmdReqOrigem}{( ) Usuário & ( ) Interna & (X) Externa & \\}%
        }%
    }%
}
\newcommand{\ReqRationale}[1]{\renewcommand{\CmdReqRationale}{#1}}%
\newcommand{\ReqAssoc}[1]{\renewcommand{\CmdReqAssoc}{#1}}%
%
%
%Função Requisito - utilizada indiretamente pelo ambiente Requisito
\newcommand{\CriaRequisito}{
    \begin{table}[!h]
        \begin{tabular}{|lllll|}%
            \hline%
                \CmdReqTipo%
            \hline%
                \multicolumn{5}{|l|}{%
                    \begin{minipage}{1\textwidth}%
                        \espacoVert%
                        \textbf{Requisito:} \CmdReqNome%#2%
                        \espacoVert%
                    \end{minipage}%
                } \\%
            \hline%
                \multicolumn{5}{|l|}{%
                    \begin{minipage}{1\textwidth}%
                        \espacoVert%
                        \textbf{Descrição:} \CmdReqDescr%#3%
                        \espacoVert%
                    \end{minipage}%
                } \\%
            \hline%
                \textbf{Prioridade:} & \CmdReqPrioridade%
            \hline%
                \textbf{Status:} & \CmdReqStatus%
            \hline%
                \textbf{Estabilidade:} & \CmdReqEstabilidade%
            \hline%
                \textbf{Origem:} & \CmdReqOrigem%
            \hline%
                \multicolumn{5}{|l|}{%
                    \begin{minipage}{1\textwidth}%
                        \espacoVert%
                        \textbf{Rationale:} \CmdReqRationale%#8%
                        \espacoVert%
                    \end{minipage}%
                } \\%
            \hline%
                \multicolumn{5}{|l|}{%
                    \begin{minipage}{1\textwidth}%
                        \espacoVert%
                        \textbf{Requisitos Associados:} \CmdReqAssoc%#9%
                        \espacoVert%
                     \end{minipage}%
                 } \\%
             \hline%
        \end{tabular}
     \end{table}     
    \stepcounter{clearPageCounter}
    \ifnum\value{clearPageCounter} = 3%
        \clearpage%
        \setcounter{clearPageCounter}{0}%
    \fi
}% FIM DO COMMAND CriaRequisito
%


% ---
% Informações de dados para CAPA e FOLHA DE ROSTO
\titulo{\#TrilhaSP}
\autor{Diego Rabatone Oliveira}
\local{São Paulo}
\data{Dezembro/2014}
\orientador[Orientador:]{Prof. Dr. Reginaldo Arakaki}
\coorientador[Co-orientadora:]{Eng$^a$. Haydée Svab}
\instituicao{%
    Universidade de São Paulo -- USP
    \par
    Escola Politécnica - EP
    \par 
    Graduação em Engenharia Elétrica - Ênfase Computação}
\tipotrabalho{Monografia}
% O preambulo deve conter o tipo do trabalho, o objetivo,
% o nome da instituição e a área de concentração
\preambulo{Monografia apresentada à Escola Politécnica da Universidade de São Paulo para a Conclusão do Curso de Engenharia.}
\areaconcentracao{Engenharia Elétrica - Computação}

% ---

% ---
% Configurações de aparência do PDF final
% alterando o aspecto da cor azul
\definecolor{blue}{RGB}{41,5,195}

% informações do PDF
\makeatletter
\hypersetup{
  %pagebackref=true,
  pdftitle={\@title},
  pdfauthor={\@author},
  pdfsubject={\imprimirpreambulo},
  pdfcreator={PDFLaTeX with abnTeX2},
  pdfkeywords={transporte}{tcc}{mobile}{software livre}{gps social},
  colorlinks=true,           % false: boxed links; true: colored links
  linkcolor=blue,            % color of internal links
  citecolor=blue,            % color of links to bibliography
  filecolor=magenta,         % color of file links
  urlcolor=blue,
  bookmarksdepth=4
}
\makeatother

% ---

% ---
% Espaçamentos entre linhas e parágrafos
% ---

% O tamanho do parágrafo é dado por:
\setlength{\parindent}{1.3cm}

% Controle do espaçamento entre um parágrafo e outro:
\setlength{\parskip}{0.2cm}  % tente também \onelineskip




% ---
% compila o indice
\makeindex
\makeglossaries
% ---
% entradas do glossario
% ---
\newacronym{rest}{REST}{\textit{Representational State Transfer}}
\newacronym{sgbd}{SGBD}{Sistema de Gerenciamento de Base de Dados}
\hyphenation{Django}%
\newglossaryentry{django}{%
    name=Django,%
    description={\'{e} um framework para desenvolvimento r\'{a}pido para web, escrito em Python, que utiliza o padr\~{a}o MVT}%
}%
\newglossaryentry{nginx}{%
    name=nginx,%
    description={\'{E} um dos servidore HTTP e de proxy reverso que mais tem ganho espa\c{c}o no mercado, em grande parte por seu elevado desempenho. Ele serve ainda como servidor de cache e permite trabalhar com \textit{load balance} entre diversos servidores}%
}%
\newglossaryentry{uwsgi}{%
    name=uWSGI,%
    description={\'{e} um servidor de aplica\c{c}\~{a}o para diversas linguagens de programa\c{c}\~{a}o, sendo Python uma delas, sendo que ele apresenta um dos melhores desempenhos}%
}%
\newglossaryentry{json}{%
    name=JSON,%
    description={acrônimo para ``\textit{JavaScript Object Notation}'', é um formato leve para intercâmbio de dados computacionais que foi proposto por Douglas Crockford e é descrito na \href{https://tools.ietf.org/html/rfc4627}{RFC 4627}}%
}%
\newglossaryentry{ionic}{%
    name=ionic,%
    description={\'{e} um SDK \textit{open source} de \textit{front-end} para desenvolvimento de aplicativos móveis híbridos, com HTML5. Mais informações em: \url{http://ionicframework.com/}}%
}%
\newglossaryentry{cordova}{%
    name=Cordova,%
    description={\'{e} uma plataforma para construção de aplicativos móveis nativos utilizando HTML, CSS e JavaScript. Em sua essência, ela encapsula uma aplicação ``web'' para cumprir tal tarefa. É uma plataforma desenvolvida pela Fundação APACHE. Mais informações em: \url{https://cordova.apache.org/}}%
}%
\newglossaryentry{angular}{%
    name=AngularJS,%
    description={\'{e} um \textit{framework} desenvolvido totalmente em JavaScript e desenvolvido pelo Google para desenvolvimento web}%
}%

\newdualentry{qrcode}{QRCode}{\textit{Quick Response Code}}{é um código de barras bidimensional, que ao ser lido pode ser traduzido em diversos tipos de conteúdos, como um texto, um número de telefone, uma URI, uma localização georreferenciada, etc. Existem diversos padrões que foram definidos ao longo do tempo, como, por exemplo: ISO/IEC 18004:2000\footnote{\url{http://www.iso.org/iso/iso_catalogue/catalogue_ics/catalogue_detail_ics.htm?csnumber=30789}} e ISO/IEC 18004:2006\footnote{\url{http://www.iso.org/iso/iso_catalogue/catalogue_tc/catalogue_detail.htm?csnumber=43655}}}
\newdualentry{mvt}{MVT}{\textit{Model View Template}}{\textit{Design Pattern} utilizada no framework Django similar à tradicional MVC (Model View Controller)}
\newacronym{its}{ITS}{Intelligent Transport Systems}
\newdualentry{ipi}{IPI}{Imposto sobre Porudtos Industrializados}{imposto que incide sobre os produtos industrializados nacionais e estrangeiros no momento do desembaraço aduaneiro de produto de procedência estrangeira, ou a saída do produto do estabelecimento industrial ou equiparado a industrial}
\newdualentry{api}{API}{Application Programming Interface}{ou Interface de Programação de Aplicações ou Interface de Programação de Aplicativos, é um conjunto de rotinas e padrões estabelecidos por um software para a utilização das suas funcionalidades por aplicativos que não pretendem envolver-se em detalhes da implementação do software, mas apenas usar seus serviços}
\newdualentry{mvc}{MVC}{\textit{Model View Controller}}{\textit{Design Pattern} muito conhecida e introduzida por \textit{Erich Gamma} em 1995}
\newdualentry{drf}{DRF}{\textit{Django Rest Framework}}{Módulo do Django para criar e expor uma API REST}
\newdualentry{psa}{PSA}{\textit{Python Social Auth}}{Módulo do Django para permitir autenticação com redes sociais}


%\newdualentry{tdd}{TDD}{\textit{Test Driven Development}}{ ou, em português, Desenvolvimento Orientado a Testes é uma técnica de desenvolvimento de software que baseia em um ciclo curto de repetições no qual escreve-se primeiro um teste para uma determinada funcionalidade e em seguida o código que vai fazer o programa passar no teste}
\glsresetall
\renewcommand*{\glsseeformat}[3][\seename]{\textit{#1}\glsseelist{#2}}
%\glsaddall

% ----
% Início do documento
\begin{document}
\frenchspacing  % Retira espaço extra obsoleto entre as frases.

% ----------------------------------------------------------
% ELEMENTOS PRÉ-TEXTUAIS
% ----------------------------------------------------------
\pretextual

% Capa
\imprimircapa

% Folha de rosto - (o * indica que haverá a ficha bibliográfica)
\imprimirfolhaderosto*

% ---
% Ficha Catalográfica - OBS: Gerada pela biblioteca da EPUSP
%TODO
\begin{fichacatalografica}
    %\includepdf{Ficha_Catalografica}
\end{fichacatalografica}

% ---
% Inserir errata
%\begin{errata}
%\end{errata}


% ---
% Inserir folha de aprovação
% ---
% Isto é um exemplo de Folha de aprovação, elemento obrigatório da NBR
% 14724/2011 (seção 4.2.1.3). Você pode utilizar este modelo até a aprovação
% do trabalho. Após isso, substitua todo o conteúdo deste arquivo por uma
% imagem da página assinada pela banca com o comando abaixo:
%
% \includepdf{folhadeaprovacao_final.pdf}
%
\begin{folhadeaprovacao}
  \begin{center}
    {\ABNTEXchapterfont\large\imprimirautor}

    \vspace*{\fill}\vspace*{\fill}
    \begin{center}
      \ABNTEXchapterfont\bfseries\Large\imprimirtitulo
    \end{center}
    \vspace*{\fill}

    \hspace{.45\textwidth}
    \begin{minipage}{.5\textwidth}
        \imprimirpreambulo
    \end{minipage}%
    \vspace*{\fill}

   Trabalho aprovado. \\
   \imprimirlocal, 17 de dezembro de 2014: 
   \end{center}

   \assinatura{\textbf{\imprimirorientador} \\ Orientador}
   \assinatura{\textbf{\imprimircoorientador} \\ Co-orientadora}
   \assinatura{\textbf{Prof. Dr. } \\ Convidado}
   %TODO
   %\assinatura{\textbf{Professor} \\ Convidado 2}
   %\assinatura{\textbf{Professor} \\ Convidado 4}

   \begin{center}
    \vspace*{0.5cm}
    {\large\imprimirlocal}
    \par
    {\large\imprimirdata}
    \vspace*{1cm}
  \end{center}
\end{folhadeaprovacao}

% ---
% Dedicatória
\begin{dedicatoria}
   \vspace*{\fill}
   \centering
   \noindent
   \textit{Este trabalho é dedicado às pessoas que batalham no dia-a-dia por uma democracia participativa, na qual a melhoria da coletividade se dá pelo trabalho conjunto de todas atrizes e atores sociais.} \vspace*{\fill} %TODO
\end{dedicatoria}

% ---
% Agradecimentos
\begin{agradecimentos}
%TODO
    À minha família, que me deu todas as oportunidades para que eu chegasse a este momento e sempre esteve ao meu lado, me apoiando e confiando em mim. Obrigado Maria Júlia Rabatone Oliveira, Luiz Francisco Oliveira, Pedro Rabatone Oliveira e Paula Rabatone Oliveira.

À minha grande companheira Haydée Svab, que também me co-orientou neste trabalho, por todo seu amor e apoio incondicional, com elogios, abraços, broncas e orientações. Sou muito feliz por tê-la em minha vida, e que muitas outras experiências venham para somarmos e construirmos juntos.

À Daniela B. Silva e ao Pedro Markun, em nome da comunidade Transparência\\
Hacker, que me abriram todo um mundo de colaboração e transformação de nossa sociedade, utilizando tecnologia, mas sem tê-la como objetivo ou única via.

Ao Leonardo Alexandre Ferreira Leite, Thiago Costa Paiva e Tássio Naia, em nome do PoliGNU, Grupo de Estudos de Software Livre da Poli-USP, que ajudei a fundar em 2009 e aonde pude aprender muito sobre compartilhamento, troca, colaboração, Software Livre, tecnologia, companheirismo e dedicação.

Às professores Cíntia Borges Margi, Maria Eugênia Boscov e ao professor Felipe Pait, por todo apoio dado ao PoliGNU e ao PoliGen, e por ajudarem a construir uma Escola melhor para todas e todos.

Ao Giuliano Salcas Olguin, Orientador Pedagógico da EPUSP, que de chefe se tornou um grande amigo, com quem tive muitas oportunidades de aprendizado e troca.

À Ângela Teresa Buscema, Assistente Técnico Acadêmico da Escola Politécnica, uma verdadeira mãe de todos os politécnicos e politécnicas.

Ao Guilherme Carmelo e à Tassiana Belarmino, uma verdadeira segunda família.

Aos(às) técnico/as de laboratório e funcionário/as da Universidade, em especial dos Bandejões.

Ao Guttember Nunes (Gutem), por todo apoio dado na reta final deste trabalho.

Ao Prof. Dr. João José Neto (JJ), por todas as horas de conversa, por todo conhecimento oferecido durante minha graduação, e por ser um grande exemplo.

E ao meu orientador, Prof. Dr. Reginaldo Arakaki, por todo aprendizado oferecido e pela confiança em meu trabalho e potencial.
\end{agradecimentos}

% ---
% Epígrafe
\begin{epigrafe}
    \vspace*{\fill}
  \begin{flushright}
  %TODO
    \textit{``Ninguém educa ninguém,\\ninguém se educa a si mesmo,\\os homens se educam entre si,\\ mediatizados pelo mundo.''\\
    (Freire, Paulo; Em: Pedagogia do Oprimido)}
  \end{flushright}
\end{epigrafe}
% ---

% ---
% RESUMOS
% resumo em português
\begin{resumo} %TODO
\lettrine{A}{ presente monografia} descreve a implementação do aplicativo \trilhasp, um aplicativo que se propõe a melhorar o fluxo de informações entre usuário e prestador de serviço público de transporte, tanto fornecendo aos gestores avaliações do sistema realizadas pelos usuários quanto disponibilizando aos usuários informações para uma tomada de decisão mais consciente ao utilizar o sistema de transporte.
Com o \trilhasp~os usuários do sistema público de transporte poderão avaliar o serviço, segundo critérios qualitativos como ``o ônibus estava muito lotado'', ``fui bem atendido pelo motorista'' e ``o ônibus estava sujo''. A opção por estes critérios qualitativos se deu pois identificou-se que eles influenciam na decisão do usuário sobre o modo de transporte preferido e não são facilmente mensurados por meio de tecnologias de automação, como é o caso da velocidade e frequência dos ônibus. Essas avaliações, que podem ser positivas ou negativas numa escala contínua, permitirão a criação de indicadores por ônibus que poderão ser utilizados pelas autoridades para melhorar o serviço e também influenciar no sistema de remuneração das empresas prestadoras de serviço.
O módulo ``Mapa'' mostrará aos usuários um mapa com todos os usuários conectados, o que permitirá ao usuário optar por ir ou não para o ponto de ônibus num determinado horário usando a informação de demanda e ``lotação'' do ponto naquele momento, o que pode melhorar a distribuição da demanda no sistema, e levar a uma melhora do serviço prestado.
Por fim, o módulo ``game'' tem por objetivo tanto atrair e reter os usuários no aplicativo quanto ser uma solução educativa, levando ao usuário informações e vivência verossímeis à realidade do sistema de transporte, como custo de um ônibus, necessidade de manutenção, etc.
O projeto foi desenvolvido utilizando tecnologias livres e também terá seu código fonte distribuído livremente.

\vspace{\onelineskip}

\vfill

\noindent 

\textbf{Palavras-chaves}: Aplicativo móvel, transporte público, mobilidade, software livre, gps social, gamificação, geolocalização
\end{resumo}

% resumo em inglês
\begin{resumo}[Abstract] %TODO
\lettrine{T}{his work} describes the implementation of the \trilhasp~application, that aims to improve the flow of information between users and service providers of the public transport system, so it can positively change the urban mobility of the city. It will allow the users to evaluate the service and also make available to the users crowdsourced information about other users so each user can take better decisions on how and when to use the system.%
%
~\trilhasp~users will be able to evaluate the transport service based on qualitative criteria such as ``How crowded was the bus'' or ``Were you well attended by the bus driver and the collector?''. This qualitative criteria was chosen because it was identified that they influence the users choice of transport mode and they are not as easy to measure with technology and automation as speed of bus frequency. This evaluations can be positive or negative on a continuous scale and they will allow the creation of indexes per bus vehicle so they can be used to improve the service and also be a criteria of service remuneration.%
%
~The ``map'' module will show a map with all currently connected users. This will allow them to take a more precise decision on how to go to take the bus. If the bus stop are too crowded the user can just postpone the trip and do other things before going to the bus stop, like going to a gym or finish the current work. This could influence the demand in a way that makes the system more rational.%
%
~There is also a ``game'' module, that aims both to engage the user on the app and also educate the user about the costs and needs of a transport system.%
%
~This project was developed with free software and it's own source code is available as a Free Software (as in freedom).

\vspace{\onelineskip}

\vfill

\noindent

Software livre, Transporte Urbano, Sistema de Posicionamento Global
\textbf{Keywords}: Free software, Urban Transportation, Global Positioning System
\end{resumo}

% ---
% inserir lista de ilustrações
\pdfbookmark[0]{\listfigurename}{lof}
\listoffigures*
\cleardoublepage

% ---
% inserir lista de listings
%% ---
\pdfbookmark[0]{\lstlistlistingname}{lol}
\begin{KeepFromToc}
\lstlistoflistings
\end{KeepFromToc}
\cleardoublepage
% ---

% ---
% inserir lista de tabelas
\pdfbookmark[0]{\listtablename}{lot}
\listoftables*
\cleardoublepage

% ---
% inserir lista de abreviaturas e siglas
%\begin{siglas}
    \newacronym{rest}{REST}{\textit{Representational State Transfer}}
\newacronym{sgbd}{SGBD}{Sistema de Gerenciamento de Base de Dados}
\hyphenation{Django}%
\newglossaryentry{django}{%
    name=Django,%
    description={\'{e} um framework para desenvolvimento r\'{a}pido para web, escrito em Python, que utiliza o padr\~{a}o MVT}%
}%
\newglossaryentry{nginx}{%
    name=nginx,%
    description={\'{E} um dos servidore HTTP e de proxy reverso que mais tem ganho espa\c{c}o no mercado, em grande parte por seu elevado desempenho. Ele serve ainda como servidor de cache e permite trabalhar com \textit{load balance} entre diversos servidores}%
}%
\newglossaryentry{uwsgi}{%
    name=uWSGI,%
    description={\'{e} um servidor de aplica\c{c}\~{a}o para diversas linguagens de programa\c{c}\~{a}o, sendo Python uma delas, sendo que ele apresenta um dos melhores desempenhos}%
}%
\newglossaryentry{json}{%
    name=JSON,%
    description={acrônimo para ``\textit{JavaScript Object Notation}'', é um formato leve para intercâmbio de dados computacionais que foi proposto por Douglas Crockford e é descrito na \href{https://tools.ietf.org/html/rfc4627}{RFC 4627}}%
}%
\newglossaryentry{ionic}{%
    name=ionic,%
    description={\'{e} um SDK \textit{open source} de \textit{front-end} para desenvolvimento de aplicativos móveis híbridos, com HTML5. Mais informações em: \url{http://ionicframework.com/}}%
}%
\newglossaryentry{cordova}{%
    name=Cordova,%
    description={\'{e} uma plataforma para construção de aplicativos móveis nativos utilizando HTML, CSS e JavaScript. Em sua essência, ela encapsula uma aplicação ``web'' para cumprir tal tarefa. É uma plataforma desenvolvida pela Fundação APACHE. Mais informações em: \url{https://cordova.apache.org/}}%
}%
\newglossaryentry{angular}{%
    name=AngularJS,%
    description={\'{e} um \textit{framework} desenvolvido totalmente em JavaScript e desenvolvido pelo Google para desenvolvimento web}%
}%

\newdualentry{qrcode}{QRCode}{\textit{Quick Response Code}}{é um código de barras bidimensional, que ao ser lido pode ser traduzido em diversos tipos de conteúdos, como um texto, um número de telefone, uma URI, uma localização georreferenciada, etc. Existem diversos padrões que foram definidos ao longo do tempo, como, por exemplo: ISO/IEC 18004:2000\footnote{\url{http://www.iso.org/iso/iso_catalogue/catalogue_ics/catalogue_detail_ics.htm?csnumber=30789}} e ISO/IEC 18004:2006\footnote{\url{http://www.iso.org/iso/iso_catalogue/catalogue_tc/catalogue_detail.htm?csnumber=43655}}}
\newdualentry{mvt}{MVT}{\textit{Model View Template}}{\textit{Design Pattern} utilizada no framework Django similar à tradicional MVC (Model View Controller)}
\newacronym{its}{ITS}{Intelligent Transport Systems}
\newdualentry{ipi}{IPI}{Imposto sobre Porudtos Industrializados}{imposto que incide sobre os produtos industrializados nacionais e estrangeiros no momento do desembaraço aduaneiro de produto de procedência estrangeira, ou a saída do produto do estabelecimento industrial ou equiparado a industrial}
\newdualentry{api}{API}{Application Programming Interface}{ou Interface de Programação de Aplicações ou Interface de Programação de Aplicativos, é um conjunto de rotinas e padrões estabelecidos por um software para a utilização das suas funcionalidades por aplicativos que não pretendem envolver-se em detalhes da implementação do software, mas apenas usar seus serviços}
\newdualentry{mvc}{MVC}{\textit{Model View Controller}}{\textit{Design Pattern} muito conhecida e introduzida por \textit{Erich Gamma} em 1995}
\newdualentry{drf}{DRF}{\textit{Django Rest Framework}}{Módulo do Django para criar e expor uma API REST}
\newdualentry{psa}{PSA}{\textit{Python Social Auth}}{Módulo do Django para permitir autenticação com redes sociais}


%\newdualentry{tdd}{TDD}{\textit{Test Driven Development}}{ ou, em português, Desenvolvimento Orientado a Testes é uma técnica de desenvolvimento de software que baseia em um ciclo curto de repetições no qual escreve-se primeiro um teste para uma determinada funcionalidade e em seguida o código que vai fazer o programa passar no teste}
\glsresetall
%\end{siglas}
\printglossaries
\cleardoublepage

% ---
% inserir lista de símbolos
%\begin{simbolos}
%\end{simbolos}

% ---
% inserir o sumario
\pdfbookmark[0]{\contentsname}{toc}
\tableofcontents*
\cleardoublepage

% ----------------------------------------------------------
% ELEMENTOS TEXTUAIS
% ----------------------------------------------------------
\textual

% Capítulos
\chapter{Introdução}
\label{chp:Introdução}

\lettrine{A}{mobilidade urbana} é hoje um dos mais latentes problemas das grandes cidades de todo o mundo. Em 2010, considerando 439 áreas urbanas nos Estados Unidos, os congestionamentos fizeram com que os motoristas gastassem 4,8 bilhões de horas e comprassem 7.2 bilhões de litros de combustível além do necessário, a um custo de \$101 bilhões de dólares~\cite{Eisele2011}. Já em 2011, esses valores subiram para 5,5 bilhões de horas, 11 bilhões de litros de combustível e \$121 bilhões de dólares~\cite{Schrank2012}. 
Assim, percebe-se que não apenas são altos os custos sociais, ambientais e econômicos, como também estes são crescentes.

A realidade brasileira não é muito diferente. De 2004 a 2007 o tempo de congestionamento de 4 das maiores cidades brasileiras (São Paulo, Rio de Janeiro, Belo Horizonte e Porto Alegre) cresceu a uma taxa anual média de 16\%~\cite{resende2009}. Se considerarmos ainda as políticas públicas de incentivo ao mercado automobilístico, mais especificamente a redução do \gls{ipi} para automóveis implementadas em maio de 2012~\cite{brasil2011} e ainda vigentes~\cite{brasil2012}, pode-se esperar que o crescimento do índice de congestionamentos no período tenha sido até maior.

Para além dos congestionamentos, existem ainda outras externalidades comuns na área do transporte, como a poluição e os acidentes de trânsito~\cite{vasconcellos1998}. A poluição, por exemplo, implica graves problemas de saúde pública, como o aumento da incidência de problemas cardiopulmonares, como bronquite e asma~\cite{kunzli2000}, e o aumento dos índices de mortalidade~\cite{finkelstein2004}.
Se essas externalidades possuem suas medidas de desempenhos já consagradas e já se estuda sobre as deseconomias que as causam, a externalidade relativa à insatisfação do usuário do transporte público é terreno ainda pouco explorado na área de desenvolvimento de software.
Sobram evidências de que há questionamentos quanto à qualidade do serviço prestado~\cite{UrbanaPE2010,Rodrigues,Rodrigues2006,Cellos2012}. Segundo~\citeonline{cardoso2013}, o reajuste das tarifas de ônibus foi o estopim das manifestações de junho de 2013, porém as insatisfações não se limitavam a elas, mas permeavam também a qualidade do serviço prestado como um todo. Bastante polêmicas e controversas, mesmo assim ainda podendo ser entendidas sob esse prisma, estão diversas queimas de ônibus como forma de protesto contra a má-qualidade do sistema. \citeonline{fora1999} apontam como uma das recomendações para se combater as deseconomias do transporte público haver financiamentos de estudos e projetos para melhor caracterizar “condições atuais de transporte e trânsito para subsidiar projetos de melhoria”.
Aqui este projeto encontra sua motivação e respaldo. Hoje, graças à emergência de recursos tecnológicos de comunicações que utilizam conceitos como redes, geolocalização, comunicação multidirecional e outros, pode-se definir processos que unam com facilidade os extremos das cadeias de transporte e, assim, permitir um melhor planejamento e correções de rumo mesmo durante a execução do planejamento do sistema de transporte, melhorando a sua eficiência tanto para o poder público, quanto para usuários e prestadores de serviço.

\section{Justificativa do projeto}\label{sec:justificativa}
	Atualmente existem diversos aplicativos que trabalham, majoritariamente, disponibilizando informações fornecidas pela SPTrans, por meio de sua \gls{api} para desenvolvedores\footnote{\url{http://www.sptrans.com.br/desenvolvedores/APIOlhoVivo/Documentacao.aspx}}, aos usuários. 
Isso se traduz em serviços de escolha de linha de ônibus, determinação de rotas e verificação de localização de ônibus e respectivo tempo estimado de chegada, e em alguns casos tempo estimado para se chegar ao destino escolhido.

Entretanto, não foram encontrados aplicativo que permitissem a coleta de avaliação dos usuários sobre elementos-chaves do sistema que não são automaticamente mensuráveis. Assim, o \trilhasp~se propõe a suprir essa demanda captando a opinião do usuário no momento do uso do transporte, permitindo a geração de informações de \textit{feedback} úteis tanto para usuários quanto para a SPTrans, equanto fiscalizadora do sistema, e para as concessionárias do sistema de transporte, permitindo um melhor controle da qualidade da operação.

Foi uma opção do projeto, neste primeiro momento, não inserir as funcionalidades já encontradas em praticamente todos os outros aplicativos e descritas no primeiro parágrafo deste item. Essas funcionalidades podem vir a ser adicionadas futuramente tornando o aplicativo mais completo, assim como são possíveis diversas evoluções, que serão exploradas na seção \ref{sec:futuro}.

\section{Objetivos}\label{sec:objetivos}
\subsection{Geral}\label{subsec:objGerais}
O objetivo geral do projeto é desenvolver um software que corrobore para a melhoria da mobilidade nas cidades pelo incremento da qualidade e da eficiência do sistema público de transporte urbano, mais especificamente rodoviário (ônibus). Será desenvolvido um produto considerando as condições de contorno da Região Metropolitana de São Paulo, que é o maior conglomerado urbano do Brasil com aproximadamente 20 milhões de habitantes. 

\subsection{Específicos}\label{subsec:objEspec}
São objetivos específicos do projeto: gerar dados e informações de valor para os atores envolvidos no sistema, a saber, operadores dos sistemas de transporte público, usuários e SPTrans; possibilitar a troca de informações triangular entre os três atores do sistema; corroborar para a eficiência do sistema pois com mais informação haverá mais controle de operação; promover transparência já que as informações geradas serão em sua maioria dados públicos sujeitos à Lei 12.527/2011; e estimular a conscientização para o usuário acerca do funcionamento do sistema público de transporte por ônibus.
	
\section{Escopo e critério de sucesso}\label{sec:Escopo}
	Ao final deste projeto o software contará com três módulos, a saber:
	\begin{itemize}%[\itshape a\upshape)]
		\item módulo de avaliação do sistema de transporte;
		\item módulo de “gamificação” que terá as funções de oferecer um jogo que atraia e incentive a participação dos usuários e também sirva para um propósito pedagógico de conscientizar os usuários sobre a gestão de um sistema de transporte; e
		\item módulo de ``gps social'', que levantará informações de demanda do sistema de transporte e as fornecerá aos usuários em tempo real por meio de um mapa;
	\end{itemize}

\section{Não-escopo}\label{sec:NãoEscopo}
	Não está no escopo de implementação deste projeto um ou mais módulos que permitam a integração do sistema com prestadores de serviço não ligados ao sistema de transporte. Um exemplo seria a integração com comércios próximos aos pontos de ônibus que poderiam oferecer descontos e ``promoções relâmpago'' para usuários de acordo com suas ``pontuações'' no game do aplicativo, enviando notificações das promoções aos celulares dos usuários próximos ao comércio.

%\section{Stakeholders}\label{sec:stakeholders}
%	Na matriz abaixo encontram-se os \emph{stakeholders} do projeto e seus respectivos poder e influência no projeto:
%
%	\bigskip
%	\begin{table}[H]
%	\centering
%	\caption{Stakeholders, relacionamento, interesse e poder}
%    \begin{tabular}{lccc}
%      \toprule
%      \headerCell{Parte interessada} &
%      \begin{minipage}{0.2\textwidth}
%        \espacoVert
%        \headerCell{Relacionamento com o Projeto}
%        \espacoVert
%      \end{minipage} &
%      \headerCell{Interesse (A/B)} &
%      \headerCell{Poder (A/B)}\\
%        
%      \midrule
%         
%			\textbf{Diego Rabatone Oliveira} &
%			Respons\'avel &
%			A &
%			A\\
%			
%			\textbf{Reginaldo Arakaki} &
%			Orientador &
%			A &
%			A\\
%
%			\textbf{Haydée Svab} &
%			Co-orientadora &
%			A &
%			A\\
%
%			\textbf{Prefeitura de São Paulo} &
%			Interessada &
%			B &
%			B\\
%			
%			\textbf{SPTrans} &
%			Interessada &
%			B &
%			B\\
%
%			\bottomrule
%	\end{tabular}
%	\end{table}

%\bigskip

\section{Restrições}\label{sec:restrições}
Todo desenvolvimento será realizado utilizando licenças livres e possiveis integrações com softwares terceiros devem levar este fato em consideração, ou seja, será preciso que haja compatibilidade de licenças.
%\section{Planejamento}\label{sec:planejamento}
\subsection{Escopo, Cronograma e Orçamento da EAP}\label{subsec:eap}
%\diagramaRetrato{EAP_2.pdf}{1}{EAP, cronograma e orçamento}{eap2}
Os itens \textit{1.2.2} e \textit{1.3} da Estrutura Analítica de Projeto dizem respeito às atividades do segundo semestre e por este motivo não foram completamente definidas/planejadas.\\
A previsão de horas de trabalho (HT) está na coluna \textit{Trabalho} da tabela \ref{tab:eapwbs}.\\
\begin{table}[H]
  \centering
  \caption {EAP/WBS}
    \begin{tabular}{llllll}
      \toprule

       \headerCell{EAP} & \headerCell{Nome} & \headerCell{Trabalho}  \\

      \midrule
      
      1        & TCC - \#TrilhaSP                          & 24d       \\
      1.1      & Documentação e especificação              & 20d 2h    \\
      1.1.1    & Definir Objetivos do projeto              & 3h        \\
      1.1.2    & Elaborar documento inicial                & 4h        \\
      1.1.3    & Realizar orçamento (HT) da EAP            & 2h        \\
      1.1.4    & Definir riscos do projeto                 & 5h        \\
      1.1.5    & Análise SWOT                              & 6h        \\
      1.1.6    & Levantar projetos existentes              & 3d 2h     \\
      1.1.6.1  & Lista de projetos                         & 6h        \\
      1.1.6.2  & Características dos projetos              & 2d        \\
      1.1.6.3  & Consolidar quadro de características      & 4h        \\
      1.1.7    & Definição dos módulos do projeto          & 1d 4h     \\
      1.1.7.1  & Módulos para os usuários                  & 1d        \\
      1.1.7.2  & Módulos para o Poder Público              & 4h        \\
      1.1.8    & Especificação dos módulos do projeto      & 7d        \\
      1.1.8.1  & Requisitos Funcionais                     & 2d        \\
      1.1.8.2  & Requisitos Não-Funcionais                 & 2d        \\
      1.1.8.3  & Modelagem de dados dos módulos            & 3d        \\
      1.1.9    & Especificação da Arquitetura              & 6d        \\
      1.1.9.1  & Estimativa de demanda de recursos         & 2d        \\
      1.1.9.2  & Levantamento de opções de infraestrutura  & 1d        \\
      1.1.9.3  & Definição da infraestrutura               & 2d        \\
      1.1.9.4  & Especificação de Infraestrutura           & 1d        \\
      1.2      & Relatórios e Apresentações                & 3d 6h     \\
      1.2.1    & 1o Semestre                               & 3d 6h     \\
      1.2.1.1  & Relatório Inicial 1sem                    & 2h        \\
      1.2.1.2  & Relatório Intermediário 1 (Abril)         & 3h        \\
      1.2.1.3  & Relatório Intermediário 2 (Maio)          & 3h        \\
      1.2.1.4  & Apresentação Intermediária 1sem           & 3h        \\
      1.2.1.5  & Relatório Final (Junho)                   & 2d        \\
      1.2.1.6  & Apresentação Final 1sem                   & 3h        \\
      1.2.2    & 2o Semestre                               & ~         \\
%      1.2.2.1  & Relatório Inicial 2sem                    & ~         \\
%      1.2.2.2  & Relatório Intermediário 1 (Setembro)      & ~         \\
%      1.2.2.3  & Relatório Intermediário 2 (Outubro)       & ~         \\
%      1.2.2.4  & Apresentação Intermediária 2sem           & ~         \\
%      1.2.2.5  & Relatório Final (Novembro)                & ~         \\
%      1.2.2.6  & Apresentação Final 2sem                   & ~         \\
      1.3      & Codificação                               & ~         \\
      1.3.1    & Prototipação (Mocks)                      & ~         \\
      1.3.1.1  & Protótipos de cada módulo                 & ~         \\
      1.3.1.2  & Design da interface                       & ~         \\
      1.3.2    & Produto Final (MVP)                       & ~         \\
      
      \bottomrule
    \end{tabular}
  \label{tab:eapwbs}
\end{table}
%\diagramaPaisagem{EAP_1.pdf}{0.9}{EAP e Linha do Tempo}{eap1}

\begin{landscape}
  \begin{figure}[ftbp]
  \caption{Gráfico de Gantt da EAP}
  \begin{center}
  \begin{ganttchart}[
    hgrid,
    vgrid,
    x unit=2.55mm,
    y unit title=0.4cm,
    y unit chart=0.57cm,
    %compress calendar,
    time slot format=isodate,
    title/.append style={draw=none, fill=\headerColor},
    title label font=\headerFontStyle,
    title label node/.append style={below=-1.6ex},
    title left shift=.05,
    title right shift=-.05,
    title height=1,
    bar/.append style={draw=none, fill=OliveGreen!75},
    bar label font=\footnotesize\color{black!50},
    bar incomplete/.append style={fill=Maroon},
    bar height=.2,
    bar top shift=.1,
    bar progress label node/.append style={above=3pt, right=2mm},
    group right shift=0,
    group top shift=.4,
    group height=.2,
    group peaks height=.2,
    %progress=today,
    %today=2014-05-04,
    %bar/.append style={fill=green},
    %bar progress label node/.append style={above=3pt, right=2mm}
  ]{2014-04-01}{2014-06-20}
  \gantttitlecalendar{month=name, week} \\
      \ganttgroup{\footnotesize{1}}{2014-04-2}{2014-06-18}\\ %TCC - \#TrilhaSP
      \ganttgroup{\footnotesize{1.1}}{2014-04-2}{2014-05-27}\\ %Documentação e especificação
      \ganttlinkedbar[progress=100]{\footnotesize{1.1.1}}{2014-04-2}{2014-04-2}\\ %Definir Objetivos do projeto
      \ganttlinkedbar[progress=100]{\footnotesize{1.1.2}}{2014-04-2}{2014-04-2}\\ %Elaborar documento inicial
      \ganttlinkedbar[progress=100]{\footnotesize{1.1.3}}{2014-04-2}{2014-04-3}\\ %Realizar orçamento (HT) da EAP
      \ganttlinkedbar{\footnotesize{1.1.4}}{2014-04-28}{2014-04-28}\\ %Definir riscos do projeto
      \ganttlinkedbar{\footnotesize{1.1.5}}{2014-04-28}{2014-04-29}\\ %Análise SWOT
      %\ganttgroup{\footnotesize{1.1.6}}{2014-04-29}{2014-05-2}\\ %Levantar projetos existentes
      \ganttlinkedbar[progress=80]{\footnotesize{1.1.6.1}}{2014-04-29}{2014-04-30}\\ %Lista de projetos
      %\ganttlink[link type=fin-to-sta]{elem6}{elem8}
      \ganttlinkedbar{\footnotesize{1.1.6.2}}{2014-04-30}{2014-05-2}\\ %Características dos projetos
      \ganttlinkedbar{\footnotesize{1.1.6.3}}{2014-05-2}{2014-05-2}\\ %Consolidar quadro de características
      %\ganttgroup{\footnotesize{1.1.7}}{2014-05-2}{2014-05-6}\\ %Definição dos módulos do projeto
      \ganttlinkedbar[progress=10]{\footnotesize{1.1.7.1}}{2014-05-2}{2014-05-5}\\ %Módulos para os usuários
      \ganttlinkedbar[progress=25]{\footnotesize{1.1.7.2}}{2014-05-5}{2014-05-6}\\ %Módulos para o Poder Público
      %\ganttgroup{\footnotesize{1.1.8}}{2014-05-6}{2014-05-15}\\ %Especificação dos módulos do projeto
      \ganttlinkedbar{\footnotesize{1.1.8.1}}{2014-05-6}{2014-05-8}\\ %Requisitos Funcionais
      \ganttlinkedbar{\footnotesize{1.1.8.2}}{2014-05-8}{2014-05-12}\\ %Requisitos Não-Funcionais
      \ganttlinkedbar{\footnotesize{1.1.8.3}}{2014-05-12}{2014-05-15}\\ %Modelagem de dados dos módulos
      %\ganttgroup{\footnotesize{1.1.9}}{2014-05-19}{2014-05-27}\\ %Especificação da Arquitetura
      \ganttlinkedbar[name=1191]{\footnotesize{1.1.9.1}}{2014-05-19}{2014-05-21}\\ %Estimativa de demanda de recursos
      \ganttlinkedbar{\footnotesize{1.1.9.2}}{2014-05-21}{2014-05-22}\\ %Levantamento de opções de infraestrutura
      \ganttlinkedbar{\footnotesize{1.1.9.3}}{2014-05-22}{2014-05-26}\\ %Definição da infraestrutura
      \ganttlinkedbar[name=1194]{\footnotesize{1.1.9.4}}{2014-05-26}{2014-05-27}\\ %Especificação de Infraestrutura
      \ganttgroup{\footnotesize{1.2}}{2014-05-2}{2014-06-18}\\ %Relatórios e Apresentações
      %\ganttgroup{\footnotesize{1.2.1}}{2014-05-2}{2014-06-18}\\ %1o Semestre
      \ganttbar[progress=100,name=1211]{\footnotesize{1.2.1.1}}{2014-05-2}{2014-05-2}\\ %Relatório Inicial 1sem
      \ganttlinkedbar[progress=90]{\footnotesize{1.2.1.2}}{2014-05-28}{2014-05-28}\\ %Relatório Intermediário 1 (Abril)
      \ganttlinkedbar{\footnotesize{1.2.1.3}}{2014-05-19}{2014-05-19}\\ %Relatório Intermediário 2 (Maio)
      \ganttlinkedbar[name=1214]{\footnotesize{1.2.1.4}}{2014-05-19}{2014-05-19}\\ %Apresentação Intermediária 1sem
      \ganttlinkedbar{\footnotesize{1.2.1.5}}{2014-06-16}{2014-06-17}\\ %Relatório Final (Junho)
      \ganttlinkedbar{\footnotesize{1.2.1.6}}{2014-06-18}{2014-06-18} %Apresentação Final 1sem
      \ganttlink[link type=fin-to-sta]{1214}{1191}
      %\ganttgroup{\footnotesize{1.2.2}}{2014-06-18}{2014-06-18}\\ %2o Semestre
      %\ganttbar{\footnotesize{1.2.2.1}}{2014-06-18}{2014-06-18}\\ %Relatório Inicial 2sem
      %\ganttbar{\footnotesize{1.2.2.2}}{2014-06-18}{2014-06-18}\\ %Relatório Intermediário 1 (Setembro)
      %\ganttbar{\footnotesize{1.2.2.3}}{2014-06-18}{2014-06-18}\\ %Relatório Intermediário 2 (Outubro)
      %\ganttbar{\footnotesize{1.2.2.4}}{2014-06-18}{2014-06-18}\\ %Apresentação Intermediária 2sem
      %\ganttbar{\footnotesize{1.2.2.5}}{2014-06-18}{2014-06-18}\\ %Relatório Final (Novembro)
      %\ganttbar{\footnotesize{1.2.2.6}}{2014-06-18}{2014-06-18}\\ %Apresentação Final 2sem
      %\ganttgroup{\footnotesize{1.3}}{2014-06-18}{2014-06-18}\\ %Codificação
      %\ganttgroup{\footnotesize{1.3.1}}{2014-06-18}{2014-06-18}\\ %Prototipação (Mocks)
      %\ganttbar{\footnotesize{1.3.1.1}}{2014-06-18}{2014-06-18}\\ %Protótipos de cada módulo
      %\ganttbar{\footnotesize{1.3.1.2}}{2014-06-18}{2014-06-18}\\ %Design da interface
      %\ganttbar{\footnotesize{1.3.2}}{2014-06-18}{2014-06-18} %Produto Final (MVP)
  \end{ganttchart}
  \end{center}
  \end{figure}
\end{landscape}

%\subsection{Matriz de responsabilidades}\label{subsec:matrizResp}

\begin{table}[H]
  \caption {Matriz de responsabilidades}
    \begin{tabular}{llc}
      \toprule
      \headerCell{EAP} & \headerCell{Nome} & \headerCell{Responsável} \\
      \midrule
      1        & TCC - \#TrilhaSP                          & Diego \\
      1.1      & Documentação e especificação              & Diego \\
      1.1.1    & Definir Objetivos do projeto              & Diego, Reginaldo e Haydée \\
      1.1.2    & Elaborar documento inicial                & Diego \\
      1.1.3    & Realizar orçamento (HT) da EAP            & Diego, Reginaldo \\
      1.1.4    & Definir riscos do projeto                 & Diego, Reginaldo e Haydée \\
      1.1.5    & Análise SWOT                              & Diego, Reginaldo e Haydée \\
      1.1.6    & Levantar projetos existentes              & Diego \\
      1.1.6.1  & Lista de projetos                         & Diego \\
      1.1.6.2  & Características dos projetos              & Diego e Haydée \\
      1.1.6.3  & Consolidar quadro de características      & Diego \\
      1.1.7    & Definição dos módulos do projeto          & Diego, Reginaldo e Haydée \\
      1.1.7.1  & Módulos para os usuários                  & Diego \\
      1.1.7.2  & Módulos para o Poder Público              & Diego \\
      1.1.8    & Especificação dos módulos do projeto      & Diego, Reginaldo e Haydée \\
      1.1.8.1  & Requisitos Funcionais                     & Diego \\
      1.1.8.2  & Requisitos Não-Funcionais                 & Diego \\
      1.1.8.3  & Modelagem de dados dos módulos            & Diego \\
      1.1.9    & Especificação da Arquitetura              & Diego e Reginaldo \\
      1.1.9.1  & Estimativa de demanda de recursos         & Diego \\
      1.1.9.2  & Levantamento de opções de infraestrutura  & Diego \\
      1.1.9.3  & Definição da infraestrutura               & Diego \\
      1.1.9.4  & Especificação de Infraestrutura           & Diego \\
      1.2      & Relatórios e Apresentações                & Diego \\
      1.2.1    & 1o Semestre                               & Diego \\
      1.2.1.1  & Relatório Inicial 1sem                    & Diego \\
      1.2.1.2  & Relatório Intermediário 1 (Abril)         & Diego \\
      1.2.1.3  & Relatório Intermediário 2 (Maio)          & Diego \\
      1.2.1.4  & Apresentação Intermediária 1sem           & Diego \\
      1.2.1.5  & Relatório Final (Junho)                   & Diego \\
      1.2.1.6  & Apresentação Final 1sem                   & Diego \\
      1.2.2    & 2o Semestre                               & Diego \\
      1.2.2.1  & Relatório Inicial 2sem                    & Diego \\
      1.2.2.2  & Relatório Intermediário 1 (Setembro)      & Diego \\
      1.2.2.3  & Relatório Intermediário 2 (Outubro)       & Diego \\
      1.2.2.4  & Apresentação Intermediária 2sem           & Diego \\
      1.2.2.5  & Relatório Final (Novembro)                & Diego \\
      1.2.2.6  & Apresentação Final 2sem                   & Diego \\
      1.3      & Codificação                               & Diego \\
      1.3.1    & Prototipação (Mocks)                      & Diego \\
      1.3.1.1  & Protótipos de cada módulo                 & Diego \\
      1.3.1.2  & Design da interface                       & Diego \\
      1.3.2    & Produto Final (MVP)                       & Diego \\
      \bottomrule
    \end{tabular}
\end{table}

%\subsection{Plano de comunicação}\label{subsec:comunicação}
O responsável pelas comunicações é o \textbf{Diego Rabatone}.\\
	\begin{table}[H]
	\caption{Plano de Comunicação}
    \begin{tabular}{lccc}
        \toprule

        \headerCell{Stakeholder} & 
        \begin{minipage}{0.24\textwidth}
          \begin{center}
            \espacoVert
            \headerCell{Informação de\\interesse}
            \espacoVert
          \end{center}
        \end{minipage} &
          \begin{minipage}{0.24\textwidth}
            \begin{center}
              \espacoVert
              \headerCell{Frequencia de\\comunicação}
              \espacoVert
            \end{center}
          \end{minipage} &
          \begin{minipage}{0.24\textwidth}
            \begin{center}
              \espacoVert
              \headerCell{Canal de\\Comunicação}
              \espacoVert
            \end{center}
          \end{minipage} \\
          
          \midrule
          
		      \textbf{Reginaldo Arakaki} &
		      \begin{minipage}{0.24\textwidth}
		        \begin{center}
		          \espacoVert
		          andamento das tarefas do wbs
		          \espacoVert
		        \end{center}
		      \end{minipage} &
		      semanal &
		      \begin{minipage}{0.24\textwidth}
		        \begin{center}
		          \espacoVert
		          presencial, email ou telefone
		          \espacoVert
		        \end{center}
		      \end{minipage} \\
		      
		      \textbf{Haydée Svab} &
		      \begin{minipage}{0.24\textwidth}
		        \begin{center}
		          \espacoVert
		          andamento das tarefas do wbs ligadas à temática de transporte
		          \espacoVert
		        \end{center}
		      \end{minipage} &
		      semanal &
		      \begin{minipage}{0.24\textwidth}
		        \begin{center}
		          \espacoVert
		          presencial, email ou telefone
		          \espacoVert
		        \end{center}
		      \end{minipage} \\
		      
		      \textbf{Prefeitura de São Paulo} &
		      \begin{minipage}{0.24\textwidth}
		        \begin{center}
		          \espacoVert
		          status de implementação do projeto
		          \espacoVert
		        \end{center}
		      \end{minipage} &
		      semestral &
		      \begin{minipage}{0.24\textwidth}
		        \begin{center}
		          \espacoVert
		          presencial
		          \espacoVert
		        \end{center}
		      \end{minipage} \\
		      
		      \textbf{SPTrans} &
		      \begin{minipage}{0.24\textwidth}
		        \begin{center}
		          \espacoVert
		          status de implementação do projeto
		          \espacoVert
		        \end{center}
		      \end{minipage} &
		      semestral &
		      \begin{minipage}{0.24\textwidth}
		        \begin{center}
		          \espacoVert
		          presencial
		          \espacoVert
		        \end{center}
		      \end{minipage} \\

		\bottomrule
	\end{tabular}
\end{table}


\chapter{Aspectos Conceituais}\label{chp:aspectosConceituais}
Os principais aspectos conceituais abordados neste projeto são:
\begin{description}
    \item[\gls{api}] \cite{apifoldoc} \hfill \\
        \gls{api} é um conjunto de rotinas e padrões estabelecidos por um \textit{software} para a utilização das suas funcionalidades por aplicativos que não pretendem considerar os detalhes da implementação do \textit{software}, mas apenas usar seus serviços.\\
        Este conceito será aplicado no \textit{backend} do projeto como interface de comunicação do \textit{app mobile} com o \textit{backend} e também como forma de disponibilizar os dados coletados pelo projeto para, por exemplo, a produção de relatórios para a \sptrans, ou disponibilizar os dados publicamente para que seja possível produzir relatórios e análises independentes.%
%
	\item[\gls{rest}] \cite{Fielding2000} \hfill \\
	    \gls{rest} é um estilo arquitetural de \textit{software} que consiste numa série planejada de restrições aplicadas a componentes, conectores e elementos de dados em sistemas de hipermídia distribuídos. \\
	    Este conceito será aplicado na implementação da \gls{api} do projeto.%
%
	\item[BigData] \hfill \\
	    Como a aplicação irá lidar com uma quantidade de usuários simultâneos da ordem de centenas de milhares, caso ganhe escala, será necessária uma infraestrutura que possa lidar com essa quantidade de informações, tanto do ponto de vista de armazenamento como do ponto de vista de transações.\\
	    Neste sentido, o maior desafio neste projeto deverá ser na aquisição da localização dos usuários, registro da mesma e, principalmente, na disponibilização dessa informação atualizada aos usuários no \textbf{módulo ``gps social''}.%
%
	\item[Gamificação] \hfill \\
	    Segundo \apudonline[p.32]{zichermann2010game}{Mastrocola2012}:
	    \begin{citacao}
	    \textit{Gamification} é fundamentalmente reescrever as regras de jogos para \textit{design} de produtos e marketing. Da rede social de geo localização \textit{FourSquare} até o \textit{social game Farmville}, e da Nike até a Marinha americana, elementos de \textit{games} como pontos, troféus, níveis, recompensas e rankings estão sendo usados em número cada vez maior.
	    \end{citacao}
	    Este conceito será aplicado na criação de um jogo que se integre ao módulo de avaliação para incentivar os usuários a avaliarem o sistema de trânsito e também com o objetivo de educar os usuários quanto às necessidades e custos na manutenção de uma frota de ônibus de passageiros.%
%
	\item[GPS Social] \hfill \\
	    O sentido de GPS Social aqui utilizado diz respeito à simultânea coleta e disponibilização de grande volume de dados de geolocalização, advindos de uma grande base de usuários, de forma que os dados sejam utilizados ``em tempo real'' pelas pessoas influenciando seus comportamentos de deslocamento. A primeira, senão a única, aplicação que foi encontrada e que faz uso intensivo desse conceito é o aplicativo \textbf{Waze}\footnote{``Waze é o maior aplicativo de navegação e trânsito do mundo baseado em uma comunidade. Junte-se aos outros motoristas em sua área e compartilhe informações de trânsito das vias em tempo real, fazendo todos economizarem tempo e combustível em seus deslocamentos diários.'' - \url{http://waze.com} - Acesso em: 01/02/2014}, que recentemente começou a adotar o termo ``Aplicativo de navegação e tráfego baseado em comunidade'' (\textit{community-based traffic and navigation app}).\\
	    Um ponto importante a ser destacado neste estilo de aplicação é que, diferente do conceito tradicional das redes sociais, nestes aplicativos, os dados de geolocalização disponibilizados não dependem de relacionamentos entre os usuários, mas unicamente da localização dos mesmos. \\
	    Uma outra leitura possível para o termo, mas que não se encaixa na aplicação proposta neste trabalho, é mais ligada ao conceito tradicional de rede social e serve para conectar usuários que estão numa mesma localidade. Exemplos de aplicativos que seguem este conceito são:%
	    %
	    \begin{itemize}
        \item Tinder\footnote{``O Tinder é um jeito divertido de se conectar com pessoas novas e interessantes próximas de você. Passe as fotos para a direita para curtir ou para a esquerda para passar. Se alguém curtir você de volta, vocês combinam! Converse com sua combinação ou tire uma foto para compartilhar um Momento com todas as suas combinações de uma vez. É uma nova maneira de se expressar e compartilhar com seus amigos.'' {\url{https://play.google.com/store/apps/details?id=com.tinder&hl=pt_BR}} - Acesso em: 01/11/2014}
%
        \item Swipe\footnote{``Anonimamente encontrara outras pessoas próximas que gostam de você. Se você também gosta delas, fazemos a ligação e então vocês podem conversar gratuitamente! Conheça novas pessoas sem ser propagado por pessoas que você não está interessado em conhecer ou conversar com elas. Este é o maior e mais livre aplicativo para conhecer novas pessoas e fazer novos amigos para paquerar, namorar, bater papo, amizade e diversão!'' {\url{https://play.google.com/store/apps/details?id=com.chirpme.swipe&hl=pt_BR}} - Acesso em: 01/11/2014}
%
        \item Brenda\footnote{``Brenda é o mais popular aplicativo de encontros para mulheres lésbicas, bissexuais ou curiosas. É rápido, fácil de usar, sem complicações e amigável às usuárias. A maioria dos recursos é totalmente gratuita, incluindo conversas ilimitadas.'' {\url{https://play.google.com/store/apps/details?id=com.benderapp.brenda&hl=pt_BR}} - Acesso em: 01/11/2014}
%
        \item Grindr\footnote{``O aplicativo de rede social exclusivamente para gays, bissexuais e curiosos mais famoso do mundo - agora mais sexy e mais rápido do que nunca. Com mais de 7 milhões de homens em 192 países, o Grindr encontra rapazes por perto. Você pode conversar e conhecer - em qualquer lugar do mundo. Encontre o rapaz perfeito agora mesmo.'' {\url{https://play.google.com/store/apps/details?id=com.grindrapp.android&hl=pt_BR}} - Acesso em: 01/11/2014}
%
	    \end{itemize}
	    Assim, o conceito de ``gps social'' será utilizado para a realização de estimativas de lotação dos pontos de ônibus e também poderá ser utilizado, futuramente, para estimar a quantidade de passageiros dentro do ônibus.%
%
	\item[Percepção do usuário/consumidor]\cite{Lai1995,Almeida2011,Almeida2007,andrade2008constructos} \hfill \\
	    Como um dos pontos fundamentais propostos para a aplicação é permitir ao usuário avaliar o serviço de transporte, em especial baseado em critérios qualitativos que não são facilmente mensuráveis. Definiu-se então que será realizada uma avaliação geral e cinco avaliações específicas, considerados critérios importantes na tomada de decisão de um usuário de ônibus, a saber:
	    \begin{itemize}
	    \item Lotação
	    \item Conforto Térmico
	    \item Higiene/Limpeza
	    \item Atendimento do Motorista
	    \item Atendimento do Cobrador
	    \end{itemize}
	    Optou-se por cinco questões específicas por ser um número não muito grande com o objetivo de não desincentivar o usuário a responder. Este número, porém, pode variar, mas recomenda-se que não passe de 15 \cite{teller2013}. As cinco questões serão colocadas numa única página, de maneira que, de início, o usuário já tenha conhecimento da quantidade de respostas que terá que dar.\\
	    Além disso, as avaliações serão realizadas numa escala contínua. Dessa forma será possível extrair um resultado que permitirá a elaboração de estatísticas descritivas para além de simples frequências. Caso se tivesse optado por utilizar uma escala Likert seria possível apenas realizar análises das frequências das respostas, restringindo as análises possíveis e dificultando a geração de índices e indicadores.\cite{favero2009}%
%
    \item[Desenvolvimento Móvel Híbrido] \hfill \\
        Ou também conhecido como \textit{Mobile Hybrid}, é definido por \citeonline[p.6]{Leadership2012} como:
        \begin{citacao}
            \textit{The hybrid approach combines native development with web technology. Using this approach, developers write significant portions of their application in cross-platform web technologies, while maintaining direct access to native APIs when required.}\\
        \textit{The native portion of the application uses the operating system APIs to create an embedded HTML rendering engine that serves as a bridge between the browser and the device APIs. This bridge enables the hybrid app to take full advantage of all the features that modern devices have to offer.}
        \end{citacao}
        Este será o conceito utilizado para o desenvolvimento do aplicativo mobile do projeto. As funcionalidades nativas que serão utilizadas serão:%
            \begin{enumerate*}[label=\itshape\alph*\upshape)]
                \item geolocalização (gps);
                \item câmera; e
                \item execução da aplicação em segundo plano.
            \end{enumerate*}%
%
    \item[\gls{mvt}] \hfill \\
        É a \gls{dp} utilizada pelo \textit{framework} \gls{django}. Segundo \citeonline{djangobook}, o \textit{framework} segue boa parte do padrão \gls{mvc}, proposto por \citeonline{gamma95}.\\
        O padrão \gls{mvc}, no \gls{django}, é implementado da seguinte maneira:
        \begin{itemize}
	        \item[M](``\textit{Model}''): camada responsável pelo acesso aos dados. É gerenciada pelo módulo de base de dados do \gls{django};
	        \item[V](``\textit{View}''): camada responsável pela seleção de quais dados serão apresentados e como eles serão apresentados. É gerenciada pelos módulos \textit{views} e \textit{template}; e
	        \item[C](``\textit{Controller}''): camada que direciona uma requisição a uma \textit{View}, baseado nas regras definidas no arquivo de configuração de endereços (\textit{urls}), chamando a função apropriada para tratar aquela requisição.
        \end{itemize}
        Como a camada ``C'' é gerenciada pelo próprio \textit{framework} e a maior parte das ações ocorrem nos módulos ``\textit{model}'', ``\textit{view}'' e ``\textit{template}'', defini-se que o \gls{django} como um \textit{framework} \gls{mvt}, no qual:
        \begin{itemize}
	        \item[M](``\textit{Model}''): camada de acesso aos dados. Esta camada contém tudo sobre os dados do projeto, como:
            \begin{enumerate*}[label=\itshape\alph*\upshape)]
                \item Como acessar os dados;
                \item Como validar os dados;
                \item Qual o comportamento do dado; e
                \item O relacionamento entre os dados.
            \end{enumerate*};
            \item[V](``\textit{View}''): camada de lógica de negócio. Ela contém toda lógica de acesso à camada de dados e direciona para o template adequado. Pode-se pensá-la como a ponte entre a camada de dados e a de templates; e
            \item[T](``\textit{Template}''): camada de apresentação. Esta camada contém decisões relativas à apresentação como, por exemplo, a maneira pela qual algo deve ser mostrado em uma página web ou outro tipo de documento.
        \end{itemize}
\end{description}
\chapter{Especificação}\label{chp:Especificação}
O projeto foi desenvolvido para dispositivos móveis utilizando \textit{tecnologia de desenvolvimento híbrida}, visando facilitar o desenvolvimento e o deploy em diversas plataformas móveis. Neste capítulo descreveremos a especificação técnica do \textit{backend} e do aplicativo mobile do projeto.

\section{Aplicações do \textit{Backend}}\label{sec:spec-backend}
Como solução de \textit{backend} foi utilizado o \textit{framework} \gls{django}, que utiliza a linguagem de programação \textbf{Python} e é organizado segundo a \textit{design pattern} \gls{mvt}, conforme pode ser visto nas Figuras \ref{fig:arqMVT} e \ref{fig:arqDjango}.%
%
\diagramaRetrato{arquitetura-app-django.png}{0.8}{Arquitetura MVT de um APP do framework Django}{arqMVT}{Jeff Croft em {\footnotesize\url{http://www.flickr.com/photos/jcroft/432038560/sizes/o/in/photostream/}}}{}%
%
\diagramaRetrato{django-arq2.eps}{1.1}{Arquitetura do framework Django}{arqDjango}{Autoria própria}{baseado em {\footnotesize\url{http://www.slideshare.net/AbhijeetShekhar1/django-39439148}}}

A estrutura de projeto do \textit{framework} é pensada de forma modular, na qual a aplicação é composta por ''\textit{apps}'' independentes que realizam funções específicas e são conectados no projeto, conforme pode ser visto na Figura \ref{fig:multiApps}.%
%
\diagramaRetrato{django-multi-apps.jpg}{0.45}{Arquitetura multi-aplicativos do Django}{multiApps}{Ian Ward em {\footnotesize\url{http://excess.org/article/2007/06/oclug-django-site/}}}{}

Considerando tal arquitetura, foram desenvolvidos os 3 módulos principais já descritos anteriormente, na Seção \ref{sec:Escopo} (\textbf{Avaliação}, \textbf{Game} e \textbf{GPS social}). Além destes, também foi necessária a criação de um módulo adicional (\textit{utils}) com a finalidade de suprir algumas integrações entre os \textit{apps} sem impactar no isolamento entre eles. A função principal deste módulo foi a de criar a \gls{api} \gls{rest} do projeto

A \gls{api} \gls{rest} foi implementada utilizando-se o pacote \gls{drf}\footnote{\url{http://www.django-rest-framework.org}}. Com o uso do \gls{drf} foi necessário desenvolver algumas classes serializadoras vinculadas aos modelos de dados que seriam expostos (\textit{models}), em seguida criar as respectivas \textit{views} e criar um \textit{Router} com as urls que seriam expostas. Sequer foi necessária a criação de templates, visto que o pacote já fornece um template padrão. O módulo expõe os dados no formato \gls{json}, para além da apresentação na interface web, sem que seja necessário qualquer desenvolvimento ou configuração.

Foi utilizado também o pacote \gls{psa} para permitir a autenticação dos usuários com seus logins de redes sociais. Neste primeiro momento foram disponibilizados os logins via \textit{Facebook}\footnote{\url{https://developers.facebook.com/docs/facebook-login/v2.2}} e via \textit{Google Social Login}\footnote{\url{https://developers.google.com/+/web/signin/}}, ambos utilizando o protocolo de autenticação OAuth 2.0\footnote{\url{http://oauth.net/2/}}. Neste primeiro momento não foi utilizada a rede social \textit{Twitter} pois a mesma não disponibiliza o email do usuário ao realizar o login, o que impede que possamos vincular a conta da rede social com usuários já cadastrados.

Como o aplicativo lida com dados georreferenciados, para \gls{sgbd} foi escolhido o \textbf{PostgreSQL} com a extensão \textbf{PostGis}, visto que esta é a solução mais amplamente utilizada no mercado e com melhor suporte, além de ser a recomendada pelos desenvolvedores do \gls{django}, de sua extensão ``geo'' e da biblioteca python (\textit{gdal (Geospatial Data Abstraction Library)}). A solução escolhida permite inclusive realizar consultas utilizando critérios de geolocalização como, por exemplo, ``Selecionar todos os registros cuja localização se encontra num raio de X metros do ponto Y'', o que é fundamental para a criação do módulo \textbf{mapa}.

O framework \gls{django} por si só não é um webserver - ele apenas possui um microserver para fins de teste e desenvolvimento; então faz-se necessário utilizar um webserver. No presente projeto optou-se pelo \gls{nginx}, o servidor web que tem crescido no ritmo mais acelerado dos que possuem pelo menos 1\% do mercado. Seu ritmo anual médio de crescimento de 2010 a 2014 43\%, e hoje ele já ocupa a segunda colocação como servidor mais utilizado na internet com 22,6\% do mercado, atrás apenas do Apache, que tem 59\% do mercado mas que tem perdido, em média, 3,71\% de \textit{market share} ao ano, conforme pode ser observado na Figura \ref{fig:nginxmarkershare}.

\diagramaRetrato{market_share_webservers.png}{0.55}{\textit{Market Share} de servidores web}{nginxmarkershare}{W3Techs - Web Technology Surveys {\footnotesize\url{http://w3techs.com/technologies/history_overview/web_server/ms/y}}}{}

Além dessa grande exposição, e de possuir uma boa documentação\footnote{\url{http://nginx.org/en/docs/}}, outra vantagem é que ele consegue cumprir a função de webserver, proxy reverso e também possui recursos de \textit{load balance}, tornado-o uma ótima alternativa em termos de escalabilidade.

O \gls{nginx} por si só não consegue "fornecer" diretamente a aplicação django, para tanto é preciso ainda mais um elemento, que é o servidor de aplicações. Para tanto, um dos que apresenta melhor desempenho nos dias de hoje para servir projetos que utilizam a linguagem de programação Python é o \gls{uwsgi}\footnote{\url{https://ivan-site.com/2012/09/benchmark-uwsgi-vs-gunicorn-for-async-workers/}}$^,$\footnote{\url{http://blog.kgriffs.com/2012/12/18/uwsgi-vs-gunicorn-vs-node-benchmarks.html}}$^,$\footnote{\url{http://www.peterbe.com/plog/fcgi-vs-gunicorn-vs-uwsgi}}. Dessa maneira, optamos pelo \gls{uwsgi} como servidor de aplicação trabalhando em conjunto com o \gls{nginx} como servidor web.

Fica como sugestão para o futuro do projeto a utilização do servidor de cache \textit{varnish}\footnote{\url{https://www.varnish-cache.org/}}, conforme recomendado pela equipe do hosting DigitalOcean\footnote{\url{https://www.digitalocean.com/community/tutorials/how-to-scale-django-beyond-the-basics}}, para conseguir escalar o projeto sem precisar necessariamente de mais recursos de máquina.

\section{Estrutura implementada da Aplicação}\label{sec:estrutura-app}
Nesta seção iremos descrever a implementação realizada de cada um dos \textit{apps}, assim como do projeto \gls{django} que integra os \textit{apps}. Por motivos organizacionais optou-se por descrever cada uma das camadas do \textit{design pattern} para todos os aplicativos simultâneamente ao invés de descrever cada aplicativo individualmente.

\subsection{Camada de Acesso a Dados (\textit{model})}\label{subsec:camada-model}
Inicialmente foi planejada uma camada de modelo de dados conforme o Diagrama Entidade-Relacionamento exposto na Figura \ref{fig:DiagER}.
\section{Diagrama Entidade Relacionamento}
    \diagramaRetrato{diagramas_er_bds.eps}{1.2}{Diagrama Entidade Relacionamento}{DiagER}{Autoria Própria}{}

\input{./texto/especificacoes/diagrama-de-classes}


\section{Aplicativo Mobile}\label{sec:spec-appmobile}

Já para o desenvolvimento Para o \textit{frontend} do projeto será utilizado \textbf{HTML5}, \textbf{CSS3} e \textbf{JavaScript}, além de algumas bibliotecas auxiliares como \textbf{Twitter Bootstrap} (versão 3).




\section{Requisitos Funcionais}\label{sec: RF}
%Requisitos Funcionais, n\~ao funcionais, regras de neg\'ocio e restri\c{c}\~oes encontram-se no anexo \ref{Anexo Requisitos}
%%\temporario{Descreve os requisitos no n\'{\i}vel que permita desenvolver o software que satisfa\c{c}a os requisitos do produto e os testes que mostrem o atendimento destes requisitos. O conte\'udo das se\c{c}\~oes podem ser desenvolvidos como documentos independentes.\\
%%Neste documento, esta se\c{c}\~ao est\'a organizada para atender \'as necessidades da An\'alise Orientada a Objetos.}

%%02
%\vfill
%\begin{Requisito}
%    \ReqTipo{funcional}%Tipo
%    \ReqNome{Remover Material}%Nome
%%%    \ReqLabel{ReqRemMat}%Label
%    \ReqDescr{O sistema recebe a mensagem do usu\'ario solicitando a remo\c{c}\~ao de determinado(s) item(ns) do banco de dados do sistema.}%Desc.
%    \ReqPrioridade{media}%Prioridade
%    \ReqStatus{proposto}%Status
%    \ReqEstabilidade{media}%Estabilidade
%    \ReqOrigem{cliente}%Origem
%    \ReqRationale{Esta funcionalidade permite a remo\c{c}\~ao de determinado(s) item(ns) do banco de dados pelo usu\'ario. É utilizada em caso de descontinuidade de um produto, quando a loja para de comercializar algum produto, entre outros.}%Rationale
%    \ReqAssoc{Banco de dados estruturado}%Assoc
%\end{Requisito}

%%01
\vfill
\begin{Requisito}
    \ReqTipo{funcional}%Tipo
    \ReqNome{Cadastrar usu\'ario}%Nome
    %\ReqLabel{cadUsu}%Label
    \ReqDescr{O usu\'ario realiza seu cadastro na plataforma.}%Desc.
    \ReqPrioridade{alta}%Prioridade
    \ReqStatus{aprovado}%Status
    \ReqEstabilidade{alta}%Estabilidade
    \ReqOrigem{interna}%Origem
    \ReqRationale{Permite realizar um ``tracking'' do hist\'orico dos usu\'arios para oferecer informa\c{c}\~oes mais precisas, al\'em de permitir que o usu\'ario utilize o ``game'', que demanda um ac\'umulo de participa\c{c}\~ao na plataforma.}%Rationale
    \ReqAssoc{}%Assoc
\end{Requisito}

%%02
\vfill
\begin{Requisito}
    \ReqTipo{funcional}%Tipo
    \ReqNome{Realizar login}%Nome
    %\ReqLabel{login}%Label
    \ReqDescr{O usu\'ario deve fazer login para ter acesso ao sistema, impedindo que pessoas n\~ao autorizadas tenham acesso a certas fun\c{c}\~oes do sistema.}%Desc.
    \ReqPrioridade{alta}%Prioridade
    \ReqStatus{aprovado}%Status
    \ReqEstabilidade{alta}%Estabilidade
    \ReqOrigem{usuario}%Origem
    \ReqRationale{M\'etodo que previne que pessoas n\~ao autorizadas tenham acesso \'as funcionalidades do sistema, em especial as que alteram o banco de dados.}%Rationale
    \ReqAssoc{Realizar login social}%Assoc
\end{Requisito}

%%03
\vfill
\begin{Requisito}
    \ReqTipo{funcional}%Tipo
    \ReqNome{Realizar login social}%Nome
    %\ReqLabel{loginsocial}%Label
    \ReqDescr{Usu\'ario utiliza seu login de redes sociais para se conectar \`a plataforma}%Desc.
    \ReqPrioridade{alta}%Prioridade
    \ReqStatus{aprovado}%Status
    \ReqEstabilidade{alta}%Estabilidade
    \ReqOrigem{usuario}%Origem
    \ReqRationale{Para aumentar a facilidade de acesso \`a plataforma \'e fundamental permitir que os usu\'arios acessem o sistema utilizando sistemas de autentica\c{c}\~ao cruzada com redes sociais, principalmente \textit{Facebook}, \textit{Twitter} e \textit{Google}.}%Rationale
    \ReqAssoc{Realizar Login}%Assoc
\end{Requisito}

%%04
\vfill
\begin{Requisito}
    \ReqTipo{funcional}%Tipo
    \ReqNome{Realizar logout}%Nome
    %\ReqLabel{loginsocial}%Label
    \ReqDescr{O usu\'ario deve fazer logout ao terminar de utilizar o sistema, impedindo que pessoas
n\~ao autorizadas tenham acesso a certas fun\c{c}\~oes do sistema.}%Desc.
    \ReqPrioridade{alta}%Prioridade
    \ReqStatus{aprovado}%Status
    \ReqEstabilidade{alta}%Estabilidade
    \ReqOrigem{usuario}%Origem
    \ReqRationale{M\'etodo que previne que pessoas n\~ao autorizadas tenham acesso \'as funcionalidades
do sistema, em especial as que alteram o banco de dados.}%Rationale
    \ReqAssoc{Realizar Login}%Assoc
\end{Requisito}

%%05
\vfill
\begin{Requisito}
    \ReqTipo{funcional}%Tipo
    \ReqNome{Listar linhas de \^onibus mais prov\'aveis}%Nome
    %\ReqLabel{loginsocial}%Label
    \ReqDescr{O sistema deve listar as linhas de \^onibus mais prov\'aveis \`a escolha do usu\'ario, de acordo com sua localiza\c{c}\~ao.}%Desc.
    \ReqPrioridade{alta}%Prioridade
    \ReqStatus{aprovado}%Status
    \ReqEstabilidade{alta}%Estabilidade
    \ReqOrigem{usuario}%Origem
    \ReqRationale{M\'etodo que reduz o tempo necess\'ario para que o usu\'ario escolha o \^onibus, tanto para que possa realizar uma avalia\c{c}\~ao quanto para que possa obter informa\c{c}\~oes sobre a linha.}%Rationale
    \ReqAssoc{Coletar posicionamento do usu\'ario}%Assoc
\end{Requisito}

%%06
\vfill
\begin{Requisito}
    \ReqTipo{funcional}%Tipo
    \ReqNome{Selecionar linha de \^onibus}%Nome
    %\ReqLabel{cadUsu}%Label
    \ReqDescr{O usu\'ario seleciona a linha de \^onibus que deseja avaliar.}%Desc.
    \ReqPrioridade{alta}%Prioridade
    \ReqStatus{aprovado}%Status
    \ReqEstabilidade{alta}%Estabilidade
    \ReqOrigem{interna}%Origem
    \ReqRationale{O usu\'ario precisa conseguir selecionar a linha de \^onibus para poder avali\'a-la.}%Rationale
    \ReqAssoc{Listas linhas de \^onibus mais prov\'aveis.}%Assoc
\end{Requisito}

%%07
\vfill
\begin{Requisito}
    \ReqTipo{funcional}%Tipo
    \ReqNome{Coletar posicionamento do usu\'ario}%Nome
    %\ReqLabel{cadUsu}%Label
    \ReqDescr{O ``sistema'' deve coletar a informa\c{c}\~ao de geolocaliza\c{c}\~ao do usu\'ario, usando GPS e outros m\'etodos como Triangula\c{c}\~ao via 3G [usar API de geolocaliza\c{c}\~ao do HTML5].}%Desc.
    \ReqPrioridade{alta}%Prioridade
    \ReqStatus{aprovado}%Status
    \ReqEstabilidade{alta}%Estabilidade
    \ReqOrigem{interna}%Origem
    \ReqRationale{Para que o sistema possa exibir a lista de poss\'{\i}veis linhas de \^onibus que o usu\'ario deve estar, incluindo o sentido da linha, \'e fundamental conhecer a localiza\c{c}\~ao do usu\'ario e seu sentido de deslocamento.}%Rationale
    \ReqAssoc{Listas linhas de \^onibus mais prov\'aveis.}%Assoc
\end{Requisito}

%%08
\vfill
\begin{Requisito}
    \ReqTipo{funcional}%Tipo
    \ReqNome{Realizar avalia\c{c}\~ao global}%Nome
    %\ReqLabel{cadUsu}%Label
    \ReqDescr{O usu\'ario avalia globalmente o servi\c{c}o de transporte p\'ublico com uma nota numa escala cont\'{\i}nua, utilizando um \textit{slider}.}%Desc.
    \ReqPrioridade{alta}%Prioridade
    \ReqStatus{aprovado}%Status
    \ReqEstabilidade{alta}%Estabilidade
    \ReqOrigem{interna}%Origem
    \ReqRationale{É importante uma avalia\c{c}\~ao global, simples e direta, para o caso do usu\'ario n\~ao querer preencher todas as avalia\c{c}\~oes espec\'{\i}ficas. Al\'em disso, o uso da escala cont\'{\i}nua \'e importante pois far\'a com que a avalia\c{c}\~ao seja uma vari\'avel quantitativa, o que permitir\'a exprimir estat\'{\i}sticas simples como m\'edia e desvio padr\~ao.}%Rationale
    \ReqAssoc{}%Assoc
\end{Requisito}

%%09
\vfill
\begin{Requisito}
    \ReqTipo{funcional}%Tipo
    \ReqNome{Realizar avalia\c{c}\~oes espec\'{\i}ficas}%Nome
    %\ReqLabel{cadUsu}%Label
    \ReqDescr{O usu\'ario avalia cada um dos crit\'erios espec\'{\i}ficos definidos como uma nota numa escala cont\'{\i}nua para cada um, utilizando um \textit{slider}.}%Desc.
    \ReqPrioridade{alta}%Prioridade
    \ReqStatus{aprovado}%Status
    \ReqEstabilidade{alta}%Estabilidade
    \ReqOrigem{interna}%Origem
    \ReqRationale{Os crit\'erios escolhidos ser\~ao de car\'ater subjetivo, crit\'erios estes que n\~ao seriam mensur\'aveis utilizando-se tecnologias de automa\c{c}\~ao. Al\'em disso, o uso da escala cont\'{\i}nua \'e importante pois far\'a com que a avalia\c{c}\~ao seja uma vari\'avel quantitativa, o que permitir\'a exprimir estat\'{\i}sticas simples como m\'edia e desvio padr\~ao.}%Rationale
    \ReqAssoc{}%Assoc
\end{Requisito}

%%10
\vfill
\begin{Requisito}
    \ReqTipo{funcional}%Tipo
    \ReqNome{Adicionar reclama\c{c}\~ao}%Nome
    %\ReqLabel{cadUsu}%Label
    \ReqDescr{Para cada "nota" negativa que o usu\'ario d\'a ser\'a mostrado a ele uma caixa de texto para preenchimento de uma reclama\c{c}\~ao [opcional].}%Desc.
    \ReqPrioridade{alta}%Prioridade
    \ReqStatus{aprovado}%Status
    \ReqEstabilidade{alta}%Estabilidade
    \ReqOrigem{interna}%Origem
    \ReqRationale{Ser\'a permitido ao usu\'ario adicionar uma descri\c{c}\~ao do motivo pelo qual ele deu uma nota negativa para aquele determinado crit\'erio, permitindo assim identificar motivos de avalia\c{c}\~oes negativas para cada crit\'erio, o que ajuda na melhoria do servi\c{c}o prestado.}%Rationale
    \ReqAssoc{}%Assoc
\end{Requisito}

%%11
\vfill
\begin{Requisito}
    \ReqTipo{funcional}%Tipo
    \ReqNome{Enviar reclama\c{c}\~ao}%Nome
    %\ReqLabel{cadUsu}%Label
    \ReqDescr{Cada reclama\c{c}\~ao escrita dever\'a estar associada a um endere\c{c}o \'unico a ser enviado, via twitter, \`a \sptrans.}%Desc.
    \ReqPrioridade{alta}%Prioridade
    \ReqStatus{aprovado}%Status
    \ReqEstabilidade{alta}%Estabilidade
    \ReqOrigem{interna}%Origem
    \ReqRationale{É fundamental que o processo de avalia\c{c}\~ao e \textit{accountability} seja o mais p\'ublicos e transparentes poss\'{\i}veis. Dessa forma, conforme apresentado anteriormente \`a \sptrans, as reclama\c{c}\~oes ser\~ao postadas publicamente, com a utiliza\c{c}\~ao de algumas \textit{hashtags} espec\'{\i}ficas pr\'e-definidas para serem respondidas pelo \'org\~ao p\'ublicamente.}%Rationale
    \ReqAssoc{}%Assoc
\end{Requisito}

\clearpage
\subsection{Tabela Resumo dos Requisitos}\label{subsec:tabResReq}
   \begin{table}[H]
        \centering
        \caption{Requisitos funcionais}
        \label{tab:reqFunc}
%        \PrintRequisitos
%        \begin{tabular}{|c|l|}
%            \hline
%            \textbf{Identifica\c{c}\~ao} & \textbf{Requisito}\\
%            \hline
%            \hyperlink{RF1}{RF1} & Cadastrar usu\'ario\\
%            \hline
%            \hyperlink{RF2}{RF2} & Realizar login\\
%            \hline
%            \hyperlink{RF3}{RF3} & Realizar login social\\
%            \hline
%            \hyperlink{RF4}{RF4} & Realizar logout\\
%            \hline
%         \end{tabular}
    \end{table}
%
%    \begin{table}[H]
%        \centering
%        \caption{Requisitos funcionais do m\'odulo de avalia\c{c}\~ao}
%        \label{tab:reqFuncAva}
%        \begin{tabular}{|c|l|}
%            \hline
%            \textbf{Identifica\c{c}\~ao} & \textbf{Requisito} \\
%            \hline
%            \hyperlink{RF5}{RF5} & Listar linhas de \^onibus mais prov\'aveis\\
%            \hline
%            \hyperlink{RF6}{RF6} & Selecionar linha de \^onibus\\
%            \hline
%            \hyperlink{RF7}{RF7} & Coletar posicionamento do usu\'ario\\
%            \hline
%            \hyperlink{RF8}{RF8} & Realizar avalia\c{c}\~ao global\\
%            \hline
%            \hyperlink{RF9}{RF9} & Realizar avalia\c{c}\~oes espec\'{\i}ficas\\
%            \hline
%            \hyperlink{RF10}{RF10} & Adicionar reclama\c{c}\~ao\\
%            \hline
%            \hyperlink{RF11}{RF11} & Enviar reclama\c{c}\~ao\\
%            \hline
%         \end{tabular}
%    \end{table}
%
%    \begin{table}[H]
%        \centering
%        \caption{Requisitos funcionais do m\'odulo de informa\c{c}\~oes}
%        \label{tab:reqFuncInfo}
%        \begin{tabular}{|c|l|}
%            \hline
%            \textbf{Identifica\c{c}\~ao} & \textbf{Requisito} \\
%            \hline
%            \hyperlink{RF1}{RF1} & Cadastrar Usu\'ario \\
%            \hline
%            \hyperlink{RF2}{RF2} & Executar Login Social\\
%            \hline
%            \hyperlink{RF3}{RF3} & Nome 3\\
%            \hline
%         \end{tabular}
%    \end{table}
%
%    \begin{table}[H]
%        \centering
%        \caption{Requisitos funcionais do m\'odulo de game}
%        \label{tab:reqFuncGame}
%        \begin{tabular}{|c|l|}
%            \hline
%            \textbf{Identifica\c{c}\~ao} & \textbf{Requisito} \\
%            \hline
%            \hyperlink{RF4}{RF4} & Nome 4\\
%            \hline
%            \hyperlink{RF5}{RF5} & Nome 5\\
%            \hline
%            \hyperlink{RF6}{RF6} & Nome 6\\
%            \hline
%         \end{tabular}
%    \end{table}
%

\section{Requisitos Não Funcionais}\label{sec: RNF}
\StartReqNFunc
%\vfill
%\begin{Requisito}
%    \ReqTipo{nao funcional}%Tipo
%    \ReqNome{O sistema deverá exigir autenticação para acesso do usuário}%Nome
%%%%    \ReqLabel{Req}%Label
%    \ReqDescr{Para qualquer funcionalidade do sistema, este deve verificar se o usuário possui privilégios para acessar esta função.}%Desc
%    \ReqPrioridade{alta}%Prioridade
%    \ReqStatus{proposto}%Status
%    \ReqEstabilidade{media}%Estabilidade
%    \ReqOrigem{cliente}%Origem
%    \ReqRationale{Necessário para garantir que os dados do negócio do cliente estejam seguros no sistema}%Rationale
%    \ReqAssoc{Funcionalidades do sistema estruturadas}%Assoc
%\end{Requisito}
%
\vfill
%
\begin{Requisito}
    \ReqNome{Autentica\unexpanded{\c{c}}\unexpanded{\~a}o Obrigat\unexpanded{\'o}ria}%Nome
    \ReqTipo{nao funcional}%Tipo
%%%    \ReqLabel{Req}%Label
    \ReqDescr{Para qualquer funcionalidade do sistema, este deve verificar se o usuário está autenticado.}%Desc
    \ReqPrioridade{alta}%Prioridade
    \ReqStatus{aceito}%Status
    \ReqEstabilidade{alta}%Estabilidade
    \ReqOrigem{interna}%Origem
    \ReqRationale{Até este momento todos os serviços do projeto requerem autenticação dos usuários.}%Rationale
    \ReqAssoc{}%Assoc
\end{Requisito}
%
\vfill
%
\begin{Requisito}
    \ReqNome{Garantir acesso apenas a pessoas autorizadas}%Nome
    \ReqTipo{nao funcional}%Tipo
%%%    \ReqLabel{Req}%Label
    \ReqDescr{Para qualquer funcionalidade do sistema, este deve verificar se o usuário possui privilégios para acessar esta função.}%Desc
    \ReqPrioridade{alta}%Prioridade
    \ReqStatus{aceito}%Status
    \ReqEstabilidade{alta}%Estabilidade
    \ReqOrigem{interna}%Origem
    \ReqRationale{Necessário para garantir que nenhum dado ou função seja acessada indevidamente, além de garantir que toda ação dos usuários seja corretamente registrada.}%Rationale
    \ReqAssoc{}%Assoc
\end{Requisito}
%
\vfill
%
\begin{Requisito}
    \ReqNome{Senhas criptografadas na base}%Nome
    \ReqTipo{nao funcional}%Tipo
%%%    \ReqLabel{Req}%Label
    \ReqDescr{Senhas gravadas no banco de dados devem estar criptografadas, nunca em texto plano.}%Desc
    \ReqPrioridade{alta}%Prioridade
    \ReqStatus{aceito}%Status
    \ReqEstabilidade{alta}%Estabilidade
    \ReqOrigem{interna}%Origem
    \ReqRationale{Gravar senhas no banco de dados em texto plano é uma falha grave de segurança em qualquer sistema.}%Rationale
    \ReqAssoc{}%Assoc
\end{Requisito}
%
\vfill
%
\begin{Requisito}
    \ReqNome{Qualidade da senha}%Nome
    \ReqTipo{nao funcional}%Tipo
%%%    \ReqLabel{Req}%Label
    \ReqDescr{Todas as senhas definidas no sistema devem ter ao menos 6 dígitos, contendo ao menos um número, uma letra e um caracter não alfanumérico (.-\_\@\#! dentre outros).}%Desc
    \ReqPrioridade{alta}%Prioridade
    \ReqStatus{aceito}%Status
    \ReqEstabilidade{alta}%Estabilidade
    \ReqOrigem{interna}%Origem
    \ReqRationale{É fundamental que as senhas tenham um nível mínimo de segurança para impedir ataques de ``força bruta''.}%Rationale
    \ReqAssoc{}%Assoc
\end{Requisito}
%
\vfill
%
\begin{Requisito}
    \ReqNome{Qualidade das senhas administrativas}%Nome
    \ReqTipo{nao funcional}%Tipo
%%%    \ReqLabel{Req}%Label
    \ReqDescr{Todas as senhas administrativas definidas no sistema devem ter ao menos 12 dígitos, contendo ao menos um número, uma letra e um caracter não alfanuméricos (.-\_\@\#! dentre outros).}%Desc
    \ReqPrioridade{alta}%Prioridade
    \ReqStatus{aceito}%Status
    \ReqEstabilidade{alta}%Estabilidade
    \ReqOrigem{interna}%Origem
    \ReqRationale{É fundamental que as senhas tenham um nível mínimo de segurança para impedir ataques de ``força bruta'', em especial as com acesso privilegiado.}%Rationale
    \ReqAssoc{}%Assoc
\end{Requisito}
%
\vfill
\clearpage
\vfill
%
\begin{Requisito}
    \ReqNome{Realizar o log das a\unexpanded{\c{c}}\unexpanded{\~o}es dos usu\unexpanded{\'a}rios}%Nome
    \ReqTipo{nao funcional}%Tipo
%%%    \ReqLabel{Req}%Label
    \ReqDescr{Realizar o log (``tracking'') de todas as ações de todos os usuários, registrando-as na base de dados.}%Desc
    \ReqPrioridade{alta}%Prioridade
    \ReqStatus{aceito}%Status
    \ReqEstabilidade{alta}%Estabilidade
    \ReqOrigem{interna}%Origem
    \ReqRationale{Essas informações servirão para posterior análise e melhoria de usabilidade da aplicação.}%Rationale
    \ReqAssoc{}%Assoc
\end{Requisito}
%
\vfill
%
\begin{Requisito}
    \ReqNome{Realizar registro do timing das a\unexpanded{\c{c}}\unexpanded{\~o}es dos usu\unexpanded{\'a}rios}%Nome
    \ReqTipo{nao funcional}%Tipo
%%%    \ReqLabel{Req}%Label
    \ReqDescr{Registrar no log o quanto tempo cada usuário demorou para realizar cada ação, ou quanto dele ele ficou em cada página da aplicação.}%Desc
    \ReqPrioridade{alta}%Prioridade
    \ReqStatus{aceito}%Status
    \ReqEstabilidade{alta}%Estabilidade
    \ReqOrigem{interna}%Origem
    \ReqRationale{Essas informações servirão para posterior análise e melhoria de usabilidade da aplicação.}%Rationale
    \ReqAssoc{}%Assoc
\end{Requisito}
%
\vfill
%
\begin{Requisito}
    \ReqNome{N\unexpanded{\'u}mero m\unexpanded{\'a}ximo de avalia\unexpanded{\c{c}}\unexpanded{\~o}es espec\unexpanded{\'i}ficas}%Nome
    \ReqTipo{nao funcional}%Tipo
%%%    \ReqLabel{Req}%Label
    \ReqDescr{Apresentar no máximo 5 avaliações específicas aos usuários.}%Desc
    \ReqPrioridade{alta}%Prioridade
    \ReqStatus{aceito}%Status
    \ReqEstabilidade{alta}%Estabilidade
    \ReqOrigem{interna}%Origem
    \ReqRationale{Mais do que 5 questões faz com que o questionário fique muito extenso e demorado para ser respondido num dispositivo móvel, dentro de um ônibus em movimento, reduzindo o número de respostas.}%Rationale
    \ReqAssoc{}%Assoc
\end{Requisito}
%
\vfill
%
\begin{Requisito}
    \ReqNome{Reclama\unexpanded{\c{c}}\unexpanded{\~o}es apenas ap\unexpanded{\'o}s salvar avalia\unexpanded{\c{c}}\unexpanded{\~o}es}%Nome
    \ReqTipo{nao funcional}%Tipo
%%%    \ReqLabel{Req}%Label
    \ReqDescr{As telas de reclamação devem aparecer após o usuário salvar suas avaliações.}%Desc
    \ReqPrioridade{alta}%Prioridade
    \ReqStatus{aceito}%Status
    \ReqEstabilidade{alta}%Estabilidade
    \ReqOrigem{interna}%Origem
    \ReqRationale{As avaliações, realizadas com ``\textit{slidebar}'' são muito rápidas. Por isso, permitir que os usuários realizem todas essas avaliações antes de abrir as caixas de ''reclamação'' para que eles digitem as reclamações é fundamental.}%Rationale
    \ReqAssoc{}%Assoc
\end{Requisito}
%
\vfill
%
%\section{Tabela de Casos de Uso}\label{sec:TabCasosDeUso}
%
%%%%%%%%%%%%%%%%%%%MODELO A SER USADO%%%%%%%%%%%%%%%%
%\begin{casodeuso}
%\nomeCdU{} %Nome do Caso de Uso
%\descricaoCdU{} %Descrição
%\eventoiniciadorCdU{} %Evento iniciador
%\atoresCdU{} %Atores
%%Lista de pré-condições
%\precondicaoCdU{} %Pré-condição 1
%\precondicaoCdU{} %Pré-condição 2
%\precondicaoCdU{} %Pré-condição 3
%%Lista de eventos do caso de uso
%\eventosCdU{} %primeiro evento
%\eventosCdU{} %segundo evento
%\eventosCdU{} %terceiro evento
%Lista de pós-condições
%\poscondicaoCdU{} %Pós-condição 1
%\poscondicaoCdU{} %Pós-condição 2
%\poscondicaoCdU{} %Pós-condição 3
%%Lista de casos de uso de extensão
%\extensaoCdU{} %Caso de extensão 1
%\extensaoCdU{} %Caso de extensão 2
%\extensaoCdU{} %Caso de extensão 3
%%Lista de casos de uso de inclusão
%\inclusaoCdU{} %Caso de inclusão 1
%\inclusaoCdU{} %Caso de inclusão 2
%\inclusaoCdU{} %Caso de inclusão 3
%%Lista de requisitos relacionados/atendidos
%\requisitoCdU{} %Requisitos relacionados/atendidos
%\requisitoCdU{} %Requisitos relacionados/atendidos
%\requisitoCdU{} %Requisitos relacionados/atendidos
%\end{casodeuso}
%%%%%%%%%%%%%%%%%%%MODELO A SER USADO%%%%%%%%%%%%%%%%
%
%
%%%%%%%%%%%%%%%%%%%%%%%%%%%%%%%%%%%
%\begin{casodeuso}
%\nomeCdU{Cadastrar Usuário} %Nome do Caso de Uso
%\labelCdU{cad-usu} %Label do Caso de Uso
%\descricaoCdU{Cadastro de um novo usuário no sistema} %Descrição
%\eventoiniciadorCdU{Solicitação de Cadastro ou tentativa de login social com usuário não cadastrado} %Evento iniciador
%\atoresCdU{} %Atores
%%Lista de pré-condições
%\precondicaoCdU{} %Pré-condição 1
%\precondicaoCdU{} %Pré-condição 2
%\precondicaoCdU{} %Pré-condição 3
%%Lista de eventos do caso de uso
%\eventosCdU{} %primeiro evento
%\eventosCdU{} %segundo evento
%\eventosCdU{} %terceiro evento
%Lista de pós-condições
%\poscondicaoCdU{} %Pós-condição 1
%\poscondicaoCdU{} %Pós-condição 2
%\poscondicaoCdU{} %Pós-condição 3
%%Lista de casos de uso de extensão
%\extensaoCdU{} %Caso de extensão 1
%\extensaoCdU{} %Caso de extensão 2
%\extensaoCdU{} %Caso de extensão 3
%%Lista de casos de uso de inclusão
%\inclusaoCdU{} %Caso de inclusão 1
%\inclusaoCdU{} %Caso de inclusão 2
%\inclusaoCdU{} %Caso de inclusão 3
%%Lista de requisitos relacionados/atendidos
%\requisitoCdU{} %Requisitos relacionados/atendidos
%\requisitoCdU{} %Requisitos relacionados/atendidos
%\requisitoCdU{} %Requisitos relacionados/atendidos
%\end{casodeuso}
%%%%%%%%%%%%%%%%%%%MODELO A SER USADO%%%%%%%%%%%%%%%%
%

\section{Tabelas Resumo de Requisitos e Casos de Uso}
\subsection{Tabela de Requisitos Funcionais}\label{subsec:tabResReqF}
   \begin{table}[H]
        \centering
        \caption{Requisitos funcionais}
        \label{tab:reqFunc}
        \PrintRequisitosFunc
    \end{table}
    

\subsection{Tabela de Requisitos Não Funcionais}\label{subsec:tabResReqNF}
   \begin{table}[H]
        \centering
        \caption{Requisitos não funcionais}
        \label{tab:reqNFunc}
        \PrintRequisitosNFunc
    \end{table}

\section{Avaliações}\label{sec:avalia}
\subsection{Avaliações Específicas}\label{subsec:avaliaspec}
Descrever quais as perguntas específicas serão realizadas e o porque delas.

\subsection{Modelo de avaliação}\label{subsec:modavalia}
Descrever os motivos da escolha de uma escala linear contínua;

Descrever se o porque da escolha de uma escala numérica e/ou uma escala não numérica.

\chapter{Metodologia}\label{chp: metodologia}

O Planejamento do projeto e sua concepção seguirão partes do \textbf{PMBOK}, seguindo o planejamento da disciplina \textbf{PCS2511 - Gerência, Qualidade e Tecnologia de Software}, oferecido em paralelo à primeira disciplina de Trabalho de Formatura.

Já com relação a implementação e testes, o projeto será desenvolvido seguindo o conceito de \gls{tdd}. Destaca-se que será fundamental a manutenção de comentários descritivos, principalmente nos testes, mas também no código fonte em si. Dessa maneira, o código fonte servirá como documentação principal do projeto.

Vale ressaltar que o \textit{framework} escolhido (\gls{django}) já possui todo suporte necessário para o desenvolvimento guiado por testes. Outro ponto importante de se destacar é que o \gls{django} também incentiva o desenvolvimento modular - no caso voltado a \textit{apps}.

O desenvolvimento contará também com versionamento de código utilizando \textbf{git} e gestão de \textit{issues} e \textit{milestones} numa única plataforma integrada de gestão de projetos.

Por fim, é fundamental ressaltar que a \sptrans~já oferece diversos recursos aos usuários, como localização de ônibus em tempo real, e que essas informações também são fornecidas via API\footnote{\url{http://www.sptrans.com.br/desenvolvedores/APIOlhoVivo/Documentacao.aspx}}. Dessa maneira pretende-se utilizar o máximo que for possível esta API para fornecer serviços básicos aos usuários, sem que tais tecnologias precisem ser reimplementadas, reduzindo muito a necessidade de armazenamento e tratamento de dados do sistema de transporte dentro da própria plataforma.
\input{./texto/ref-teorico}
\input{./texto/resultados}

% ----------------------------------------------------------
% Finaliza a parte no bookmark do PDF
% para que se inicie o bookmark na raiz
% e adiciona espaço de parte no Sumário
% ----------------------------------------------------------
\phantompart
%\bookmarksetup{startatroot}%
% ---

% ---
% Conclusão
% ---
\chapter*{Conclusão}
\addcontentsline{toc}{chapter}{Conclusão}
\section{Possíveis futuras funcionalidades}\label{sec:futuro}
\subsection{Novos Módulos}\label{subsec:futuro-novos-mod}
 - API da SPTRANS
  -- localizar ônibus
  -- tempo de espera pelo ônibus
  -- tempo previsto de trajeto
  -- ônibus que passam pela localização atual
  -- linhas que fazem um determinado trajeto, dado início e fim
  -- Oferecimento de vantagens tarifárias a usuários que contribuem com avaliações e/ou vinculadas ao desempenho no game

\subsection{Amplianção dos módulos atuais}\label{subsec:futuro-expansao-mod}

Não está no escopo de implementação deste projeto um ou mais módulos que permitam a integração do sistema com prestadores de serviço não ligados ao sistema de transporte. Um exemplo seria a integração com comércios próximos aos pontos de ônibus que poderiam oferecer descontos e ``promoções relâmpago'' para usuários de acordo com suas ``pontuações'' no game do aplicativo, enviando notificações das promoções aos celulares dos usuários próximos ao comércio.


% ----------------------------------------------------------
% ELEMENTOS PÓS-TEXTUAIS
% ----------------------------------------------------------
\postextual


% ----------------------------------------------------------
% Referências bibliográficas
% ----------------------------------------------------------
\bibliography{./monografia}

% ----------------------------------------------------------
% Glossário
% ----------------------------------------------------------
%
% Consulte o manual da classe abntex2 para orientações sobre o glossário.
%
%\glossary

% ----------------------------------------------------------
% Apêndices
% ----------------------------------------------------------

% ---
% Inicia os apêndices
% ---
%%%%%%%%%%%%%%%%%%%\begin{apendicesenv}

% Imprime uma página indicando o início dos apêndices
%%%%%%%%%%%%%%%%%%%\partapendices

%\input{./texto/apendices/Hardware}

%%%%%%%%%%%%%%%%%%%\end{apendicesenv}
% ---


% ----------------------------------------------------------
% Anexos
% ----------------------------------------------------------

% ---
% Inicia os anexos
% ---
%\begin{anexosenv}

% Imprime uma página indicando o início dos anexos
%\partanexos
% ---
%\end{anexosenv}

%---------------------------------------------------------------------
% INDICE REMISSIVO
%---------------------------------------------------------------------

%\phantompart
%\printindex
%\section*{Lista de Acrônimos}
%\printglossary[type=\acronymtype]

%\chapter*{Glossário}

% ---
% Define nome e preâmbulo do glossário
\phantompart
\renewcommand{\glossaryname}{Gloss\'{a}rio}
%\renewcommand{\glossarypreamble}{Esta é a descrição do glossário. Experimente}
% ---
% Traduções para o ambiente glossaries
\providetranslation{Glossary}{Glossário}
\providetranslation{Acronyms}{Siglas}
\providetranslation{Notation (glossaries)}{Notação}
\providetranslation{Description (glossaries)}{Descrição}
\providetranslation{Symbol (glossaries)}{Símbolo}
\providetranslation{Page List (glossaries)}{Lista de Páginas}
\providetranslation{Symbols (glossaries)}{Símbolos}
\providetranslation{Numbers (glossaries)}{Números}
% ---
% Estilo de glossário
\setglossarystyle{index}
%\setglossarystyle{altlisthypergroup}
%\setglossarystyle{tree}
% ---
% Imprime o glossário
%\cleardoublepage
\phantomsection
\printindex
\addcontentsline{toc}{chapter}{\glossaryname}
%\printglossaries

\end{document}